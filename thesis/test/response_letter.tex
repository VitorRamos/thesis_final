% LaTeX rebuttal letter example. 
% 
% Copyright 2019 Friedemann Zenke, fzenke.net

\documentclass[11pt]{article}
\usepackage[utf8]{inputenc}
\usepackage{lipsum} % to generate some filler text
\usepackage{fullpage}

\usepackage{xifthen}
\newcounter{reviewer}
\setcounter{reviewer}{0}
\newcounter{point}[reviewer]
\setcounter{point}{0}

% command declarations for reviewer points and our responses
\newcommand{\reviewersection}{\stepcounter{reviewer} \bigskip \hrule
                  \section*{Reviewer \thereviewer}}

\newenvironment{point}
   {\refstepcounter{point} \bigskip \noindent {Point \textbf{\arabic{point}} -- } \ }
   {\par }

\newenvironment{action}
   {\medskip \noindent \begin{sf}\textbf{Action}:\ }
   {\medskip \end{sf}}

\newenvironment{reply}
   {\medskip \noindent \begin{sf}\textbf{Reply}:\ }
   {\medskip \end{sf}}


\begin{document}

\section*{Response to the reviewers}
% General intro text goes here
We thank the reviewers for their critical assessment of our work. 
In the following we address their concerns point by point. 

% Let's start point-by-point with Reviewer 1
\reviewersection
In this work, the authors propose an analytical model to estimate energy-optimal software configurations and by using DVFS and DPM techniques to decrease the energy consumption of the high-performance computing applications. This paper is well-written. However, it is still can be improved from some perspectives.

% Point one description 
\begin{point}
The Related Work section is too long. Moreoever, the authors fail to concisely indicate that comparing with the existing works, what is the outstanding contribution made by the authors in such a long section. 
\end{point}

% Our reply
\begin{action}
This section was reformulated and split into the theoretical background in section II, and related work in section III where we give a more clear distinction from other studies.
\end{action}

\begin{point}
eq. (2) and eq. (20) should be Eq. (2) and Eq. (20).
\end{point}

\begin{reply}
Thanks for noticing this.
\end{reply}

\begin{action}
Corrected.
\end{action}

\begin{point}
What is the unit of "s" in Eq. (4)? Why "speed" multiplied by "time" is "workload"?
\end{point}

\begin{reply}
The unity of $s$ is instructions/seconds, so we have instructions/seconds*seconds, resulting in instructions, which is something that can represent the workload.
\end{reply}

\begin{action}
The unity of $s$ was added in the text section II.B "…and s is the speed of execution in instructions/seconds…".
\end{action}

\begin{point}
Below Eq. (6), the $"T_{total}"$ seems weird.
\end{point}

\begin{reply}
Indeed, the total should have been subscribed.
\end{reply}

\begin{action}
We reformulated complex nomenclature to simplify the equations.
\end{action}

\begin{point}
Did you see the difference of the "dynamic" in Eq. (7) and in Eq. (8)?
\end{point}

\begin{reply}
They are the same, Eq. (8) is just the definition of it.
\end{reply}

\begin{action}
The Eq. (7) was removed and replaced by the textural explanation "there are three main components of power dissipation,  namely, static power, dynamic power, and leakage power, that accumulated compose the total power draw…" in section IV.A.
\end{action}

\begin{point}
The captions of some FIGUREs are ended by period, some are ended without period.
\end{point}

\begin{reply}
Thanks for noticing this.
\end{reply}

\begin{action}
Corrected.
\end{action}

\begin{point}
FIGURE 11 should be explained more.
\end{point}

\begin{reply}
Indeed, the paragraph that explained the figure was too concise and unclear.
\end{reply}

\begin{action}
This paragraph and the figure caption were reformulated. The following paragraph was added: "By default, operating systems do not implement DPM at the core level, and in HPC, generally, the user explicitly chooses the number of cores to runs his job. To give a better idea of how much DPM at the core level can impact energy consumption, we analyze the choice of the number of cores during one year in the UFRN's HPC center. The result is plotted in figure…" in section V.F.
\end{action}

\begin{point}
The energy model and workload model also can be modeled/estimated by Markov Process. The authors may want to refer to and cite the following article. X. Liu, T. Han and N. Ansari, "Intelligent battery management for cellular networks with hybrid energy supplies," 2016 IEEE Wireless Communications and Networking Conference, Doha, 2016, pp. 1-6.
\end{point}

\begin{reply}
Indeed, we had not considered this type of model  in our related works.
\end{reply}

\begin{action}
We have included this reference in the discussion of the related work.
\end{action}

% Begin a new reviewer section
\reviewersection

This work presents the modeling of 13 parallel applications employed to define energy-optimal configurations for a number of different problem sizes. The results show that approach generates about 10 times less energy overhead for the same level of accuracy.

\begin{point}
A massive number of symbols are introduced in the formation. It will help understanding the optimization by providing a proper nomenclature at the beginning of this paper or somewhere and clearly distinguish control variables, problem parameters and constants.
\end{point}

\begin{action}
All symbols are defined and better explained in the document, as suggested by the reviewer. Abbreviations have been added in the appendix.
\end{action}

\begin{point}
The pseudo codes of algorithms need to be clarified, and the current description is not clear.
\end{point}

\begin{action}
Pseudo codes are defined and better explained in the document as suggested by the reviewer.
\end{action}

\begin{point}
The section of related work is too general, and more studies should be added to comparatively show the contributions of this paper.
\end{point}

\begin{reply}
Thank you for the remark.
\end{reply}

\begin{action}
The related work section has been reformulated and split into an introductory theoretical background, in Section II, and related work, in Section III, where we more clearly distinguish our contribution from other studies.
\end{action}

\begin{point}
Please clarify whether the assumptions of the model are reasonable in reality.
\end{point}

\begin{reply}
As suggested, although the hypotheses are well accepted in the literature, their validation was clarified in the results evaluation section.
\end{reply}

\begin{point}
The authors are suggested to use some real-life datasets (e.g., in the above suggested references) to evaluate the proposed method.
\end{point}

\begin{reply}
We thank the reviewer for the suggestion.
\end{reply}

\begin{action}
Section V "Experimental validation" extends benchmarks to real-life data.
\end{action}

\begin{point}
The authors should discuss in more detail how the algorithm parameters are computed/selected and their dependence on an actual problem. The super-parameter settings for all tested algorithms need more discussions as regards how they affect the results and how sensitive are the results to these settings.
\end{point}

\begin{action}
A discussion on the selection of parameters, and their impact on the model, was incremented in section V.C.
\end{action}

\begin{point}
The complexity of the proposed algorithm should be analyzed.
\end{point}

\begin{reply}
We propose a model based on parameters characterizing any algorithm, rather than a specific algorithm. 
\end{reply}

% Begin a new reviewer section
\reviewersection

This work proposes an analytical model that can be used to estimate energy-optimal software configurations and provide knowledgeable hints to improve DVFS and DPM techniques for single-node HPC applications. However, there are many problems and drawbacks that affect the quality of this paper:

\begin{point}
Please give more details about how DPM optimization problems are solved to obtain both frequency and number of active cores. 
\end{point}

\begin{action}
DPM problems can be modeled in the broadest way, and so can optimization methods. As suggested, we provide an example in section V.F "DVFS and DPM Optimization" showing how our model contributes to the DPM optimization problem with just a search algorithm.
\end{action}

\begin{point}
Many symbols, notations are not well designed, and should be simplified. Single-letters (English or Greek) should be used, e.g., "Ttotal" may be replaced by "bar over T" or big Gamma.
\end{point}

\begin{action}
As suggested, we simplify symbols most of the symbols we could to single letters except the ones already used in the literature, to follow the same notation as the original author.
\end{action}


\begin{point}
Please clearly give the references of many equations, e.g., (12) and (13) if they were proposed by others
\end{point}

\begin{reply}
The eq 12 is proposed by us and that was mentioned in the paragraph. "Thus, the proposed model for one processing core of a multi-core processor is derived by using..." in section IV.A. The eq 13 is a simple deduction that we can do with the described parameters.
\end{reply}

\begin{action}
As requested, we added the citations to clarify if it was proposed by us or not.
\end{action}

\begin{point}
The comparison between the proposed model and SVR is not convincing. There are many advanced prediction methods, e.g., "Dendritic neuron model with effective learning algorithms for classification, approximation and prediction," IEEE Transactions on Neural Networks and Learning Systems, 30(2), pp. 601 - 614, Feb. 2019.
\end{point}

\begin{reply}
Thank you for the remark; the document was updated accordingly. Indeed, we compare it not only to SVR, but to many models, MLP, KNN, Decision Tree for regression, and now we include the reference cited above. Nevertheless, our comparison's primary goal is to show that it can achieve better accuracy with considerably less training and data. All those models mentioned can, in fact, achieve better accuracy but with a lot more data and fine-tuning. This makes it unfeasible to our problem, where to construct an extensive database will end up consuming a massive amount of time and energy.
\end{reply}

\begin{action}
We include the reference cited above and explain in the text why we chose SVR for the comparison in the first paragraph of section V.
\end{action}

\begin{point}
There are many studies proposed to solve similar energy optimization problems, e.g., TTSA: An Effective Scheduling Approach for Delay Bounded Tasks in Hybrid Clouds; Application-Aware Dynamic Fine-Grained Resource Provisioning for Virtualized Cloud Data Centers. Authors could also give some comments on them or add some discussion in the future work on how to apply them to improve the current method.
\end{point}

\begin{reply}
Indeed, the above-mentioned approaches can profit from our model to compute the most effective scheduling. Our model can fit any algorithm that needs a real-world model of application energy.
\end{reply}

\begin{action}
As suggested by the reviewer, those works will be taken into account in future work, as we apply the models to modern DVFS and DPM algorithms.
\end{action}

\begin{point}
After the first sentence in Sec. II, please cite some references, e.g., "Predicting the results of RNA molecular specific hybridization using machine learning," in IEEE/CAA Journal of Automatica Sinica, vol. 6, no. 6, pp. 1384-1396, November 2019; "Forecasting of software reliability using neighborhood fuzzy particle swarm optimization based novel neural network," in IEEE/CAA Journal of Automatica Sinica, vol. 6, no. 6, pp. 1365-1383, November 2019
\end{point}

\begin{action}
As requested, we include these references.
\end{action}

\begin{point}
English presentation needs much improvement, e.g., Abstract, Line -3, "to a standard machine-learning modeling" => "with a standard machine-learning model". Line -2, "about 10x less energy overhead" => "about 10 times less energy overhead".
\end{point}

\begin{action}
We double-checked the grammar.
\end{action}

\end{document}