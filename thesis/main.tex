\documentclass[
papersize=a4,
pagelayout=default,
fontname=latinmodern,
fontsize=11pt,
twoside,
final,
faculty=fpms,
]{umons-Thesis}

% Styles
\headstyles{umons}
\pagestyle{umons}
%\setcounter{secnumdepth}{4}

\usepackage{breakcites}
\usepackage[main=english]{babel}
\usepackage[autostyle=true]{csquotes}
\usepackage[babel=true]{microtype}
\usepackage{float}
\usepackage{graphicx}
\usepackage{subcaption}
\usepackage{booktabs}
\usepackage{mathtools}
\usepackage{siunitx}
\usepackage{multirow}

\sisetup{%
	binary-units = true,%	loads units for binary data (\bit, \byte)
}
\usepackage[%
hidelinks,%				hides links (removing color and border)
pdfusetitle,%			uses the content of \author{}... for metadata
pdfdisplaydoctitle,%	displays document title instead of file name in title bar
]{hyperref}
\usepackage[capitalise]{cleveref}
\usepackage{outlines}
\usepackage{listings}


\definecolor{light-gray}{gray}{1}
\definecolor{codegreen}{rgb}{0,0.6,0}
\definecolor{codegray}{rgb}{0.5,0.5,0.5}
\definecolor{codepurple}{rgb}{0.58,0,0.82}
\definecolor{backcolour}{gray}{1}
\definecolor{commentgreen}{RGB}{2,112,10}

\colorlet{punct}{red!60!black}
\definecolor{background}{HTML}{EEEEEE}
\definecolor{delim}{RGB}{20,105,176}
\colorlet{numb}{magenta!60!black}


\lstdefinestyle{jsonstyle}{
	commentstyle=\color{commentgreen},
	backgroundcolor=\color{light-gray},
	basicstyle=\ttfamily\small,
	breakatwhitespace=false, 
	breaklines=true, 
	captionpos=b, 
	keepspaces=true, 
	numbers=left, 
	numbersep=5pt, 
	showspaces=false, 
	showstringspaces=false,
	showtabs=false, 
	tabsize=1,
	literate={\ \ }{{\ }}1
}

\lstdefinelanguage{json}{
	basicstyle=\ttfamily\small\normalfont,
	numbers=left,
	numberstyle=\scriptsize,
	stepnumber=1,
	numbersep=4pt,
	showstringspaces=false,
	breaklines=true,
	frame=lines,
	tabsize=1,
	literate={\ \ }{{\ }}1,
	backgroundcolor=\color{background},
	morecomment=[f][\color{red}][0]{*},
	literate=
	{:}{{{\color{punct}{:}}}}{1}
	{,}{{{\color{punct}{,}}}}{1}
	{\{}{{{\color{delim}{\{}}}}{1}
	{\}}{{{\color{delim}{\}}}}}{1}
	{[}{{{\color{delim}{[}}}}{1}
	{]}{{{\color{delim}{]}}}}{1},
}

\lstdefinestyle{ccodestyle}{
	language=C,
	commentstyle=\color{commentgreen},
	backgroundcolor=\color{light-gray},
	basicstyle=\ttfamily\small,
	breakatwhitespace=false, 
	breaklines=true, 
	captionpos=b, 
	keepspaces=true, 
	numbers=left, 
	numbersep=5pt, 
	showspaces=false, 
	showstringspaces=false,
	showtabs=false, 
	tabsize=2,
	xleftmargin=0.5cm,
	emph={int,char,double,float,unsigned,return},
	emphstyle={\color{blue}}
}

\lstdefinestyle{pythonStyle}{
	%backgroundcolor=\color{backcolour},
	backgroundcolor=\color{light-gray},
	basicstyle=\ttfamily\small,
	commentstyle=\color{codegreen}\itshape,
	keywordstyle=\color{magenta}\bfseries,
	%numberstyle=\tiny\color{codegray},
	stringstyle=\color{codepurple},
	belowcaptionskip=\medskipamount,
	breakatwhitespace=false, 
	breaklines=true, 
	captionpos=t, 
	keepspaces=true,
	frame=top,frame=bottom,
	framexbottommargin=5pt,
	%framextopmargin=5pt,
	xleftmargin=0.5cm,
	numbers=left, 
	numbersep=5pt, 
	showspaces=false, 
	showstringspaces=false,
	showtabs=false, 
	tabsize=2
}


\lstset { %
	language=C++,
	backgroundcolor=\color{black!5}, % set backgroundcolor
	basicstyle=\footnotesize,% basic font setting
}

\newcommand*\rot{\rotatebox{90}}
\def\BibTeX{{\rmfamily B\kern-.05em{\sc i\kern-.025em b}\kern-.08em
		T\kern-.1667em\lower.7ex\hbox{E}\kern-.125emX}}

\DeclareOldFontCommand{\rm}{\normalfont\rmfamily}{\mathrm}
\DeclareOldFontCommand{\sf}{\normalfont\sffamily}{\mathsf}
\DeclareOldFontCommand{\tt}{\normalfont\ttfamily}{\mathtt}
\DeclareOldFontCommand{\bf}{\normalfont\bfseries}{\mathbf}
\DeclareOldFontCommand{\it}{\normalfont\itshape}{\mathit}
\DeclareOldFontCommand{\sl}{\normalfont\slshape}{\@nomath\sl}
\DeclareOldFontCommand{\sc}{\normalfont\scshape}{\@nomath\sc}

\crefname{lstlisting}{listing}{listings}
%%        %%
% DOCUMENT %
% ======== %


% Metadata
% --------

\author{Vitor \textsc{Ramos Gomes da Silva}}

\date{\today}

\title{Energy-optimal configurations for high-performance computing applications: automated low-impact characterization and performance optimization of shared-memory applications}
%\title{My very long two-line PhD thesis title}
%\title{My supra long PhD thesis title just to verify that nothing is broken because of a long title}

\committeeMembers{%
	Prof. Carlos Alberto \textsc{VALDERRAMA}, Supervisor\\
	Prof. Pierre \textsc{MANEBACK}, Co-supervisor\\
	Prof. Thierry \textsc{DUTOIT}, Chair\\
	Prof. Sidi \textsc{MAHMOUDI}, Chair\\
	Prof. Samuel \textsc{XAVIER-DE-SOUZA}, Chair 
}



% Text
% ----

\begin{document}
	
	\umonsThesisTitlePage
	
	\frontmatter
	
	\begin{umonsThesisAbstract}
		Energy consumption is a key to enabling exascale High performance Computing (HPC). However, energy-optimized hardware and software combinations could still be inefficient if the software operates poorly. 
		
		Software operation relies on dynamic scaling of frequency and voltage (DVFS) and dynamic power management (DPM), but none have priority information on the application, leading to inefficient software operation. 
		
		This work proposes a set of tools, models, and algorithms for energy optimization aimed at high-performance computing based on knowledge of the application and the specific architecture. The main contributions of this work are:
		
		A framework called PascalSuite automatically measures and compares multiple executions of a parallel application according to various scenarios characterized by input arrangements, number of threads, number of cores, and frequencies. As a result, PascalSuite can automate designing application models with an overhead of less than 1\%.
		
		A full-system energy model based on the CPU frequency and the number of cores. The model aims to understand and optimize the energy behavior of parallel applications in HPC systems according to application parameters, such as the degree of parallelism and CPU parameters related to dynamic and static power.
		
		A methodology that combines measurement data with a heuristic algorithm to provide insights into choosing the best phase divisions. Our heuristic can reduce the scan space from $10^{7000}$ to $10^2$ with an average error of 10\% and up to 38\% reduction in energy consumption using optimal distribution compared to standard Linux DVFS.
		A novel normalised time representation of the application serving to characterize the application parameters and model, named application finger print. 
	\end{umonsThesisAbstract}
	
	% \vspace{2cm}
	% \textbf{Keywords :} Energy Model; Dynamic Frequency and Voltage Scaling; Dynamic Power Management; High Performance Computing
	
	\tableofcontents*
	\listoffigures*
	\listoftables*
	
	
	\mainmatter
	
	\chapter{Introduction} \label{chapter:introduction}
	In this chapter, we present the motivations for this work, as well as its objectives and main contributions.
	
\section{Motivation} \label{sec:motivation}
Data center energy efficiency has become of crucial importance in recent years due to its high economic, environmental, and performance impact. For example, the leading petaflop supercomputers consume a range of 1–18 MW of electrical power, with 1.5 MW on average, which can be easily translated into millions of dollars per year in electricity bills \cite{Group2012HandbookSahni}.
Data center energy consumption was estimated to be between 1.1\% and 1.5\% of worldwide electricity usage in 2010 \cite{Dayarathna2016DataSurvey,Corcoran2017EmergingICT}, generating as much pollution as a nation such as Argentina  \cite{Mathew2012Energy-awareNetworks}.
In some cases, the power costs exceed the cost of purchasing hardware~\cite{Rivoire2007ModelsOptimizations}.
Furthermore, the energy costs of powering a typical data center doubles every five years~\cite{Buyya2013Introduction}.
Therefore, with such a steep increase in power use, electricity bills have become a significant expense for today's data centers \cite{Poess2008EnergyCenters,Gao2013QualityCenters}. 
For these reasons, data center energy efficiency is now considered a primary concern for data center operators, often ahead of the traditional considerations of availability and security.

There are several approaches for green computing, from electrical materials to circuit design, systems integration, and software. These techniques may differ, but they share the same goal---to substantially reduce overall system energy consumption without a corresponding negative impact on delivered performance.
The processor and main memory are  the components that usually dominate power consumption, as shown in \cref{fig:powerbreakdown}.
The processor can consume as much as 50\% of the total energy~\cite{Fan2007PowerComputer, Barroso2007TheComputing, Malladi2012TowardsDRAM}. For that reason, modern processors incorporate several features for power management~\cite{Rotem2012Power-managementBridge, Brown2005ACPILinux, Hackenberg2015AnProcessor, Intel20200thLake}, such as dynamic power management (DPM) and dynamic voltage and frequency scaling (DVFS). 
DPM encompasses a set of techniques for obtaining energy-efficient computing by deactivating or reducing the system components' performance when they are idle or partially utilized~\mbox{\cite{Shuja2012Energy-efficientCenters, Benini2000AManagement}}.
DVFS allows the frequency and voltage to be adjusted in run-time depending on current needs.


\begin{figure}[H]
	\centering
	\captionsetup[subfigure]{justification=centering}
	
	\begin{subfigure}[b]{0.45\textwidth}
		\includegraphics[width=\textwidth]{models/figures/power_breakdown/Xeon E5-2698 + DDR4 + SSD.png}
		\caption{}
		\label{fig:powerbreakdown_a}
	\end{subfigure}
	%
	\begin{subfigure}[b]{0.45\textwidth}
		\includegraphics[width=\textwidth]{models/figures/power_breakdown/Xeon E5507 + DDR3 + HDD.png}
		\caption{}
		\label{fig:powerbreakdown_b}
	\end{subfigure}
	
	\hfill
	\caption{Power breakdown of a typical node of an HPC cluster at full use. The system used in this study (\textbf{a}) was built in 2016 and equipped with two Intel Xeon E5-2698, 128 GB of DDR4 memory and SSD as storage, while (\textbf{b}) the case study in \cite{Malladi2012TowardsDRAM} was built in 2012 and equipped with two Xeon E5507, 32GB of DDR3 memory and HDD as storage.}
	\label{fig:powerbreakdown}
\end{figure}

DVFS is motivated by the well-known fact that frequency and power have a near-cubic relationship \cite{Dayarathna2016DataSurvey, Group2012HandbookSahni}; this implies that running the CPU at a lower frequency causes a linear reduction in performance and a near-cubic reduction in power, which could lead to a near-square reduction in CPU energy.
Because of this, it is possible to achieve dramatic energy savings just with frequency control, depending on the system and its architecture.
Although very promising, the system software has yet to determine when and what voltage and frequency to use when running applications.
Otherwise, not only will performance deteriorate, but, in the worst case, energy consumption would also increase  \cite{Group2012HandbookSahni}.
Indeed, reducing the frequency results in a longer execution time, which increases the energy consumption of other system components, such as memory and disks.
There is also an overhead of time and energy associated with a voltage and frequency switch that needs to be considered.
Thus, finding the most appropriate voltage and frequency to use in all circumstances is not easy.
Therefore, since its introduction in 1994 \cite{Group2012HandbookSahni}, there has been a tremendous amount of research on DVFS algorithms.

The DPM technique can achieve substantial energy savings on systems where the static power is high, or the system remains inactive for a long time.
In that case, the problem is to determine when and which components to turn on/off.
With DPM, energy savings of 70\% have been reported~\cite{Shuja2012Energy-efficientCenters, Benini2000AManagement}. 

However, at the same time, while these power-saving techniques reduce system energy, they can compromise  performance leading  to a complex trade-off that needs to be carefully exploited to produce more energy-efficient algorithms.
Indeed, this study investigates whether the construction of an energy consumption model of an application can lead to significant energy savings.

We propose an analytical energy model for a given application in the function of the two control variables present in most HPC systems: CPU operating frequency and number of active cores. The model is composed of three application-dependent parameters and three parameters relating to the architecture of the system. The application parameters incorporate characteristics of the percentage of parallelism and the input size. The system architecture parameters include power-related and technology-dependent components, such as dynamic, static, and leakage power.

\section{Objectives} \label{sec:objectives}


\section{Contributions*} \label{sec:contributions}

Although much work has been done on DVFS, the focus is still on the consumer electronics and laptop markets. 
For HPC, the notion of energy perception is relatively new~\cite{Feng2003MakingSupercomputing}. 
Moreover, the operational characteristics of non-HPC and HPC systems are significantly different. 
First, the workload on non-HPC systems is very interactive with the end-user, but the workload on the HPC platform is not.
Second, activities conducted on a non-HPC platform tend to share more machine resources.
In contrast, in HPC, each job often runs with dedicated resources. 
Third, an HPC system is usually much larger than a non-HPC system, making it more challenging to gather information, organize, and execute global decisions.
Therefore, it is worthwhile to investigate whether a DVFS scheduling algorithm, which works well for conventional computing, remains effective for HPC.

Our paper proposes a full-system energy model based on the CPU frequency and the number of cores. The model aims to understand and optimize the energy behavior of parallel applications in HPC systems according to application parameters, such as the degree of parallelism and CPU parameters related to dynamic and static power. The proposed model differs from existing ones, including the frequency and number of cores in the same equation for estimating the energy for a specific application in a given configuration. This model can serve as a base for considering DVFS and DPM optimization problems, including frequency and active cores. It can also be used to analyze the contribution of each parameter (ex: level of parallelism) to energy consumption. Furthermore, the number of cores is essential in HPC since applications are designed to run on multiple cores.

The proposed energy model is the product of an application-agnostic power model and an architecture-specific application performance model. The power model is based on the CMOS logic gates power draw as a function of the frequency~\cite{Sarwar1997CmosCalculation, Butzen2007LeakageGates} augmented to include the number of cores. The performance model is based on Amdahl's law \cite{Amdahl1967ValidityCapabilities, Eyerman2010ModelingDesign, Gustafson1988ReevaluatingLaw}, which can be used to estimate runtime in multi-core systems. In addition, this model has been extended to include execution frequency and input size, characterizing the application on the target architecture.


\cref{tab:related_work} summarizes the models comparing the system dependencies and the controllable variables.
\begin{table}[H]
	\footnotesize
	\caption{Related work summary.\label{tab1}}
	\setlength{\tabcolsep}{2.3mm}\begin{tabular}{cccc}
		\toprule
		%\begin{adjustwidth}{-\extralength}{0 cm}
		
		\textbf{Model}  & \textbf{System Dependency}       & \textbf{Variable}                     & \textbf{Controllable Variables}         \\ \midrule
		Merkel et al. \cite{Merkel2006BalancingSystems} & performance counters    & number of activities & - \\
		Roy et al. \cite{Roy2013AnAlgorithms}&performance counters &io operations, total time    & - \\
		Yakun et al. \cite{Shao2013EnergyProcessor}&number of instructions &frequency&frequency \\
		Lewis et al. \cite{Lewis2008Run-timeSystems}&energy of subcomponents&energy of subcomponents & - \\
		Mills et al. \cite{Mills2014EnergySystems}&power of subcomponents&total time, frequency & frequency \\ 
		Our model & - & frequency, cores, input size & frequency, cores, input size\\ \bottomrule
	\end{tabular}
	%\caption{Related work summary}
	\label{tab:related_work}
	%\bottomrule
\end{table}

The main contributions of the proposed model are:


% Contribuições por capitulo...
\begin{itemize}
	\item Simple model: faster to fit and compute, good for DVFS and DPM optimization.
	\item Parameters with logical meaning: helps to understand the contribution of each specific term.
	\item Analytical analysis: several analyses can be derived from the equation.
	\item Controllable variables: the equation is in the function of parameters that we can control directly.
\end{itemize}

\section{Organization}
	
	\chapter{Theoretical background} \label{chapter:theoretical_background}
	This chapter will give the reader a better understanding of some terms used throughout the thesis, as well as details of configurations, the chosen benchmarks, and the architecture of the case study.
	\section{Frequency control} \label{sec:frequency_control}

In modern systems, control over the processor frequency can be done by hardware with independent circuits as well as by software. For that, there is the Advanced Configuration and Power Interface (ACPI) an open standard adopted by operating systems to configure hardware components related to power management.

In the ACPI two important states for the DVFS are defined that optimize the energy consumption. They are "C", which is activated when the processor is not executing any instructions, and "P", which is activated while the processor is operating. These states have several levels and at each level the frequency and voltage are changed.

State P starts at level P0, where frequency and voltage are the maximum possible, then P1, where both decrease, until reaching the last state, Pn, where frequency and voltage are the lowest possible. The change of state depends on the level of utilization of the processor. To stay in each state, the level of processor usage must be within specific limits. After exceeding these limits for a certain time, the state will change to the next state corresponding to that new level of processor usage. The number of possible states depends on each manufacturer.

After an idle time, the processor begins to activate C states, starting with C0 where it is still fully active, then moving to C1, where some features are disabled, up to Cn, where all possible features are disabled. The ACPI standard establishes the functionalities that can be disabled between level C1 and level C3, as seen in Table \ref {tab:states_c}. The other levels are specific to each manufacturer.

\begin{table}[H]
	\centering
	\begin{tabular}{| l | l | l |}
		\hline
		Mode & Name & Functionality \\ \hline
		C0 & operating state & Active processor \\ \hline
		C1 & Halt & Stop executing instructions \\ \hline
		C2 & Stop-Clock & Disable the internal clock \\ \hline
		C3 & Sleep & Disable cache coherence \\ \hline
	\end{tabular}
	\caption{C states}
	\label{tab:states_c}
\end{table}

In state C, the higher the level the greater the energy savings, but returning to the fully functional level is more difficult. In states P, there is a trade-off between performance and energy savings. \cref{fig:p_state} best illustrates the change of states, in which we can see which parts of the circuits are deactivated in states C, the latency to return to the active state, power consumption and also shows the relationship of states P with often.

\begin {figure} [H]
\centering
\begin{subfigure}[t]{0.5\textwidth}
	\centering
	\includegraphics[width=\columnwidth]{intro/figures/c_states.png}
	\caption{C states}
\end{subfigure}%
~
\begin{subfigure}[t]{0.5\textwidth}
	\centering
	\includegraphics[width=\columnwidth]{intro/figures/p_states.jpg}
	\caption{States P}
\end{subfigure}
\caption {Illustration of states C and P} {Altered image from \protect \url {https://www.thomas-krenn.com/en/wiki/Processor_P-states_and_C-states}}
\label{fig:p_state}
\end{figure}


% Are states C and P orthogonal, do they operate independently?

For this management, the operating system provides in the user space a way to control the frequency. This work used an operating system that has Linux as its core.

Linux is compatible with several modern architectures and is widely used in servers, smartphones and supercomputers. It was based on the UNIX system which has the philosophy of treating everything on the system as a file, including settings and input and output devices, such as keyboard, mouse and hard drive. Another important feature is that it is modular and parts of the system can be loaded or removed during execution.

On Linux there are several frequency management options \cite {Brown2005ACPILinux}. The main ones are acpi-cpufreq, Intel P-state, AMD powernow. In this work, acpi-cpufreq is used, which is standard and allows direct control of frequency through system files. Acpi-cpufreq is a Linux module that uses implemented policies that dynamically decide the frequency to be used. Some of these policies are:

\begin{itemize}
\item Performance - configured as often as possible
\item Powersave - configured as low as possible
\item Userspace - the user chooses the frequency to be used
\item Ondemand - controls the frequency depending on the processor load. When the load increases the frequency also increases accordingly.
\item Conservative - similar to Ondemand but more smoothly, the frequency increase is continuous instead of jumping.
\end{itemize}

\section{Power consumption monitoring} \label{sec:power_consumption_monitoring}
The Running Average Power Limit (RAPL) and Intelligent Platform Management Interface (IPMI) interfaces were used to measure the power consumed. described in \cite{IPMI2013ConfigurationGuide}.

\subsection{IPMI}
IPMI \cite{IPMI2013ConfigurationGuide} is a set of specifications for autonomous subsystems that provides processor, firmware and operating system independent management and monitoring. The use of IPMI allows system administrators to avoid having to travel to the server location, which is often far away, to perform their tasks. Also, servers are located in places with low temperature and with a lot of noise due to the ventilation system, and one should avoid spending too much time in these places. With remote management, it is possible to turn the system on and off, remotely access the (BIOS) and reinstall the system in case of any serious failure.

\begin{figure}[H]
\centering
\includegraphics[height = 7.5cm] {intro/figures/IPMI-Block-Diagram.png}
\caption{IMPI diagram} {Image taken from \protect \url {https://pt.wikipedia.org/wiki/Intelligent_Platform_Management_Interface}, the main components of IPMI and how they communicate are shown}
\label{fig:IPMI}
\end{figure}

Access to the IPMI network can be done using the HTTP protocol or a tool made available by the manufacturer (ipmitool), which also performs access via the network. It is also used to monitor the status of the platform with a set of sensors coupled with system temperatures, voltages, fans and power supplies.

\subsection{RAPL}

Modern Intel microprocessors, based on the SandyBridge architecture, include the RAPL \cite {Rotem2012Power-managementBridge, Hahnel2012RAPL, Hackenberg2015AnProcessor} interface designed to limit the use of energy on a chip while ensuring maximum performance. This interface supports energy measurement capabilities through an integrated circuit that estimates energy use based on a model driven by architectural event counters for all components. It also provides temperature readings and current leak models. Estimates are made available in model-specific registers (MSR), updated in milliseconds. The energy estimates offered by RAPL were validated by Intel, which showed excellent results.

\section{Case-Study architecture} \label{sec:casestudyarchitecture}
The experiments were executed in one computer node equipped with two Intel Xeon E5-2698 v3 processors with sixteen cores each and two hardware threads for each core. 
The overall view of the architecture is shown in \cref{fig:architecture}.
The maximum non-turbo frequency was 2.3 GHz, and the total physical memory of the node was 128 GB (8 $\times$ 16 GB). Turbo frequency and hardware multi-threading were disabled during all experiments. The operating system used was Linux CentOS 6.5, kernel 4.16. 

\begin{figure}[H]
\centering
\includegraphics[width=\columnwidth]{models/figures/architecture.png}
\caption{Node architecture (the image was made with the lstop application).}
\label{fig:architecture}
\end{figure}

The Linux kernel has many different policies for power management, depending on the driver. In the default driver, the acpi-cpufreq, the options are Powersave, Performance, Ondemand, Conservative, and Userspace. Each governor has a policy on how the frequency is selected. In this investigation, the frequency control was performed using the Userspace governor, which allows the user or any userspace program to set the CPU to a specific frequency. The core control was accomplished by modifying the appropriate system files with the default CPU-hotplug driver.

The architecture was equipped with the intelligent platform management interface (IPMI), a set of interfaces allowing out-of-band management of computer systems and platform-status monitoring via the local network~\cite{Schwenkler2006IntelligentInterface}. It can monitor variables and resources, such as the system's temperature, voltage, fans, and power supplies, with independent sensors attached to the hardware.

\section{Case-Study applications} \label{sec:casestudyapplication}
The applications blackscholes, bodytrack, canneal, dedup, fluidanimate, freqmine, raytrace, swaptions, vips and x264 from the PARSEC \url{https://parsec.cs.princeton.edu/download.htm} (accessed on 20 February  2020) parallel benchmark suite, version 3.0~\cite{Bienia2008TheSuite}, OpenMC \cite{Romano2015OpenMC:Development} and LINPACK (HPL) \cite{Dongarra1988TheExplanation}, were chosen as case studies. The PARSEC benchmark focused on emerging workloads and was designed to represent the next-generation shared-memory programs for chip-multiprocessors. It covers an ample range of areas, such as financial analysis, computer vision, engineering, enterprise storage, animation, similarity search, data mining, machine learning, and media processing. The OpenMC and the LINPACK are two classic HPC programs.

%The other two experiments use applications Raytrace~\cite{Bienia2008} and OpenMC~\cite{Romano2015OpenMC:Development}. Raytrace is an application addressed on rendering photorealistic scenes. It is available on the PARSEC Benchmark suite, version 2.0. The PARSEC benchmark, focused on emerging workloads, represents the next-generation shared-memory programs for chip multiprocessors. It covers a wide range of areas such as financial analysis, computer vision, engineering, enterprise storage, animation, similarity search, data mining, machine learning, and media~processing.
%
%\textls[-20]{{OpenMC is an application that implements the Monte-Carlo method to simulate the transport of neutrons and photons. It is a classic program aimed at high-performance computing.}}


	
	\chapter{Parallel Scalability Suite} \label{chapter:pascal_suite}
	High-performance computing systems have become increasingly dynamic, complex, and unpredictable. To help build software that uses full-system capabilities, performance measurement and analysis tools exploit extensive execution analysis focusing on single-run results. Despite being effective in identifying performance hotspots and bottlenecks, these tools are not sufficiently suitable to evaluate the overall scalability trends of parallel applications. Either they lack the support for combining data from multiple runs or collect excessive data, causing unnecessary overhead. 
	In this chapter, we present a tool for automatically measuring and comparing several executions of a parallel application according to various scenarios characterized by the input arrangements, the number of threads, number of cores, and frequencies.
	Unlike other existing performance analysis tools, the proposed work covers some gaps in specialized features necessary to better understand computational resources scalability trends across configurations.
	In order to improve scalability analysis and productivity over the vast spectrum of possible configurations, the proposed tool features automatic instrumentation, direct mapping of parallel regions, accuracy-preserving data reductions, and ease of use.
	As it aims at accurately understanding scalability trends of parallel applications, detailed single-run performance analyses show minimal intrusion (less than 1\% overhead).
	
	\section{PascalAnalyzer} \label{sec:pascal_analyzer}
High-performance computing systems have become increasingly dynamic, complex, and unpredictable. To help build software that uses full-system capabilities, performance measurement and analysis tools exploit extensive execution analysis focusing on single-run results. Despite being effective in identifying performance hotspots and bottlenecks, these tools are not sufficiently suitable to evaluate the overall scalability trends of parallel applications. Either they lack the support for combining data from multiple runs or collect excessive data, causing unnecessary overhead. 
In this work, we present a tool for automatically measuring and comparing several executions of a parallel application according to various scenarios characterized by the input arrangements, the number of threads, number of cores, and frequencies.
%In this work, we present a tool for measuring and comparing several executions of a parallel application automatically with different input arrangements, number of threads, cores, and frequencies to understand its scalability by observing how it uses the available computational resources across configurations. 
Unlike other existing performance analysis tools, the proposed work covers some gaps in specialized features necessary to better understand computational resources scalability trends across configurations. 
%The proposed work covers some gaps and unspecialized features of existing performance analysis tools. 
%The tool features automatic instrumentation and direct mapping of parallel regions, accuracy-preserving data reductions, and ease of use to compare multiple configurations for improving analysis and productivity. 
In order to improve scalability analysis and productivity over the vast spectrum of possible configurations, the proposed tool features automatic instrumentation, direct mapping of parallel regions, accuracy-preserving data reductions, and ease of use.
%In addition, it aims at parallel scalability, instead focus on single-run performance, aggregating a minimal level of intrusion (<1\% on average), which is fundamental for accurately understanding the behavior and scalability trends of the parallel application.
As it aims at accurately understanding scalability trends of parallel applications, detailed single-run performance analyses show minimal intrusion (less than 1\% overhead).

\section{Introduction to profiling tools} \label{sec:introduction_to_profiling_tools}

Developing parallel programs capable of exploiting the computational power of high-performance systems is as well-known as it is challenging~\cite{Huck2007, Islam2019, Weber2019}. 
During the development process, the code optimization step is a fundamental part of the software construction strategy and is supported by performance measurement and analysis tools~\cite{Bergel2019, Weber2019, Huck2005, Geimer2010, Shende2006, Adhianto2010, Miller1995, Galobardes2015, Pillet2007, Islam2019}. The main objective of these tools is to help developers understand the execution characteristics, allowing the identification of bottlenecks and behavioral phenomena that compromise the program’s efficiency, thus guiding possible improvements~\cite{Brink2020, Huck2007}.

Nowadays, due to the complexity of parallel systems, correctly identifying and locating performance and scalability bottlenecks depends on the developer's ability to compare several measurements in different execution configurations~\cite{Bergel2019, Silva2018}. This investigative process tends to be tedious and complex, requiring in-depth knowledge of the problem domain, parallel systems, and measurement and analysis tools. 
%Mastering the analysis tools becomes necessary mainly because they differ from each other in terms of measurement strategy, the collected metrics, and the many points and focus of observation. 
Mastering the analysis tools is necessary because they differ from each other in terms of measurement strategy, metrics collected, and the many points and focus of observation.
Even though some solutions propose organizing and combining metrics from different sources, these distinct characteristics can compel developers to exploit more than one tool depending on their analysis~interests.

%Despite their differences, every existing tool has its own contributions in helping to understand the inner working of a program, and some can present a vast and varied set of metrics with a high degree of detail. 
Every existing tool has its own contributions in helping to understand the inner working of a program. Some can present a vast and varied set of metrics with a high degree of detail. 
However, developers may have difficulty using these tools when their interest is in analyzing parallel scalability. 
%The main challenge arises from the focus on investigating a single run, because scalability analysis requires contrasting runs of different configurations. 
The main challenge arises from the emphasis on studying a single run because scalability analysis requires contrasting runs of different configurations.
%In this sense, it tends to be more significant to measure and compare only a few data across executions of multiple configurations than to collect a vast amount of finer-grain data of single-configuration runs.
In terms of productivity, measuring and comparing only a few data between runs of multiple configurations tends to be more significant to scalability analysis than collecting a vast amount of finer-grain data of single-configuration runs.
Furthermore, due to the number of measurements, there is a natural overhead associated with the execution of the analysis tool itself. Some works indicate this overhead can reach 40\% of the program runtime, which influences and damages the observation of its efficiency variation~\cite{Eriksson2016}.
As such, a measurement and analysis approach that uses only the data needed to understand the application's efficiency variation can be advantageous. First, because measuring and collecting only the necessary data limits the degree of intrusion and, second, because this strategy makes the analysis tool simpler and easier to use. 
%Finally, the tool's availability can also be a drawback as some apply to specific compilers and execution frameworks. Indeed, the framework version is a limitation for some traditional tools like HPCToolkit~\cite{Adhianto2010} and TAU (Tuning and Analysis Utilities)
%MDPI: Is this an abbreviation? If affirmative, please add the explanation. Please carefully check and confirm -- INCLUDED ON THE NEXT PARAGRAPH
~\cite{Shende2006}.


From this context, we present a measurement tool that focuses on analyzing the parallel scalability of programs. It primarily provides runtime and power consumption measures in an approach that favors inspection of the overall behavior of the program before starting a more in-depth investigation of points that naturally require more detail, time and effort.
The work explores and combines strategies already used in other solutions, such as tracing, post-mortem analysis, and automatic instrumentation based on shared libraries. Additionally, it includes an instrumentation model that automatically relates parallel code regions to analysis regions, using a measurement strategy based on the intrinsic and primary behavior of the parallel frameworks. Due to this approach, the tool allows analyzing programs developed for shared-memory environments regardless of the compiler or framework versions. 
The framework version, which in some cases also requires specific compiler versions, is a limitation for some traditional tools like HPCToolkit~\cite{Adhianto2010} and TAU (Tuning and Analysis Utilities)~\cite{Shende2006}.

The proposed tool uses a tracing-based measurement technique to identify and measure the boundaries of the analysis regions. The tracing approach supports a hierarchical region analysis model allowing users to see how the efficiency of inner parts can impact the program's overall scalability. In addition, the tool uses an aggregation feature to reduce the volume of data produced while preserving accurate analysis capability with minimal overhead. The tool also includes usability features that reduce the developer's efforts during the measurement and analysis process. These features allow one to automate the process of running and merging data collected from multiple configurations, bypassing manual efforts and avoiding the drawbacks of non-computer-oriented analysis. 
%The tool does not display visuals elements natively. Still, its interface allows the collected data to be exported and viewed from the control terminal in tabular form. Data exported via the interface are organized into data frames that can be interpreted by other tools or even rendered directly from graphic libraries.
The collected data, available from the terminal in tabular form, can be exported as data blocks that can be interpreted by other tools or even rendered by graphics libraries to facilitate the visualization and data analysis.


%The proposed tool is an alternative for realizing parallel scalability analysis more efficiently than single-run-centric performance measurement and analysis tools. 
We offer an alternative tool for realizing parallel scalability analysis more efficiently than single-run-centric performance measurement and analysis tools. In addition to that, our framework can also be used to promote energy-aware software; methods that rely on accurate and efficient energy collection could benefit from energy consumption data for the entire program and specific regions of the application \cite{10.1007/978-3-319-58667-0_22,10.1007/978-3-319-58667-0_21,7016382,10.1145/3321551}. To summarize, the main contributions of this work are:

\begin{itemize}
	\item A lightweight automated process of measuring and comparing runs with different configurations.
	\item Transparent linking of parallel code sections to analysis regions.
	\item Hierarchical views and degrees of efficiency for the inner parts of a program.
	\item The imposed overhead is low and minimally intrusive, mainly because of the measurement approach and the number of metrics collected.
	\item It provides energy consumption measures for the different configuration parameters. 
\end{itemize}

The following sections describe how the tool is built and used for performance analysis. Section \ref{sec:state_of_the_art_profiling_and_tracing_tools} presents the related works. Section \ref{sec:pascal_architecture} describes the tool architecture and goals; it explains the main features, usage, and collected data. The experimental results are presented in Section \ref{sec:pascal_framework_validation}. Finally, the contributions are summarized in Section \ref{sec:conclusions_pascal}, with an outlook of future works.

\section{State of the art: profiling and tracing tools} \label{sec:state_of_the_art_profiling_and_tracing_tools}

Performance analysis aimed at code optimization is an essential activity for developing parallel applications with scalable performance. For this purpose, performance analysis tools, like Caliper~\cite{Boehme2016}, HPCToolkit~\cite{Adhianto2010}, Scalasca~\cite{Geimer2010}, TAU~\cite{Shende2006}, Vampir~\cite{Weber2019}, and VTune~\cite{Intel2021Vtune}, are useful because they help developers identify where the application uses the available computing resources inefficiently through collecting detailed data of its execution. Such tools differ from each other mainly in terms of data measurement and analysis strategy: tracing or profiling, post-mortem or real-time; individual or comprehensive observation focus; and for providing or not providing visual elements that facilitate the judgment of collected data. Other solutions, like the DASHING framework~\cite{Islam2019} and the HATCHET library~\cite{Brink2020}, allow inspection of performance data from multiple sources. In this case, the goal of DASHING and HATCHET is to provide a more robust data set to support the analysis process.

In the performance analysis process, observing the parallel program's execution as a sequence of events representing significant activities is essential to understand its behavior~\cite{Pantazopoulos1997}. Events are basic units of the analysis process, and the way they are observed influences strategies for collecting performance data. Profiling-based analysis tools collect information from events that occurs in the program execution and commonly operate by statistical sampling using interruptions. These interruptions can be caused, for example, by periodic breaks or hardware events. Interruptions allow checking the system state and the information collected depends on the focus of observation. Profiling-based tools use statistical techniques to describe the program behavior in terms of aggregate performance metrics. They usually ignore the chronology of events, but are known to be useful for identifying, for example, load imbalance, high communication time, or excessive routine calls. Paradyn~\cite{Miller1995}, and Periscope~\cite{Gerndt2010} are tools that adopt this strategy.

On the other hand, tracing-based analysis tools collect performance data from events that occur when the program takes over a particular state. Tracing can provide valuable information about the time dimension of a program, allowing users to check when and where transitions of routines, communications, and specific events occur. This measurement strategy tends to be more invasive and intrusive, generating a larger dataset than the profiling-based approach. Vampir~\cite{Weber2019} and Paraver~\cite{Labarta2005} are examples of this group. Some tools like HPCToolkit~\cite{Adhianto2010}, Scalasca~\cite{Geimer2010}, TAU~\cite{Shende2006} and VTune~\cite{Intel2021Vtune} support both profiling and~tracing.

%In this work we propose tracing to measure the program and identify when parts of its code were executed. 
In this work we propose tracing to identify when parts of a code were executed. 
%To mitigate some of the disadvantages usually related to tracing, the tool implements features such as automatic instrumentation mode and aggregation resource, in addition to adopting an analysis based only on time. 
To mitigate some of the disadvantages related to tracing, in addition to adopting a time-based analysis, the tool implements features such as automatic instrumentation mode and resources aggregation.
%The automatic instrumentation mode makes it possible for users to analyze the program's execution without modifying the compiled code. 
Our automatic instrumentation mode allows users to analyze program execution without modifying the source or the executable.
Although many other analysis tools use the automatic instrumentation mode, this work presents a different way of mapping and measuring the code parts. 
%In our proposal, parallel regions are the main focus of analysis, and there is no limitation on the version of OpenMP used in the development, as in tools such as HPCToolkit~\cite{Adhianto2010} and TAU~\cite{Shende2006}. 
In our proposal, parallel regions are the main focus of analysis. 
%The strategy is efficient in identifying the scalability trend of parts or the whole parallel program in execution.
The strategy is effective enough to identify the scaling trends of parts or of the entire running parallel program.
%Furthermore, collecting only data related to runtime reduces the associated overhead with the measurement process. 
Furthermore, collecting only specific runtime data reduces the overhead associated with the measurement process.
As with other similar tools, such as HPCToolkit~\cite{Adhianto2010} and TAU~\cite{Shende2006}, there is no limitation on the version of OpenMP~used.

Runtime overhead is an attribute associated with the set of additional instructions that are executed to collect the program measurements. The time to perform this ``extra code'' varies in line with the tool and is directly related, in quantity and degree of detail, to the aspects it measures. 
Limiting this overhead is crucial because it can compromise the understanding of program behavior, and divert optimization efforts to less effective points. 
According to some comparative and practical studies in this regard, tests on traditional tools have shown a runtime overhead ranging from 2\% to 40\%~\cite{Eriksson2016}.
%Fundamental to the proposal of this work, the overhead presented by the proposed tool differs according to the program under analysis but can be considered negligible (<1\%) for analyzing scalability trends of parallel applications.
Although different depending on the program analyzed, the overhead resulting from our instrumentation strategy can be considered negligible (less than 1\%) and optimal for analyzing trends in parallel applications.

Several performance analysis tools employ a post-mortem approach. 
%In this kind of approach, the tools collect performance metrics during the program execution, but the data interpretation is carried out only after its execution. 
In this case, the tool performs performance measurements and data collection while the program is running, and then the collected data will be interpreted.
HPCToolkit~\cite{Adhianto2010}, Scalasca~\cite{Geimer2010}, TAU~\cite{Shende2006} and VTune~\cite{Intel2021Vtune} are examples of post-mortem performance analysis tools. 
%Post-mortem performance analysis tools may require the storage of large volumes of performance data, but they are more suitable to provide an overall view of the execution. 
Depending on the type of analysis, these tools may require storing large amounts of data, but they are best suited to provide an overall view of the execution.
%On the other hand, real-time analysis tools enable performance analysis during the execution of the program. 
In contrast, run-time analysis tools perform both operations at run time.
This approach allows detecting waiting states and communication inefficiencies accurately. 
%However, an online analysis environment usually requires the coordinated usage of extra tools, which increases the analysis infrastructure complexity. 
Periscope~\cite{Gerndt2010} and Paradyn~\cite{Miller1995} are examples of tools in this group.
However, run-time analysis generally requires the coordinated action of tools, which increases the structural complexity.
%The need for synchronous initialization, communication between all resources used in the analysis, and the additional load on the network are disadvantages of the real-time approach. 
The need for synchronous initialization and communication between analysis resources as well as their impact on the runtime environment are the drawbacks of this approach.
%In addition, the analysis that depends on data collected at the end of the program's execution, as in the case of scalability analysis, are potentially damaged by the infrastructure overhead and do not, in practice, benefit from characteristics of real-time tools. 
In addition, the analysis, particularly that of scalability, which relies on the data collected at the end of the execution, is potentially degraded by the infrastructure overhead and therefore does not benefit from the characteristics of real-time tools.


\textls[-15]{Energy management solutions can also provide similar features for example GEOPM~\cite{10.1007/978-3-319-58667-0_21},} allows the measurement of time and energy of specific regions. The main advantage that we could not find in any other framework is that our tool can provide specific data of energy consumption of parallel regions in a completely automated way. For example, in GEOPM, it is necessary to instrument the source code to obtain the same result.

%To investigate the scalability trend, it is more important to identify the program's general behavior than its state at a given moment of execution. 
%Although the current performance analysis tools are helpful and allow to identify numerous aspects regarding the program's behavior, none of them presents characteristics directly focused on observing the scalability trend of parallel code. An indispensable feature for this purpose includes automating the program's measurement process in different configurations because a more accurate understanding of scalability requires measuring and comparing multiple code runs.
%Focused on more practical and objective analysis, we propose an alternative tool to observing a parallel program's scalability trend. It collects only information that allows deriving the overall program runtime, or parts of it, according to the user's interest, and includes relevant functionalities to an efficient measurement and analysis process. Among the supported features it provides: the automation of program runs based on different parameters; the generation of data sets that contain metrics with different execution parameters; and the reduction of the generated data volume on disk.

%Although the current performance analysis tools are helpful and allow to identify numerous aspects regarding the program status, to investigate scalability trends, it is more important to identify the behavior of parallel code rather than its state at a given moment of the execution, therefore the post-mortem approach is more suitable. 
By focusing on objective analysis, we offer an alternative tool to observe the scalability trend of a parallel program. Thus, we collect just the information allowing us to infer the overall behavior of the program, or, according to the interest of the user, specifically chosen parts of it, including features relevant to an effective measurement and analysis process. 
To this end, it is essential to also automate the processes to obtain comparative measurements of several runs. 
Supported features are the automation of program executions according to various configuration parameters, including a user-specified granularity, the generation of datasets containing comparative measurements of runtime parameters, and efficient optimization of the amount of produced data.
%In this work, the tool mitigates the disadvantage related to the volume of data set by aggregating the collected data.
Regarding the amount of data collected, our approach uses data aggregation, which facilitates storage and further analysis.


%\section{Design and Features} \label{sec:design_and_features}
%
%This section presents the architecture, components, and interconnections used in the design of the proposed tool. It also includes a discussion about main features, instrumentation modes, and output file structure that contains the data measured and collected by the proposed tool.

\section{PascalAnalyzer architecture} \label{sec:pascal_architecture}
The tool we propose to measure, combine and compare multiple runs of a parallel application implements two main concepts to achieve this goal: actuators and sensors.

Actuators characterize the parameters we intend to observe. They are variables representing elements external to the program and which, when changed, can influence aspects such as the performance and efficiency of the running application.~Therefore, analyzing the result of actuators' variation is essential to understand how these elements impact program behavior, especially scalability and power consumption. By default, the tool implements actuators controlling the number of active cores and threads, the program input, and the CPU operating frequency.

The sensor is the concept we use to represent the elements addressed to measure and monitor the variation of actuators. Both sensors and actuators are plugged into the tool core as modules. Currently, the tool implements three types of sensors:

\begin{itemize}
	\item Begin/end: collects data at launch and end of each program run.
	\item Time sample: periodically collects data.
	\item Events-based: collects data when a specific event happens.
\end{itemize}

Currently, there are sensors to measure time, energy, and performance counters. The default sensor is a \textit{begin/end} type used to collect the execution time of the whole application. To measure energy consumption, we developed sampling sensors capable of retrieving data from RAPL (Running Average Power Limit) and IPMI (Intelligent Platform Management Interface), which are interfaces that provide power information from the CPU
%MDPI: Please add the explabnation % added in abbreviations
and the entire system. Apart from that, there are \textit{time sample} and \textit{begin/end} sensors to gather performance counters data.

Measurements that support scalability analysis are taken from event-based sensors. For this, the tool includes markers that trigger events to automatically identify the boundaries of parallel regions defined by developers in codes that use POSIX Threads or OpenMP.
%MDPI: Are these abbreviations? If affirmative, please add the explanations. Please carefully check and confirm % added in abbreviations
This feature is essential for relating parallel code sections to analysis regions and measuring execution times directly from binary code. The triggers for this measurement mode are implemented in a wrapper library and therefore already available in the system. The tool also provides manual marks that can be inserted into the source code to monitor specific parts of the program. Furthermore, any system event can be set as a trigger.
\cref{fig:pascal_architecture} describes the integration of each part of the proposed software. The idea is that actuators and sensors are modular parts that can easily be added or removed. The tool's core is responsible for all operations, launching the application, and data gathering.

\begin{figure}[H]	
	%\includegraphics[width=15 cm]{Definitions/logo-mdpi}
	\includegraphics[scale=0.7]{pascalanalyzer/figures/designfeatures/pascal_philosophy.pdf}
	\caption{Architecture showing the interconnections of the central parts of the tool. Either binary (using the wrapper library) or an instrumented source code, the target application can be launched on the target platform by the tool core following the configuration parameters chosen by the user while deploying actuators/sensors. The resulting data collection is stored in Json file format for post analysis and visualization (GUI). \label{fig:pascal_architecture}}
\end{figure} 

\section{Instrumentation and Intrusiveness} \label{sec:instrumentation_and_intrusiveness}

The instrumentation module is one of the most crucial. In addition to automating the instrumentation, it determines the overload and level of intrusiveness of the instrumentation process. This module is designed to execute the fewest instructions possible in the most optimized manner. Currently, there is instrumentation support for C/C++ languages through shared libraries that work with either automatic or manual instrumentation. Manual instrumentation is preferred when it is necessary to examine the application program in specific code sections, regardless of the type of content. 

The manual instrumentation mode is implemented in three routines: one for initialization, another to mark the beginning of the region of interest ({\tt analyzer\_start}), and finally, a routine to mark the end of that region ({\tt analyzer\_stop}). The initialization routine is called when loading the library to create the necessary data structures and set up the data exchange communication. The routines {\tt analyzer\_start} and {\tt analyzer\_stop} identify threads in a region and store timestamps. These routines are implemented in such a way that only one thread at a time writes in a designated position of a two-dimensional array, thus ensuring thread safety and eliminating the necessity of locks.

The automatic instrumentation includes a routine allowing us to intercept the creation of threads via the {\tt LD\_PRELOAD} environment variable. This routine overwrites parts of an existing native library. In this manner, the functions responsible for thread spawning, such as {\tt pthread\_create} in the pthread library, {\tt GOMP\_parallel} (GCC implementation), {\tt \_\_kmpc\_fork\_call} (Clang implementation) in the OpenMP framework, and similar functions for other compilers, are intercepted to identify the parallel regions automatically. This approach is less intrusive than other tools, such as using a debugger interface with breakpoints or performing binary code instrumentation.

A key point of performance analysis tools, particularly important when analyzing real-time program execution, is that instrumentation must have a negligible impact on program behavior and execution time. Indeed, as mentioned in Section \ref{sec:pascal_architecture}, we support three types of sensors, each of which has a different degree of intrusion.
There is hardly any sensors intrusion at the start and then at the end of the program execution since it is simply the invocation of both data collection routines.

With sampling sensors, the degree of intrusion depends on the sampling rate and the total execution time of the application under study. Therefore, the intrusion needs to be assessed on a case-by-case basis with this type of sensor.

%It is fixed concerning the number of instructions and diverges with the number of threads. For example, suppose the user intends to analyze a parallelized region that is executed by four processing units. In that case, the execution time overhead is approximately equal to $4 \times T_{measurement}$, where $T_{measurement}$ corresponds to the time taken to perform a measurement, comprising the identification of events within the limits of the region involved in the measurement.

%The overhead for this example is the same when measuring two parallel regions with two cores and one parallel region with four cores. The \cref{lst:overheadsample1} and~\ref{lst:overheadsample2} present code snippets that correspond to the two scenarios mentioned and which, when analyzed using the tool, are therefore impacted by the same overhead.


The instrumentation overhead does not depend on the number of instructions or the runtime required to process the set of instructions to be analyzed. However, it varies according to the number of processing units used and measurements taken. To estimate the order of magnitude of the overhead ($T_{measurement}$), %MDPI: Is the italics necessary? Please carefully check and confirm. % yes
we measured recurring calls to the proposed sampling functions delimiting the regions in a simple benchmark code. This experiment was carried out in the same target architecture described in the experimental results of Section \ref{sec:casestudyarchitecture}, and the code structure is presented on Listing~~\ref{lst:overheadcode}.

\lstset{style=ccodestyle, frame=tb}

\begin{lstlisting}[label={lst:overheadcode}, language=C, caption={Code used to measure the overhead of instrumentation functions ({\tt analyzer\_start} and {\tt analyzer\_stop}) defined as $T_{measurement}$.}]
int main(int argc, char** argv) {
	get_time(begin);
	#pragma omp parallel for
	for (c=0; c<n_iterations; c++) {
		analyzer_start(1);
		usleep(1e4); // to simulate a simple operation
		analyzer_stop(1);
	} 
	get_time(end);
	time = begin-end;
}
\end{lstlisting}



The time required to execute $N$ %MDPI: Is the italic necessary? Please carefully check and confirm % yes
calls to the {\tt analyzer\_start} and {\tt analyzer\_stop} functions was obtained by the routine {\tt gettimeofday}, from the {\tt sys/time.h} library. The algorithm ran with 1 %MDPI: We used the scientific notations, please confirm. % ok
$\times~10^{4}$, 2 $\times~10^{4}$, 3 $\times~10^{4}$, 4 $\times~10^{4}$, 5 $\times~10^{4}$, and 1 $\times~10^{5}$ calls to the pair of functions, each test repetited ten times, and the mean, median, and variance values computed. \cref{fig:overhead} shows these results.

\begin{figure}[H]
\centering
\captionsetup{justification=centering}
\begin{subfigure}[b]{0.45\textwidth}	
	\includegraphics[width=\textwidth]{pascalanalyzer/figures/designfeatures/instrumentation_overhead_summary.pdf}
%	\caption{\centering}
	\label{fig:overhead_1}
\end{subfigure}
%
\begin{subfigure}[b]{0.46\textwidth}
	\includegraphics[width=\textwidth]{pascalanalyzer/figures/designfeatures/instrumentation_overhead_boxplot.pdf}
%	\caption{\centering}
	\label{fig:overhead_2}
\end{subfigure}
\caption{Measuring variance of the time to single instrumentation, 
	%MDPI: 1. For this figure, please use the scientific notations.
	%          2. Please add commas to numbers in the Ox axis of the figures, e.g.: 20,000 30,000 40,000 etc. - CHANGED
	i.e., a call to {\tt analyzer\_start} and {\tt analyzer\_stop} while varying the number of measurements. (\textbf{a}) Time consumed from sampling a region one time in seconds. %MDPI: We moved the explanations of subfigures here, please confirm.
	(\textbf{b}) Box plot showing the statistics of the sampling time of a single region.}
\label{fig:overhead}
\end{figure}

In Figure \ref{fig:overhead}a, we can see the mean, median, and variance of the time for a single call to our instrumentation function while varying the number of calls/measurements. The Figure \ref{fig:overhead}a complements these results showing the variance in each execution. %From these results, we can approximate the average value of $T_{measurement}$ to be $4e^{-6}$ seconds in this system. In other words, for each one million measurements, an overhead of approximately four seconds is added to the total program execution time.

\cref{tab:overhead} shows the results from the same experiment above comparing the time with and without the analyzer. We can observe that proportional impact (overhead) was constant while the number of interactions increased. \cref{tab:overhead} also presents the data referring to the simulation using the TAU profiling tool. For this simulation, we replaced the analyzer directives with the time measurement directives ({\tt TAU\_PROFILE\_START} and {\tt TAU\_PROFILE\_STOP}) of TAU, which allowed us to approximate the measurement and analysis conditions. 

\begin{table}[H]
\caption{Instrumentation overhead estimation varying the number of samples collected for TAU and~Analyzer.}
\label{tab:overhead}
%\centering
\newcolumntype{C}{>{\centering\arraybackslash}X}
\begin{tabularx}{\textwidth}{CCCCCC}\toprule
	\multirow{2}{*}{\vspace{-6pt}\textbf{Iterations}} & \multicolumn{3}{c}{\textbf{Time (s)}} & \multicolumn{2}{c}{\textbf{Overhead (\%)}} \\ \cmidrule{2-6}
	& \textbf{Real Time} & \textbf{TAU} & \textbf{Analyzer} & \textbf{TAU} & \textbf{Analyzer} \\ \midrule
	10,000	& 100.933	& 100.992	& 101.053	& 0.058	& 0.118 \\
	20,000	& 201.863	& 201.980	& 202.094	& 0.057	& 0.114 \\
	30,000	& 302.797	& 302.977	& 303.142	& 0.059	& 0.114 \\
	40,000	& 403.738	& 403.978	& 404.176	& 0.059	& 0.108 \\
	50,000	& 504.675	& 504.959	& 505.212	& 0.056	& 0.106 \\
	100,000	& 1009.359	& 1009.927	& 1010.432	& 0.056	& 0.106 \\ \midrule
\end{tabularx}
\end{table}

\textls[-20]{From the results, it is possible to see that the tool proposed in this work has a higher overhead than that presented by the TAU, but the analysis capabilities are distinct. TAU adds the individual runtime of each thread to define the execution time of a parallel region. This strategy does not consider the simultaneous action of threads in processing instructions and can count the same period of time more than once, damaging a precise measurement of the parallel region. The analyzer works around this problem because the intersection periods are counted just once. In this sense, although the TAU has lower overhead, a scalability analysis depends on a more accurate measurement like the one proposed in this work.}
\newpage
Varying the number of threads impacts the cost of instrumentation. As shown in \mbox{\cref{tab:overhead2}}, there is a trend for the percentage of overhead to increase concerning the application's runtime. However, it is also possible to see that this increase is negligible, considering the exponential increase in the number of threads. The increase in the processing load, on the other hand, has a beneficial impact on the relationship between execution time and the percentage of overhead. This behavior occurs because the runtime will increase with the greater need for processing, and the overhead will only change if the number of threads is also changed.

%The longer the execution time elapsed in each measurement, the smaller is the proportional impact of $T_{measurement}$ on the analysis as shown in Table .
%(9.181-9.142)/10000 = 0.0039s/Iteration
%(91.81-91.42)/100000 = 0.0000039s/Iteration
%100*(9.181-9.142)/9.181 = 0.4247 %over
%100*(91.81-91.42)/91.81 = 0.4247 %over
%The instrumentation overhead generated when measuring a region is located in the time of the outermost region. 
%This means that the execution time indicated by the tool for a region is only influenced by the instrumentation when there are more internal analysis regions. 
%In practice, the implementation strategy used in this tool offers the user the possibility of verifying more precisely the performance of certain parts of the code.


% Please add the following required packages to your document preamble:
% \usepackage{multirow}
%\begin{table*}
%\centering
%\caption{Overhead estimation varying the number of samples collected}
%\label{tab:overhead}
%\begin{tabular}{rrrrrrr}
%\midrule
%\multirow{2}{*}{Iterations} & \multicolumn{2}{c}{Time (s)} & \multicolumn{3}{c}{Time %estimation of single instrumentation (s)} & \multirow{2}{*}{Overhead (\%)} \\ %\cmidrule{2-6}
% & without analyzer & with analyzer & median %& average & variance & \\ \midrule
%10000 & 9.14 & 9.18 & 4.05e-06 & %4.03e-06 & 2.40e-14 & 0.44 \\
%20000 & 18.28 & 18.36 & 4.03e-06 & %4.04e-06 & 1.24e-14 & 0.44 \\
%30000 & 27.43 & 27.54 & 3.88e-06 %& 3.88e-06 & 1.35e-14 & 0.42 \\
%40000 & 36.57 & 36.72 & 3.89e-06 %& 3.90e-06 & 2.22e-14 & 0.43 \\
%50000 & 45.71 & 45.91 & 3.94e-06 %& 3.93e-06 & 2.52e-14 & 0.43 \\
%100000 & 91.42 & 91.81 & 3.93e-06 %& 3.90e-06 & 2.10e-14 & 0.43 \\ \midrule
%\end{tabular}
%\end{table*}

\begin{table}[H]
\caption{Instrumentation overhead while varying the number of threads with the number of samples fixed to one million.}
\label{tab:overhead2}
%\centering
\newcolumntype{C}{>{\centering\arraybackslash}X}
\begin{tabularx}{\textwidth}{CCCC}\toprule
	\multirow{2}{*}{\vspace{-6pt}\textbf{Threads}} & \multicolumn{2}{c}{\textbf{Time (s)}} & \multirow{2}{*}{\vspace{-6pt}\textbf{Overhead (\%)}} \\ \cmidrule{2-3}
	& \textbf{Direct} & \textbf{with Analyzer} \\ \midrule
	1 & 1009.360 & 1010.430 & 0.107 \\
	2	& 504.733	 & 505.374	 & 0.127 \\
	4	& 252.399	 & 252.742	 & 0.135 \\
	8	& 126.215	 & 126.390	 & 0.138 \\
	16	& 63.109	 & 63.199 & 0.143 \\ \bottomrule
\end{tabularx}
\end{table}

For pluggable sensors, overhead is generally not a concern as they run on a separate thread and have very low CPU usage, which presents only a few scenarios where they can cause interference.

\textls[-30]{One of these scenarios is where the application needs all the machine's resources at the same time that we have sensors that can respond faster than the processing speed at a given sample rate. In this case, the overhead would be directly associated with the sample rate and how the system handles concurrency. In general, this is rarer to happen in HPC (High-performance computing) since it would require an application with a perfect linear scaling.}

Another possible scenario is I/O overhead, where some network, disk, or memory resource becomes unavailable due to high sensor usage. However, this scenario is even rarer as sensors seldom produce data that quickly and, in all the performed tests, this was not close to being an issue.
	
\section{Features and Usage} \label{sec:features_and_usage}

The analyzer is a simple and easy-to-use tool that allows the user to understand the general behavior of a program before investing in efforts that require in-depth and sometimes longer-term analysis. However, to meet the objectives that best match the needs and resources of the user, the tool provides functionalities to parameterize the measurement process. Among the main ones are:
\begin{enumerate}
	\item Automated execution and deployment of parameterized code, actuators, and sensors
	%Automated code execution with different actuators and sensors saving time of a manual process.
	\item Automated low-overhead binary code instrumentation.
	\item No need of the source code or to recompile, unless the user desires it. 
	%Automated instrumentation in the binary with low overhead eliminating the need to access source code or recompile.
	\item Flexible user-defined observation regions: analysis of contexts within and beyond parallel regions.
	%The flexibility to identify specific observation regions: which allows the analysis of contexts beyond the blocks of instructions defined as parallel.
	\item Automatic instrumentation of parallel and remaining intermediate serial regions.
	%The ability to compute intermediate regions and automatically derive sensor values in certain parts of the program between user-defined regions. In scalability analysis, this becomes very useful since parallel regions can be instrumented automatically and the remaining regions can already be classified as serial, allowing the identification of all parts of the program.
	\item Aggregation, by region, of the collected metrics, which significantly reduces the amount of data stored on disk and RAM usage; useful in cases where it would be impossible to store information about all sensor events, such as the instrumentation of a for loop with billions of iterations. In this case it is possible to choose how to group the data either by mean, median, minimum or maximum values.
	\item Hierarchical regions: The proposed analyzer also facilitates the identification of aligned regions, enabling the identification of the calling hierarchy of inner regions and block analysis.
\end{enumerate}
\newpage
The tool can be used in the command line or via its API. The API provides calls to integrate sensors and actuators as shown in the Listing \ref{lst:pascal_api}.
%MDPI: Please provide the explanation % added in abbreviations


\lstset{style=pythonStyle,frame=tb}
\begin{lstlisting}[label={lst:pascal_api}, language=python, caption={Python script showing some API features provided by the tool, on how a custom run can be configured.}, captionpos=b]
from analyzer.run import Run
from analyzer.actuators import CPUFrequencies
from analyzer.actuators import CPUCores
from analyzer.sensors import RAPL
from analyzer.sensors import fingerprint
from itertools import product

lsensors = [
RAPL(), 
fingerprint(counters=["INSTRUCTIONS"])
]
cores = [
CPUCores(c) for c in range(1,32)
]
freqs = [
CPUFrequencies(2800000), 
CPUFrequencies(2600000)
]

# all combinations
configs = list(product(cores,freqs))
app = Run(application="a.out",
repetitions=10, 
instr_auto=True, 
sensors=lsensors)
app.run(configs)
app.savedata("out.json")
\end{lstlisting}



\section{Exported Data Structure} \label{sec:exported_data_structure}

The data collected by the tool is exported to a .json file and stored on disk. The data structure is divided into two large groups. One group contains information about the configuration parameters driving deployment and execution. The other includes the performance measurements. The data is grouped in keys representing the processed input, the number of processing elements used in the execution, and the simulation ID. Since the tool can perform the same configuration many times, a simulation ID distinguishes runs that use the same input and number of processing elements.


\cref{lst:datafile} presents an example file exported by the analyzer. There we can see the main parts of the exported structure, with the header where essential information about the system is written followed by the description of the collected data, where the list of actuators is presented as well as the specific type of information that was collected from the sensors. Finally the samples are presented separated by actuator configurations, and sensor type.

The keys and values in the .json file do not represent information that is easy for the user to identify and understand. However, it can be easily interpreted by a script or visualization software such as the PaScal Viewer~\cite{Silva2018}. The proposed tool provides a visualization module, responsible for interpreting and presenting data in an organized and user-friendly manner.

\lstset{style=jsonstyle, frame=tb}
\begin{lstlisting}[label={lst:datafile}, language=json, caption={Sample of exported data file showing the internal structure and organization of the data.}]
{
	"config": {
		****************** Header ******************
		"pkg": "./application", 
		"execdate": "00-00-0000_00:00:00",
		"kernel": "Linux-4.4",
		"command": "analyzer ...",
		"hostname": "host",
		...
		******** Data collected description ********
		"data_descriptor": {
			******** Mandatory collected values ********
			"values": [ 
			"start_time", "stop_time", ... 
			],
			************** Actuators list **************
			"keys": [ "A1", "A2" ...],
			********* Sensor data description **********
			"extras": {
				"regions": { ... }
				"sensors": { ... }
			}
		},
		"arguments": ["input_1", ...]
	},
	************** Data collected **************
	"data": {
		***** Actuator values separated by ";" *****
		"X;Y;Z": {
			********* Data collected by sensor *********
			"regions": { ... },
			"sensors": { ... },
			"start_time": ti,
			"stop_time": tf
		}
	}
}
\end{lstlisting}
	%\author{ \parbox{3 in}{\centering Huibert Kwakernaak*
%         \thanks{*Use the $\backslash$thanks command to put information here}\\
%         Faculty of Electrical Engineering, Mathematics and Computer Science\\
%         University of Twente\\
%         7500 AE Enschede, The Netherlands\\
%         {\tt\small h.kwakernaak@autsubmit.com}}
%         \hspace*{ 0.5 in}
%         \parbox{3 in}{ \centering Pradeep Misra**
%         \thanks{**The footnote marks may be inserted manually}\\
%        Department of Electrical Engineering \\
%         Wright State University\\
%         Dayton, OH 45435, USA\\
%         {\tt\small pmisra@cs.wright.edu}}
%}

\section{An Accurate Tool for Modeling, Fingerprinting, Comparison, and Clustering of Parallel Applications Based on Performance Counters}
The analysis of application performance is essential to better exploit its potential on High-Performance Computing (HPC) architectures. Access to performance counters, available in modern processors, allows collecting key information about program behavior to provide the most appropriate HPC execution strategy.
In this context, we have developed an accurate tool based on performance counters, which facilitates modeling, fingerprinting, behavior comparison and clustering of applications.
It provides a high-level Python API for accessing and configuring performance counters; while execution and counters data gathering is  performed by a C++ module to reduce overhead. 
Indeed, the accuracy of this multiplatform tool was also compared to existing alternatives.  
Key features, such as performance counters collection, post-processing, and comparison, enable fingerprinting of applications, an important step in understanding program behavior for later classification and optimization according to the parameters characterizing the target HPC platform.
For demonstration purposes, the tool was used in the clustering of Polybench applications, a frequently used benchmark set for kernels monitoring. 
%%, a frequently used benchmark for compiler optimization and testing. 
This clustering facilitated the identification of applications with similar and comparable behaviors in terms of input size, data access and transfer, resource utilization and computation, which facilitates the creation of test sets for a given environment based on specific measurement parameters.


%%%%%%%%%%%%%%%%%%%%%%%%%%%%%%%%%%%%%%%%%%%%%%%%%%%%%%%%%%%%%%%%%%%%%%%%%%%%%%%%
\subsection{INTRODUCTION}

Hardware Performance Counters are special registers available on most modern processors capable of counting micro-architectural events such as instructions executed, cache-hit, branches miss-predicted, energy estimation and much more.  In new architectures, there are hundreds of hardware events that can be monitored, and more are added to each new generation. 
Performance counters were initially introduced for debugging, but since then they have provided a lot of useful information about running applications without slowing down the execution.
They have been used in several other areas, such as software profiling \cite{Melo2010, Kufrin2005, Knupfer2011}, CPU power modeling \cite{Zamani2012ASystems}, dynamic frequency and voltage scaling, vulnerability research and malware defense \cite{Demme2013OnCounters}.

%\cite{Geimer2010, Geimer2010TheArchitecture, Shende2006TheSystem}

Exploiting Performance Monitoring Units (PMU) requires an intimate knowledge of the micro-architecture and kernel API, as well as an awareness of an ever increasing complexity. 
Otherwise, the measurement performance and accuracy will be seriously affected. Although many tools have been developed using performance counters, programmable interfaces capable of providing good accuracy are still lacking, especially for high-level programming languages. Indeed, apart from PAPI \cite{Weaver2013, Mucci1999} and Perfmon \cite{EranianPerfmon2Interface, Eranian2005TheSpecification} there are only a few APIs allowing access to these counters, and many others are poorly documented, unstable, or designer for a specific purpose.

Performance metrics may have different definitions and programming interfaces on different platforms. 
Therefore, besides gathering information, post-processing modules are also needed. 
Such modules will overcome the lack of precision of counters on some architectures, as indicated in \cite{Weaver2008,Weaver2013a,Das2019SoK:Security}. 
Events that must be precise and deterministic (such as retired instructions) show a variation on run-to-run and over-count on x86\_64 machines, even in strictly controlled environments. These effects are almost always non-intuitive to casual users and pose problems when strict determinism is desirable. 

To meet the above mentioned requirements, our strategy combines low-level, efficient and accurate access to PMUs, facilitated by high-level programmable interfaces. This allows the user to perform all configurations and post-processing in Python,  while the underlined architecture details and information gathering are supported by a C++ module, thus preserving accuracy with limited overhead. 
In order to understand the behavior of programs for future comparisons and classification, the proposed tool makes another contribution: the ability to fingerprint and cluster applications.

The definition of reference parameters, such as input size of programs, and performance measures uniformization, allow us to cluster benchmarks such as PolyBench \cite{PolyBench/C3.2}, a collection of numerical computations with static control flow extracted from various application domains, with interesting results. 
In addition to contributing to the standardization of kernel execution and monitoring, this clustering has identified applications with similar and comparable behavior in terms of input size, data transfer and access, resources used and computation; which facilitates the creation of test sets for a given environment, according to specific measurement parameters. 

The rest of this article is organized as follows. subsection 2 provides the related work regarding tools available and requirements. subsection 3 presents the performance counters and motivation. subsection 4 presents our clustering and application analysis tool, architecture and components. subsection 5 shows evaluation results: the comparison of existing API and the clustering of the PolyBench benchmarks. Finally, subsection 8 concludes this article.

\subsection{RELATED WORK}

There are only a few APIs allowing access to performance counters.
PAPI \cite{Weaver2013, Mucci1999}, one of the most used libraries for accessing hardware performance counters, was originally developed to provide portable access to the counters found on a diverse collection of modern microprocessors. Rather than learning and writing a new performance infrastructure every time it is ported to a new machine. Measurement code can be written in the PAPI API, which hides the underlying interface.  
PAPI was developed on C and a few non-official libraries were ported to Python. The main problem we found using PAPI was the Python version of has a considerable overhead, it also does not have an easy way to create raw events or low-level control without having to use a special driver. And as our tests will show later, counters sampling over time does not produce good results either.

There are also available a set of interfaces using their own drivers, mainly because the counters are only accessible in kernel mode (ring 0) to control the events for which the counter must be started or stopped. Some events are fixed and others require the development of a dedicated kernel driver. 

Perfctr \cite{MikaelPettersson.TheInterface.} supports per-kernel-thread and system-wide monitoring for most major processor architectures. It is distributed as a stand-alone kernel patch. The interface is mostly used by tools built on top of the PAPI performance toolkit. 

The Intel VTUNE \cite{IntelCorp.TheAnalyzer.} performance analyzer comes with its own kernel interface, implemented by an open-source driver. The interface supports system-wide monitoring only and is very specific to the needs of the tool.

The problem with the approach of a tool and its own kernel interface is dangerous because, as mentioned on \cite{EranianPerfmon2Interface}, there is clearly code duplication, but more importantly, there is no coordination between the various interfaces that may coexist sharing access to the same PMU resource. To solve this problem, Perfmon2 \cite{Eranian2005TheSpecification} offers a standard interface that all tools can use. Unfortunately, it has not been widely adopted, just supported by a few architectures like the IA64. Instead, Linux comes up with a performance counters subsystem which provides a complete set of configurations.


\subsection{READING PERFORMANCE COUNTERS}

Although hundreds of events are available for monitoring, only a limited number of counters can be used simultaneously.
Therefore, the events to be monitored should be carefully selected and configured using the available counters.

The number of counters available varies between processor architectures, e.g., modern Intel CPUs \cite{Intel2013IntelGuide} support three fixed and four programmable counters per core. Fixed counters monitor events such as Instruction Retired (how many instructions were completely executed), logical cycles, and reference cycles, while programmable counters can be configured to monitor architectural and non-architectural events. 
However, if additional counters are needed, the available ones must be multiplexed.
The configuration of the counters is done by writing in Model-Specific Registers (MSR), only accessible in ring 0 (kernel mode) as indicated previously. 


% \begin{itemize}
%     \item rdmsr - Reads the contents of a 64-bit model specific register (MSR) specified in the ECX register into registers EDX:EAX. This instruction must be executed at privilege level 0 or in real-address mode
%     \item rdpmc - Is slightly faster that the equivalent rdmsr instruction. rdpmc can also be configured to allow access to the counters from userspace, without being privileged.
% \end{itemize}

Operating systems provide an abstraction of these hardware capabilities to access counters and MSRs. 
On the Linux system, where our work was developed, there is a performance monitoring subsystem which provides per-task and per CPU counters, counter groups, and related event features. 
All events are seen as 64-bit virtual counters, regardless of the width of the underlying hardware counters. 
They are accessible via special file descriptors, one file descriptor per virtual counter, opened via the perf\_event\_open() system call.  %These system call do not use rdpmc, but rdpmc is not necessarily faster than other methods of reading event values.
Counter events can be processed by interrupt, polling or on time. The interrupt operates by hooking a user-defined function to a specified event, such as a counter overflow, and whenever this event happens, a signal will be generated passing the control to the designated handler function. With polling, whenever an event occurs on the system, the counter value is queued by the operating system and the user can read from this queue using a system call. The last option is to sample over time reading counters every n second.

Ideal hardware performance counters provide exact deterministic results. Real-world PMU implementations do not always live up to this ideal \cite{Weaver2008, Weaver2013a, Das2019SoK:Security, McGuire2009}. Events that should be exact and deterministic (such as the number of executed instructions) show run-to-run variations and over counts on x86 64 machines, even when running in strictly controlled environments. 
These effects are non-intuitive to casual users and cause difficulties when strict determinism is desirable, such as when implementing deterministic replay or deterministic threading libraries. 
Because of that, we have implemented a methodology to reduce the noise and over counts on performance counters.


\subsection{IMPLEMENTATION}

%This module provide a high-level abstraction API to Linux perf events without overhead while executing

This tool is composed of 5 modules the are Profiler, Events, Workload, Analyzer and libpfm4. 
The Workload and libpfm4 module are developed on C/C++ and interfaced with python using Python C API with SWIG (a software development tool to connects programs written in C/C++ with a variety of high-level programming languages). 
The libpfm4 module, developed by \cite{Eranian2008}, is used as an auxiliary library.

%The main features that this library provide is a precise synchronous start the application and counter the event, a low overhead on sampling the events on time and an easy way to find and configure event in groups or standalone. 

\subsubsection{ARCHITECTURE}

In figure \ref{fig:achitecture} we can see how these modules interact with each other.
\begin{figure}[H]
    \centering
    \includegraphics[width=\textwidth]{fingerprint/figures/architecture.png}
    \caption{Modules interconnection}
    \label{fig:achitecture}
\end{figure}

The Profiler is the user interface for configuring and creating events.
It is also responsible for calling the Workload module to run the application and retrieve the data after the execution is complete.

The Workload module, developed on C++, is the core of the library.
It is responsible for creating the application and the sampling process.
It provides a precisely synchronized start, it halts the program before the execution of the first instruction, using the debug interface on Linux (place), and launches the application once the counters have been properly reset and ready to run.
This module, built-in as a Python module using the Python C API, defines a set of functions, macros, and variables to access most aspects of the Python run-time system.
% The workload run simplified code
% \begin{lstlisting}[language=c++]
% reset_counters();
% start_counters();
% start_application();
% while(application.is_running()){
%      sleep(secounds);
%      sample_counters();
% }
% return sampled data;
% \end{lstlisting}
The Event module provides a description of events, events parameters, configurations and PMUs available.

System calls for reading and event creation are done directly by the Workload module or through the libpfm4 Python link. 
The libpfm4 is also used to find events encodings and convert an event name (expressed as a string) to the corresponding event encoding, either as a raw event number (as documented by the hardware vendor) or the OS-specific encoding.
In the latter case, the library is able to prepare the OS-specific data structures needed by the kernel to setup the event.

The Analyzer module is responsible for the post-processing of the data. 
It takes the data from several runs of the application and provides a set of functions to remove outlines, interpolate, filter and compare.
% \subsubsection{FUNCTIONALITY}
% This subsection is going to describe the main functionality of the API.
% Profiler module:
% \begin{lstlisting}[language=Python]
% Profiler(events_groups, program_args=None)
% \end{lstlisting}
% This constructor creates a profiler object from a list of events groups with their names, optionally can pass the application that will be executed.
% \\
% \begin{lstlisting}[language=Python]
% set_program(program_args)
% \end{lstlisting}
% Set the program and his arguments to be executed from a list.
% \\
% \begin{lstlisting}[language=Python]
% start_counters(pid=0)
% \end{lstlisting}
% Monitor events by pid. If pid its set to 0, monitor the entire system.
% \\
% \begin{lstlisting}[language=Python]
% enable_events()
% disable_events()
% reset_events()
% read_events()
% \end{lstlisting}
% Control events
% \\
% \begin{lstlisting}[language=Python]
% run(sample_period, reset_on_sample=False)
% \end{lstlisting}
% Run the application and sample the events on the time, it receives the sample time and if a flat that controls the reset on the sample. It returns the data with the events sampled. 
% \\
% \begin{lstlisting}[language=Python]
% run_background()
% \end{lstlisting}
% Run the application in background
% \begin{lstlisting}[language=Python]
% run_program(...)
% save_data(data, name)
% \end{lstlisting}
% Simple functions to run a program multiple times and store the results in a file.
% \\
% \\
% Events module:
% \begin{lstlisting}[language=Python]
% get_supported_pmus()
% get_supported_events(name)
% get_event_description(name)
% get_event_attrs(name)
% \end{lstlisting}
% These functions return a list with a description for the PMU, event, and attributes.
% \\
% \\
% Analyser module:
% \begin{lstlisting}[language=Python]
% load_data(name)
% \end{lstlisting}
% Load the data capture into an Analyser object.
% \\
% \begin{lstlisting}[language=Python]
% process(verbose=False)
% \end{lstlisting}
% Process the data, using the method described on the post-processing subsection \ref{sec:posprocessing}.
% \\
% \begin{lstlisting}[language=Python]
% compare(a1, a2, feature, npoints)
% \end{lstlisting}
% Compare two objects on a specific feature.

\subsubsection{POST-PROCESSING}
\label{sec:posprocessing}
% With this API we create clusterize the applications of Polybench applications.

With the data collected from multiple runs of the application, the first goal is to obtain a single curve that minimizes the noise caused by the operating system and inaccuracies of the counters. 
In figure \ref{fig:multiple_exec} we can see the raw result of multiple runs of the Polybench 2mm program measuring the number of executed instructions. 
For better visualization, the horizontal axis has been normalized over the interval from 0 to 100. 
This implies that we no longer analyze the program on the time scale, but in the interval in which it took place.

\begin{figure}[H]
    \centering
    \includegraphics[width=\textwidth]{fingerprint/figures/PERF_COUNT_HW_INSTRUCTIONS.png}
    \caption{Multiple executions of the Polybench 2mm program with different input sizes}
    \label{fig:multiple_exec}
\end{figure}

We apply a median filter to each set of runs, sorting the values and removing edge values.
Then we calculate the average curve, whose final result can be observed on the figure \ref{fig:single_curve}.

\begin{figure}[H]
    \centering
    \includegraphics[width=\textwidth]{fingerprint/figures/PERF_COUNT_HW_INSTRUCTIONS_1.png}
    \caption{Single instance representation}
    \label{fig:single_curve}
\end{figure}

 After that, we interpolate the curve using the B-spline \cite{Hang2017CubicApplications} algorithm to get the same number of points to all curves. 
 The result of this step can be observed in figure \ref{fig:interporlation}. 

\begin{figure}[H]
    \centering
    \includegraphics[width=\textwidth]{fingerprint/figures/PERF_COUNT_HW_INSTRUCTIONS_2.png}
    \caption{Interpolation}
    \label{fig:interporlation}
\end{figure}

Finally, we use the Savgol filter \cite{Luo2005PropertiesDifferentiators} in order to smooth the data to increase the signal-to-noise ratio without too much distortion. The result is shown in figure \ref{fig:filtering}.

\begin{figure}[H]
    \centering
    \includegraphics[width=\textwidth]{fingerprint/figures/PERF_COUNT_HW_INSTRUCTIONS_3.png}
    \caption{Filtering}
    \label{fig:filtering}
\end{figure}

\subsection{RESULTS}

In this subsection, we first show the comparison between our tool and others already established, as well as the clustering process.

\subsubsection{ACCURACY COMPARISON}

% \begin{table*}[h]
% \centering
% \caption{Average}
% \begin{tabular}{|c|c|c|c|c|c|c|}
% \hline
% Counter                                 & Pined values & Linux API   & PAPI      & PAPI Python & Perf tool   & MyPerf      \\ \hline
% INSTRUCTIONS\_RETIRED                  & 226990030    & 227000691   & 227000620 & 225901249   & 227000572   & 227000650   \\ \hline
% BRANCH\_INSTRUCTIONS\_RETIRED          & 9240000      & 9250617     & 9250566   & 9239617     & 9250552     & 9250501     \\ \hline
% BR\_INST\_RETIRED:CONDITIONAL          & 8220000      & 8220000     & 8220000   & 8209717     & 8220000     & 8220000     \\ \hline
% MEM\_UOP\_RETIRED:ANY\_LOADS           &              & 2484182672  &           &             & 2484383940  & 2484155029  \\ \hline
% MEM\_UOP\_RETIRED:ANY\_STORES          &              & 189962002   &           &             & 189961539   & 189961321   \\ \hline
% UOPS\_RETIRED:ANY                      &              & 12291082129 &           &             & 12290811038 & 12290901997 \\ \hline
% PARTIAL\_RAT\_STALLS:MUL\_SINGLE\_UOP  &              & 600878      &           &             & 600151      & 600330      \\ \hline
% ARITH:FPU\_DIV                         &              & 5801446     &           &             & 5801000     & 5800977     \\ \hline
% FP\_COMP\_OPS\_EXE:X87                 &              & 48785528    &           &             & 48784834    & 48786021    \\ \hline
% INST\_RETIRED:X87                      &              & 17200008    &           &             & 17200008    & 17200007    \\ \hline
% FP\_COMP\_OPS\_EXE:SSE\_SCALAR\_DOUBLE &              & 5401694     &           &             & 5401842     & 5401679     \\ \hline
% \end{tabular}
% \label{tab:mean}
% \end{table*}

% \begin{table*}[h]
% \centering
% \caption{Standard deviation}
% \begin{tabular}{|c|c|c|c|c|c|}
% \hline
% Counter                                    & Linux API & PAPI & PAPI Python & Perf tool & MyPerf \\ \hline
% INSTRUCTIONS\_RETIRED                  & 396       & 133  & 337763      & 110       & 175    \\ \hline
% BRANCH\_INSTRUCTIONS\_RETIRED          & 297       & 208  & 8485        & 379       & 91     \\ \hline
% BR\_INST\_RETIRED:CONDITIONAL          & 0         & 0    & 3383        & 0         & 0      \\ \hline
% MEM\_UOP\_RETIRED:ANY\_LOADS           & 37399     &      &             & 39217     & 38953  \\ \hline
% MEM\_UOP\_RETIRED:ANY\_STORES          & 1513      &      &             & 1035      & 687    \\ \hline
% UOPS\_RETIRED:ANY                      & 345246    &      &             & 335832    & 333298 \\ \hline
% PARTIAL\_RAT\_STALLS:MUL\_SINGLE\_UOP  & 1222      &      &             & 252       & 521    \\ \hline
% ARITH:FPU\_DIV                         & 1760      &      &             & 1621      & 1544   \\ \hline
% FP\_COMP\_OPS\_EXE:X87                 & 1283      &      &             & 1920      & 3311   \\ \hline
% INST\_RETIRED:X87                      & 4         &      &             & 4         & 3      \\ \hline
% FP\_COMP\_OPS\_EXE:SSE\_SCALAR\_DOUBLE & 1547      &      &             & 3259      & 2097   \\ \hline
% \end{tabular}
% \label{tab:std}
% \end{table*}

\begin{table*}[h]
\centering
\caption{Comparison}
\resizebox{\textwidth}{!}{%
\begin{tabular}{|c|c|c|c|c|c|c|c|c|c|}
\hline
\multicolumn{6}{|c|}{Average*$10^{-6}$} & \multicolumn{4}{c|}{Standard deviation}\\ \hline
Counters & \begin{tabular}{c}Pined\\values\end{tabular} & \begin{tabular}{c}Linux\\API\end{tabular} & PAPI & \begin{tabular}{c}PAPI\\Python\end{tabular} & MyPerf  & \begin{tabular}{c}Linux\\API\end{tabular} & PAPI & \begin{tabular}{c}PAPI\\Python\end{tabular} & MyPerf \\ \hline
INSTRUCTIONS\_RETIRED                  & 226.99       & 227       & 227  & 225.9       & 227     & 396       & 133  & 337763      & 175    \\ \hline
BRANCH\_INSTRUCTIONS\_RETIRED          & 9.24         & 9.25      & 9.25 & 9.24        & 9.25    & 297       & 208  & 8485        & 91     \\ \hline
BR\_INST\_RETIRED:CONDITIONAL          & 8.22         & 8.22      & 8.22 & 8.21        & 8.22    & 0         & 0    & 3383        & 0      \\ \hline
MEM\_UOP\_RETIRED:ANY\_LOADS           &              & 2484.18   &      &             & 2484.16 & 37399     &      &             & 38953  \\ \hline
MEM\_UOP\_RETIRED:ANY\_STORES          &              & 189.96    &      &             & 189.96  & 1513      &      &             & 687    \\ \hline
UOPS\_RETIRED:ANY                      &              & 12291.08  &      &             & 12290.9 & 345246    &      &             & 333298 \\ \hline
PARTIAL\_RAT\_STALLS:MUL\_SINGLE\_UOP  &              & 0.6       &      &             & 0.6     & 1222      &      &             & 521    \\ \hline
ARITH:FPU\_DIV                         &              & 5.8       &      &             & 5.8     & 1760      &      &             & 1544   \\ \hline
FP\_COMP\_OPS\_EXE:X87                 &              & 48.79     &      &             & 48.79   & 1283      &      &             & 3311   \\ \hline
INST\_RETIRED:X87                      &              & 17.2      &      &             & 17.2    & 4         &      &             & 3      \\ \hline
FP\_COMP\_OPS\_EXE:SSE\_SCALAR\_DOUBLE &              & 5.4       &      &             & 5.4     & 1547      &      &             & 2097   \\ \hline
\end{tabular}
}
\label{tab:counters}
\end{table*}

To validate the tool, we compared the results of the counters obtained with different APIs. 
We used the hand-crafted assembly benchmark from \cite{Weaver2013a}, designed to test determinism and accuracy of PMUs.
We compared the values obtained from the Linux API, PAPI on C and PAPI on Python. 
The events used for this comparison were instructions retired, branch instructions, memory read, memory load, and arithmetic operations.
We ran the benchmark 30 times and calculated the mean and standard deviation as shown in table \ref{tab:counters}. Some events could not be measured using PAPI because the tool does not accept raw events and there are no equivalent events.

Since the benchmark was hand-crafted with assembly, we know exactly the value for some counter events. 
For this reason, the number of instructions, branch instructions, and conditional branch are pin. However, some other events are architecture-specific and there is no pined value. 
In the latter case, we can still compare to the Linux API, which should be closest to the reality.

The differences using the Linux low-level API, PAPI, and our tool are negligible (the average percentage distance is less than 0.01\% in all the cases). 
As expected, PAPI on Python had the largest difference (with an average distance of 0.25\%) mainly due to an unsynchronized start that resulted in the loss of some instructions at the beginning of the execution. 
This can be an important problem if the application contains a small number of instructions.

The standard deviation of the 30 executions shows that our tool has the smallest variation on a run-to-run on most events. On the contrary, PAPI on Python shows a big variation compared to the others.

\subsubsection{CLUSTERING}

To cluster applications first we need to have a way to compare two programs, for that we define a new variable that tries to compute a fingerprint to the program, this variable has to look similar when we execute the same program with different conditions and inputs. Thinking in the simplest model of a program as a Turing machine as everything can be done with a tap of memory and set of rules we empirically define the variable input size as described on the equation \ref{eq:input_size}. On real computers this is not too far from reality, most of the computations and input and output operations somehow pass through the memory, so analyzing the relationship of the total number of instructions and memory instructions can give us a good fingerprint.

\begin{equation}
    I_{sz} = \frac{I}{I_{m}} \\
    \label{eq:input_size}
\end{equation}

Where $I_{sz}$ we called input size, $I$ the number of instructions executed, $I_m$ number of memory instructions execute. We observe that this variable demonstrate to have the proprieties that we are looking for, produce similar results to the same program with different input size, and environment. This can be used to identify programs but also to see the similarities between different applications.

To compute the distance between two programs we use the Canberra metric \cite{Jurman2009CanberraLists}, described on the equation \ref{eq:canberra}.

\begin{equation}
    d(p,q)= \sum_{i=1}^{n}{\frac{|p_i-q_i|}{|p_i|+|q_i|}}
    \label{eq:canberra}
\end{equation}

Where $p$ and $q$ are n-dimensional vectors.

We compute the input size for all 30 applications of the Polybench with 3 different inputs. Running each program 15 times, with a sampling rate of 0.01 seconds collecting the total number of instruction, number of loads and write instructions, number of floating point operations, besides some software counters. After applying the post-processing to the data collected we compute the input size and perform a Hierarchical clustering using the linkage method of Ward \cite{Murtagh2011WardsAlgorithm}, that minimizes the total within-cluster variance. The results of the clustering can be seen on the figure \ref{fig:sping_force_is} and on the dendrogram on figure \ref{fig:dendograma_input_size}.

%The exact HPCs was:
%PERF_COUNT_HW_INSTRUCTIONS,
%MEM_UOPS_RETIRED:ALL_LOADS,
%MEM_UOPS_RETIRED:ALL_STORES,
%FP_ARITH_INST_RETIRED:SCALAR

\begin{figure*}[h]
    \centering
    \includegraphics[width=\textwidth]{fingerprint/figures/dendograma_input_size.png}
    \caption{Dendrogram}
    \label{fig:dendograma_input_size}
\end{figure*}

From this dendrogram, we can have an idea of how close two applications are. We choose the number of clusters that maximize the number of hits of the same program with different input sizes on the same cluster, in this case, was 5 clusters.

It was observed that as input size grows the program behavior tends to a specific curve, but for small input sizes some have variation, so in some cases, the same program has been classified in more than one cluster. This can also happen if specific parts of the program are triggered with specific inputs, in which case it will also belong to more than one cluster.

In order to have an overall classification of each program, we can pick up the frequency in which each appeared in the clusters and classified it in the cluster in which it appeared more often. In this case, the clusters are:

\begin{itemize}
    \item Cluster 1: 2mm, 3mm, cholesky, correlation, covariance, floyd-warshall, gemm, gramschmidt, lu, ludcmp, nussinov, symm
    
    \item Cluster 2: deriche, doitgen, syrk
    
    \item Cluster 3: adi, fdtd-2d, jacobi-2d, syr2k
    
    \item Cluster 4: atax, bicg, durbin, gemver, gesummv, mvt, trisolv, trmm
    
    \item Cluster 5: heat-3d, seidel-2d
\end{itemize}

On the figure, \ref{fig:sping_force_is} we display the clusters using the spring force algorithm where each program with a specific input is a node and the edge weight is the Canberra distance. To a better visualization, the name of the inputs was replaced by numbers, the EXTRALARGE is 3, LARGE 2 and MEDIUM 1. 

\begin{figure}[H]
    \centering
    \includegraphics[width=\textwidth]{fingerprint/figures/graph_input_size.png}
    \caption{Spring force of Canberra distance}
    \label{fig:sping_force_is}
\end{figure}

The figure above shows a different way to observe in a reduced space the distance between clusters and applications and how they are organized. From this graph, we see that clustering it's well partitioned and we can clearly separate each cluster. It is also interesting to note that the applications of the cluster formed by the circle symbol are more separated, which may indicate that they were classified in this way because they did not fit into any other cluster.

% \begin{figure*}[t]
%     \begin{subfigure}{\textwidth}
%         \includegraphics[width=\textwidth]{fingerprint/figures/dendograma_input_size.png}
%     \end{subfigure}
%     \\
%     \begin{subfigure}{\textwidth}
%         \includegraphics[width=\textwidth]{fingerprint/figures/dendograma_floating.png}
%     \end{subfigure}
%     %and so on
% \end{figure*}

To have an idea of how the behavior of the variable input size for each clusters is, it was plotted for the applications of clusters 1 and 2 shown on the figures \ref{fig:c1_input_size_0}, \ref{fig:c1_input_size_1}.

\begin{figure}[H]
    \centering
    \includegraphics[width=\textwidth]{fingerprint/figures/cluster_input_0.png}
    \caption{Input size - Cluster 1}
    \label{fig:c1_input_size_0}
\end{figure}

From these figures, we can have an idea of what behavior was classified as the same class. On cluster 1 on the figure \ref{fig:c1_input_size_0} all applications showed the same behavior with approximately the same amplitude.

\begin{figure}[H]
    \centering
    \includegraphics[width=\textwidth]{fingerprint/figures/cluster_input_1.png}
    \caption{Input size - Cluster 2}
    \label{fig:c1_input_size_1}
\end{figure}

On figure \ref{fig:c1_input_size_1} we observe that curves that have similar shape regardless of scale on vertical and horizontal axis have also been classified as the same clusters. This is the wanted comportment for the classification because we are only interested in the overall shape of the curve.


% \begin{figure}[H]
%     \centering
%     \includegraphics[width=\textwidth]{fingerprint/figures/cluster_input_4.png}
%     \caption{Input size - Cluster 3}
%     \label{fig:c1_input_size_2}
% \end{figure}

% To conclude this study we also clustering applications by its floating point behavior. For that we only use the counter correspondent to floating point operations. The figure \ref{fig:dendogram_fp} shows the dendrogram and the \ref{fig:sping_force_fp} the spring force graph plot.

% \begin{figure*}[h]
%     \includegraphics[width=\textwidth]{fingerprint/figuresdendograma_floating.png}
%     \caption{Dendrogram}
%     \label{fig:dendogram_fp}
% \end{figure*}

% \begin{figure}[H]
%     \centering
%     \includegraphics[width=\textwidth]{fingerprint/figures/graph_floating2.png}
%     \caption{Spring force of Canberra distance}
%     \label{fig:sping_force_fp}
% \end{figure}

% The figures \ref{fig:cluster_fp1},\ref{fig:cluster_fp2},\ref{fig:cluster_fp3} show the behavior on the same cluster for the float point operations. For this classification the number of clusters was 8.

% \begin{figure}[H]
%     \centering
%     \includegraphics[width=\textwidth]{fingerprint/figures/cluster_fp_0.png}
%     \caption{Floating point - Cluster 1}
%     \label{fig:cluster_fp1}
% \end{figure}

% \begin{figure}[H]
%     \centering
%     \includegraphics[width=\textwidth]{fingerprint/figures/cluster_fp_4.png}
%     \caption{Floating point - Cluster 2}
%     \label{fig:cluster_fp2}
% \end{figure}

% \begin{figure}[H]
%     \centering
%     \includegraphics[width=\textwidth]{fingerprint/figures/cluster_fp_5.png}
%     \caption{Floating point - Cluster 3}
%     \label{fig:cluster_fp3}
% \end{figure}

% Here we can also observed again that applications with the same shape were classified as the same clusters.

\subsection{CONCLUSIONS}

%\cite{Processors2012, Gregg2017}
From the results, we observe that the API present overhead similar or lower to other low-level APIs, with the advantage of being in high abstraction and simplified configuration with a few lines of code, is possible to configure and gather counter data.

The tool developed provide also provided a way to fingerprint programs and compute similarities between different programs or the same program with different inputs. This can be useful to reduce applications spaces for benchmarks as was done in Polybench clustering but also to analyze the behavior of a parameter providing insights to the programmer to find a possible bottleneck.

We also provide a precise definition to input size that made possible fingerprint the programs, but it also can help programmers of benchmark applications to better create inputs with more precise growth of a particular parameter.

%\subsection{FUTURE WORK}
%
%We pretend to use this tool to create a data set of applications behavior and automatically classify any program as belonging to a cluster or a set of clusters. The idea is to have a set of clusters that can describe most applications in this way we can know specific behaviors of the applications. This can be applied to various areas such as voltage scaling and frequency of the processor, once knowing the behavior of a particular program we can have an idea of what is the best strategy to control the frequency in order to save energy or increase performance.


%%%%%%%%%%%%%%%%%%%%%%%%%%%%%%%%%%%%%%%%%%%%%%%%%%%%%%%%%%%%%%%%%%%%%%%%%%%%%%%%

\subsection{APPENDIX}

% \begin{figure*}[h]
%     \centering
%     \includegraphics[width=\textwidth]{dendograma_input_size.png}
%     \caption{Dendogram input size}
%     \label{fig:dendograma_input_size}
% \end{figure*}

% \begin{figure*}[h]
%     \centering
%     \includegraphics[width=\textwidth]{dendograma_floating.png}
%     \caption{Dendogram floating}
%     \label{fig:dendograma_floating}
% \end{figure*}

This tool and all the analysis are available on github: \url{github.com/VitorRamos/performance_features}

% \begin{lstlisting}[language=bash]
% pip install performance-features
% \end{lstlisting}

%\subsection{ACKNOWLEDGMENT}
%
%This research was developed in the super-computing centers of the University of MONS in collaboration with the center of the Federal University of Rio Grande do Norte.

	\section{Pascal Analyzer validation} \label{sec:pascal_framework_validation}
In this Section, we discuss the measurements and simulation results for three distinct cases, including two real applications. We also demonstrate how external libraries and standalone visualization tools can be used to render the collected data and investigate performance aspects such as the program's scalability capacity.

We used three experiments to assess the tool and demonstrate its ability to support analysis aimed at observing parallel scalability. The first includes a runtime imbalance between processing units in a specific parallel region. In this case, the objective is to present how the analyzer helps users observe the impact of the inefficiency of a code part on the whole program. The code for this experiment is presented in \cref{lst:codes_regionscomparison}. It consists of two simple parallel regions with the same functionality that divide the iterations of a loop between the available threads. The difference between the regions lies in the strategies used to manipulate the sum variable, used in the example to store the values of the calculations performed in each thread. We assume that the {\tt anyop()} function, in this case, invariably has the same runtime in all function calls.

The command presented on \cref{lst:pascal_usage_ex} was used as base to perform experiments. For the experiment of \cref{lst:codes_regionscomparison}, we use just the parameters {\tt -c}, {\tt -r}, {\tt -t}, {\tt -a} and {\tt -o}, including the 64 value to {\tt -c} option.

\lstset{style=ccodestyle, frame=tb}
\begin{lstlisting}[label={lst:pascal_usage_ex}, language=bash, caption={Command line showing how experiments were run through a terminal.}]
	analyzer 
	application
	-c 1,2,4,8,...,32 # threads/cores
	-r 3 # number of repetitions
	-t auto # automated instrumentation
	-a 1 # aggregation mode
	--ipmi ip user psswd # energy sensor
	--idtm 5 # idle time between runs
	--dhpt # disable hyper-thread
	--dcrs # disable cores
	--ipts ... # application specific inputs
	-o application.json # output file
\end{lstlisting}

\lstset{style=ccodestyle, frame=tb}
\begin{lstlisting}[label={lst:codes_regionscomparison}, language=C, caption={Sample code used to visualize the impact of regions on program scalability.}]
	int main(int argc, char **argv) {
		unsigned long sum = 0;
		int ops = atoi(argv[1]);
		
		#pragma omp parallel for schedule(static) reduction(+: sum)
		for (int i = 0; i < ops; i++)
		sum += anyop();
		
		#pragma omp parallel for schedule(static)
		for (int i = 0; i < ops; i++) {
			#pragma omp critical
			sum += anyop();
		}
	}
\end{lstlisting}

The command return a .json file with all the information necessary for our analysis. We can quickly generate tables with the collected data from this file, thus supporting observation and analysis of scalability, energy trends, and model fitting. The code described in \cref{lst:regtab} and \cref{tab:regtable_dedup} are samples of how users can easily read and view collected data from the control terminal in a tabular way.

\lstset{style=pythonStyle, frame=tb}
\begin{lstlisting}[label={lst:regtab}, language=python, captionpos=b, caption={Example of using the Python API to load analyzer files.}]
	from analyzer import Data
	
	data = Data("application.json")
	data.energy(regions=True)
	# data.speedup()
	# data.efficiency()
	# data.regions()
\end{lstlisting}

%The result can be seen in tabular form, as shown in Table.~\ref{tab:regtable_dedup}. Using these tables, we can quickly generate figures showing the application's behavior from various points of view. In Figures \ref{fig:speedup_raytrace} and \ref{fig:speedup_openmc} we observe the trend of the speedup with the number of cores and input.
\begin{table}[H]
	\caption{Dataframe generated automatically from collected samples using the Python API.}
	\label{tab:regtable_dedup}
	
	\resizebox{\columnwidth}{!}{%
		\begin{tabular}{ccccccc}
			\toprule
			\textbf{Repetition} & \textbf{Input} & \textbf{Cores} & \textbf{Regions} & \textbf{Start Time (s)} & \textbf{Stop Time (s)} & \multicolumn{1}{c}{\textbf{ipmi Energy (J)}} \\ \midrule
			1 & 1 & 1 & 1 & 0.00 & 66.31 & 13,156.19 \\
			1 & 1 & 1 & 2 & 0.00 & 66.27 & 13,148.87 \\
			... & ... & ... & ... & ... & ... & ... \\
			1 & 5 & 30 & 4 & 0.00 & 8.46 & 1903.82 \\
			1 & 5 & 30 & 5 & 0.02 & 58.16 & 13,067.01 \\ \bottomrule
	\end{tabular}}
\end{table}

\subsection{Case of study}\label{subsec:pfv_case_of_study}

The analyzer does not display graphics and other visual elements natively. However, the simple use of external libraries allows generating graphics and visualizing points of the program's behavior that you want to observe. In addition, specialized visualization tools can also be used to view the results. PaScal Viewer \cite{Silva2018}, for example, natively interprets the analyzer's output files, complementing its functionality and creating an integrated and appropriate environment for the program's parallel scalability analysis.

To analyze the experiment presented in \cref{lst:codes_regionscomparison}, we use PaScal Viewer to observe in a hierarchical way how the different OpenMP clauses impacted the region's efficiencies. The efficiency pinpoints how a program can take advantage of increasing processing elements on a parallel system. It is defined as the ratio of speedup to the number of processing units. In this sense, PaScal Viewer displays an efficiency diagram for each analysis region, as presented in \cref{fig:pv_regionscomparison}. Comparing these diagrams, it is possible to see how the critical clause damages the scalability. In addition, it is also possible to relate that this second region affects the efficiency of the whole program due to the code characteristics.

If the reduction clause replaces {\tt \#pragma~omp~crtical}, the second region and the entire program become more efficient, as shown in \cref{fig:pv_regionscomparison_2}. The PaScal Viewer diagrams mentioned in this work, the x-axis (i1--i7) corresponds to different data inputs and the y-axis (2--64) to the number of threads used in the processing. Figures~\ref{fig:pv_regionscomparison} and \ref{fig:pv_regionscomparison_2} were rendered using a tool feature that smoothes the color transition of diagrams. This feature uses interpolation to create a visual element where the color transition is less pronounced. The diagram axes only show the initial and final values with this option. The user can visualize diagrams with only discrete values or even compare the two presentation modes side by side. \cref{fig:visualizationmodes} shows the difference between discrete and smoothed modes considering the whole program and the simulation scenario that uses the clause {\tt \#pragma~omp~crtical}.
\begin{figure}[H]
	\begin{subfigure}[b]{0.45\textwidth}
		\includegraphics[width=\textwidth]{pascalanalyzer/figures/results/regionscomparison_r1.pdf}
%		\caption{\centering}
		\label{fig:pv_regionscomparison_a}
	\end{subfigure}
	\begin{subfigure}[b]{0.45\textwidth}
		\includegraphics[width=\textwidth]{pascalanalyzer/figures/results/regionscomparison_r2.pdf}
%		\caption{\centering}
		\label{fig:pv_regionscomparison_b}
	\end{subfigure}
	\begin{subfigure}[b]{0.45\textwidth}
		\includegraphics[width=\textwidth]{pascalanalyzer/figures/results/regionscomparison_whole.pdf}
%		\caption{\centering}
		\label{fig:pv_regionscomparison_c}
	\end{subfigure}
	\caption{Efficiency diagrams and impact of inner regions on program scalability. (x-axis = inputs; y-axis = number of threads). (\textbf{a}) First region diagram. (\textbf{b}) Second region diagram. (\textbf{c}) Whole program~diagram.}
	\label{fig:pv_regionscomparison}
\end{figure}
\unskip
\begin{figure}[H]
	\begin{subfigure}[b]{0.45\textwidth}
		\includegraphics[width=\textwidth]{pascalanalyzer/figures/results/efficiency_rg_1.pdf}
%		\caption{\centering}
		\label{fig:pv_regionscomparison_a_2}
	\end{subfigure}
	\begin{subfigure}[b]{0.45\textwidth}
		\includegraphics[width=\textwidth]{pascalanalyzer/figures/results/efficiency_rg_2.pdf}
%		\caption{\centering}
		\label{fig:pv_regionscomparison_b_2}
	\end{subfigure}
	\begin{subfigure}[b]{0.45\textwidth}
		\includegraphics[width=\textwidth]{pascalanalyzer/figures/results/efficiency_rg_3.pdf}
%		\caption{\centering}
		\label{fig:pv_regionscomparison_c_2}
	\end{subfigure}
	\caption{Efficiency diagrams after removing {\tt \#pragma~omp~crtical} clause. (x-axis = inputs; y-axis = number of threads). (\textbf{a}) First region diagram. (\textbf{b}) Second region diagram. (\textbf{c}) Whole program~diagram.}
	\label{fig:pv_regionscomparison_2}
\end{figure}
\unskip
\begin{figure}[H]
	\begin{subfigure}[b]{0.45\textwidth}
		\includegraphics[width=\textwidth]{pascalanalyzer/figures/results/diagram_discretevalues.pdf}
%		\caption{\centering}
		\label{fig:discretevalues}
	\end{subfigure}
	%
	\begin{subfigure}[b]{0.45\textwidth}
		\includegraphics[width=\textwidth]{pascalanalyzer/figures/results/diagram_smoothedvalues.pdf}
%		\caption{\centering}
		\label{fig:smoothedvalues}
	\end{subfigure}
	
	\caption{Visualization modes of diagrams provided by PaScal Viewer. (\textbf{a}) PaScal Viewer diagram on discrete mode. (\textbf{b}) PaScal Viewer diagram on smoothed mode.}
	\label{fig:visualizationmodes}
\end{figure}


%The analyzer allows that the result can be seen in tabular form, as shown in Table.~\ref{tab:regtable_dedup}. Using these tables, we can quickly generate figures that show the application's behavior from various points of view. In Figures \ref{fig:speedup_raytrace} and \ref{fig:speedup_openmc} we can observe the trend of the speedup with the number of cores and input.

%Figures \ref{fig:raytrace_en} and \ref{fig:openmc_en} present the energy variation with the number of cores.

\textls[-19]{Other analysis objectives not supported by PaScal Viewer, such as visualization of the speedup curve or energy consumption, can be supported by traditional plots. Figures~\ref{fig:speedup} and \ref{fig:energy}} present the charts rendered for analysis by Raytrace and OpenMC applications. In these figures, it is possible to observe that the program efficiency varies according to the increase in the number of threads and the execution of inputs with higher processing loads. From the diagrams in \cref{fig:efficiency_all}, it is possible to observe that the applications present different behaviors concerning their efficiency variations and their scalability capabilities. OpenMC maintains its efficiency almost constant. This pattern represents strong scalability and indicates that the program can maintain its efficiency level when it uses a larger number of threads and processing larger inputs. On the other hand, \cref{fig:efficiency_all}a demonstrates that Raytrace achieves higher efficiency values when processing larger inputs. However, it is also possible to observe that increasing the number of processing units fixing the input size will not improve or hold the efficiency value.
\begin{figure}[H]
	\begin{subfigure}[b]{0.45\textwidth}
		\includegraphics[width=\textwidth]{pascalanalyzer/figures/results/speedup_completo_rtview_2 (1).pdf}
%		\caption{\centering}
		\label{fig:speedup_raytrace}
	\end{subfigure}
	%
	\begin{subfigure}[b]{0.45\textwidth}
		\includegraphics[width=\textwidth]{pascalanalyzer/figures/results/speedup_completo_openmc_kernel_novo (1).pdf}
%		\caption{\centering}
		\label{fig:speedup_openmc}
	\end{subfigure}
	\caption{\textls[-30]{Speedup of the applications for several input sizes. (\textbf{a}) Raytrace speedup. (\textbf{b}) OpenMC speedup. }}
	\label{fig:speedup}
\end{figure}
\unskip
\begin{figure}[H]
	\begin{subfigure}[b]{0.46\textwidth}
		\includegraphics[width=\textwidth]{pascalanalyzer/figures/results/energy_completo_rtview_2 (1).pdf}
%		\caption{\centering}
		\label{fig:raytrace_en}
	\end{subfigure}
	%
	\begin{subfigure}[b]{0.46\textwidth}
		\includegraphics[width=\textwidth]{pascalanalyzer/figures/results/energy_completo_openmc_kernel_novo (1).pdf}
%		\caption{\centering}
		\label{fig:openmc_en}
	\end{subfigure}
	
	\caption{Energy consumption identified in the execution of applications while varying the number of cores for several input sizes.
		%MDPI: Please add commas to numbers to indicate thousand on the Oy axis of the figures, e.g. 20,000 30,000 40,000.
		(\textbf{a}) Raytrace energy consumption. (\textbf{b}) OpenMC energy consumption.}
	\label{fig:energy}
\end{figure}
\unskip
%Furthermore, using an external graphical interface module, we can load the .json file and intuitively visualize the program's scalability showing the efficiency. The efficiency pinpoint how efficiently a program can take advantage of increasing processing elements on a parallel system. It is defined as the ratio of speedup to the number of processors.
%For example, Figure \ref{fig:efficiency_all} shows examples of graphics generated for the same applications as above. In these figures, it is possible to observe that the program efficiency varies according to the increase in the number of threads and the execution of inputs with higher processing loads. From the Figure \ref{fig:efficiency_raytrace} we can see that as we increase the input size, the efficiency increases while slowly decreasing when the number of cores grows, indicating a tendency for a scalable program. In the second Figure \ref{fig:efficiency_openmc} the efficiency is practically constant while varying the number of cores and the input indicating a strong scalable program.
\begin{figure}[H]
	\begin{subfigure}[b]{0.46\textwidth}
		\includegraphics[width=\textwidth]{pascalanalyzer/figures/results/efficiency_raytrace.png}
%		\caption{\centering}
		\label{fig:efficiency_raytrace}
	\end{subfigure}
	\begin{subfigure}[b]{0.46\textwidth}
		\includegraphics[width=\textwidth]{pascalanalyzer/figures/results/efficiency_openmc.png}
%		\caption{\centering}
		\label{fig:efficiency_openmc}
	\end{subfigure}
	
	\caption{Efficiency diagram varying the input size and the number of cores. The color bar indicates the efficiency value in percentage. (\textbf{a}) Raytrace efficiency diagram. (\textbf{b}) OpenMC efficiency diagram.}
	\label{fig:efficiency_all}
\end{figure}

The input values shown in Figures~\ref{fig:speedup}--\ref{fig:efficiency_all} indicate different data sets for processing. For example, the ``input 2'' represents a load that will require sequential processing with runtime twice as long as ``input 1''. Likewise, the ``input 3'' represents a load that will require sequential processing with runtime twice as long as ``input 2'', and so on.

Even if the analyzer does not display graphics natively, analysis aimed at observing scalability and energy consumption of applications depends on precise measurements, reinforcing the advantage of the analyzer proposed in this work.

% \begin{figure}[H]
% \centering
% \includegraphics[width=\linewidth]{pascalanalyzer/figures/results/efficiency_raytrace.png}
% \caption{Efficiency for Raytrace}
% \label{fig:efficiency_raytrace}
% \end{figure}

% \begin{figure}[H]
% \centering
% \includegraphics[width=\linewidth]{pascalanalyzer/figures/results/efficiency_raytrace_detailed.png}
% \caption{Efficiency for Raytrace}
% \label{fig:efficiency_raytrace_detailed}
% \end{figure}

% \begin{figure}[H]
% \centering
% \includegraphics[width=\linewidth]{pascalanalyzer/figures/results/efficiency_openmc.png}
% \caption{Efficiency for OpenMC}
% \label{fig:efficiency_openmc}
% \end{figure}

% \begin{figure}[H]
% \centering
% \includegraphics[width=\linewidth]{pascalanalyzer/figures/results/efficiency_openmc_detailed.png}
% \caption{Efficiency for OpenMC}
% \label{fig:efficiency_openmc_detailed}
% \end{figure}
\section{PMC Module validation} \label{sec:fingerprint_tool_validation}

In this subsection, we first show the comparison between our tool and others already established, as well as the clustering process.

%\subsection{Accuracy comparison} \label{subsec:ftv_accuracy_comparison}

% \begin{table*}[h]
% \centering
% \caption{Average}
% \begin{tabular}{|c|c|c|c|c|c|c|}
% \hline
% Counter                                 & Pined values & Linux API   & PAPI      & PAPI Python & Perf tool   & MyPerf      \\ \hline
% INSTRUCTIONS\_RETIRED                  & 226990030    & 227000691   & 227000620 & 225901249   & 227000572   & 227000650   \\ \hline
% BRANCH\_INSTRUCTIONS\_RETIRED          & 9240000      & 9250617     & 9250566   & 9239617     & 9250552     & 9250501     \\ \hline
% BR\_INST\_RETIRED:CONDITIONAL          & 8220000      & 8220000     & 8220000   & 8209717     & 8220000     & 8220000     \\ \hline
% MEM\_UOP\_RETIRED:ANY\_LOADS           &              & 2484182672  &           &             & 2484383940  & 2484155029  \\ \hline
% MEM\_UOP\_RETIRED:ANY\_STORES          &              & 189962002   &           &             & 189961539   & 189961321   \\ \hline
% UOPS\_RETIRED:ANY                      &              & 12291082129 &           &             & 12290811038 & 12290901997 \\ \hline
% PARTIAL\_RAT\_STALLS:MUL\_SINGLE\_UOP  &              & 600878      &           &             & 600151      & 600330      \\ \hline
% ARITH:FPU\_DIV                         &              & 5801446     &           &             & 5801000     & 5800977     \\ \hline
% FP\_COMP\_OPS\_EXE:X87                 &              & 48785528    &           &             & 48784834    & 48786021    \\ \hline
% INST\_RETIRED:X87                      &              & 17200008    &           &             & 17200008    & 17200007    \\ \hline
% FP\_COMP\_OPS\_EXE:SSE\_SCALAR\_DOUBLE &              & 5401694     &           &             & 5401842     & 5401679     \\ \hline
% \end{tabular}
% \label{tab:mean}
% \end{table*}

% \begin{table*}[h]
% \centering
% \caption{Standard deviation}
% \begin{tabular}{|c|c|c|c|c|c|}
% \hline
% Counter                                    & Linux API & PAPI & PAPI Python & Perf tool & MyPerf \\ \hline
% INSTRUCTIONS\_RETIRED                  & 396       & 133  & 337763      & 110       & 175    \\ \hline
% BRANCH\_INSTRUCTIONS\_RETIRED          & 297       & 208  & 8485        & 379       & 91     \\ \hline
% BR\_INST\_RETIRED:CONDITIONAL          & 0         & 0    & 3383        & 0         & 0      \\ \hline
% MEM\_UOP\_RETIRED:ANY\_LOADS           & 37399     &      &             & 39217     & 38953  \\ \hline
% MEM\_UOP\_RETIRED:ANY\_STORES          & 1513      &      &             & 1035      & 687    \\ \hline
% UOPS\_RETIRED:ANY                      & 345246    &      &             & 335832    & 333298 \\ \hline
% PARTIAL\_RAT\_STALLS:MUL\_SINGLE\_UOP  & 1222      &      &             & 252       & 521    \\ \hline
% ARITH:FPU\_DIV                         & 1760      &      &             & 1621      & 1544   \\ \hline
% FP\_COMP\_OPS\_EXE:X87                 & 1283      &      &             & 1920      & 3311   \\ \hline
% INST\_RETIRED:X87                      & 4         &      &             & 4         & 3      \\ \hline
% FP\_COMP\_OPS\_EXE:SSE\_SCALAR\_DOUBLE & 1547      &      &             & 3259      & 2097   \\ \hline
% \end{tabular}
% \label{tab:std}
% \end{table*}

\begin{table}[H]
	\centering
	\caption{Comparison}
	\resizebox{\textwidth}{!}{%
		\begin{tabular}{llllllllll}
			\hline
			\multicolumn{6}{l|}{\textbf{Average*$10^{-6}$}} & \multicolumn{4}{l}{\textbf{Standard deviation}}\\ \hline
			\textbf{Counters} & \begin{tabular}{l}\textbf{Pined} \\ \textbf{values}\end{tabular} & \begin{tabular}{l}\textbf{Linux} \\ \textbf{API}\end{tabular} & \textbf{PAPI} & \begin{tabular}{l}\textbf{PAPI} \\ \textbf{Python}\end{tabular} & \textbf{MyPerf}  & \begin{tabular}{|l}\textbf{Linux} \\ \textbf{API}\end{tabular} & \textbf{PAPI} & \begin{tabular}{l}\textbf{PAPI} \\ \textbf{Python}\end{tabular} & \textbf{MyPerf} \\ \hline
			INSTRUCTIONS\_RETIRED                  & 226.99       & 227       & 227  & 225.9       & 227     & 396       & 133  & 337763      & 175    \\
			BRANCH\_INSTRUCTIONS\_RETIRED          & 9.24         & 9.25      & 9.25 & 9.24        & 9.25    & 297       & 208  & 8485        & 91     \\
			BR\_INST\_RETIRED:CONDITIONAL          & 8.22         & 8.22      & 8.22 & 8.21        & 8.22    & 0         & 0    & 3383        & 0      \\
			MEM\_UOP\_RETIRED:ANY\_LOADS           &              & 2484.18   &      &             & 2484.16 & 37399     &      &             & 38953  \\
			MEM\_UOP\_RETIRED:ANY\_STORES          &              & 189.96    &      &             & 189.96  & 1513      &      &             & 687    \\
			UOPS\_RETIRED:ANY                      &              & 12291.08  &      &             & 12290.9 & 345246    &      &             & 333298 \\
			PARTIAL\_RAT\_STALLS:MUL\_SINGLE\_UOP  &              & 0.6       &      &             & 0.6     & 1222      &      &             & 521    \\
			ARITH:FPU\_DIV                         &              & 5.8       &      &             & 5.8     & 1760      &      &             & 1544   \\
			FP\_COMP\_OPS\_EXE:X87                 &              & 48.79     &      &             & 48.79   & 1283      &      &             & 3311   \\
			INST\_RETIRED:X87                      &              & 17.2      &      &             & 17.2    & 4         &      &             & 3      \\
			FP\_COMP\_OPS\_EXE:SSE\_SCALAR\_DOUBLE &              & 5.4       &      &             & 5.4     & 1547      &      &             & 2097   \\ \hline
		\end{tabular}
	}
	\label{tab:counters}
\end{table}

To validate the tool, we compared the results of the counters obtained with different APIs. 
We used the hand-crafted assembly benchmark from \cite{Weaver2013Non-determinismImplementations}, designed to test determinism and accuracy of PMUs.
We compared the values obtained from the Linux API, PAPI on C and PAPI on Python. 
The events used for this comparison were instructions retired, branch instructions, memory read, memory load, and arithmetic operations.
We ran the benchmark 30 times and calculated the mean and standard deviation as shown in table \ref{tab:counters}. Some events could not be measured using PAPI because the tool does not accept raw events and there are no equivalent events.

Since the benchmark was hand-crafted with assembly, we know exactly the value for some counter events. 
For this reason, the number of instructions, branch instructions, and conditional branch are pin. However, some other events are architecture-specific and there is no pined value. 
In the latter case, we can still compare to the Linux API, which should be closest to the reality.

The differences using the Linux low-level API, PAPI, and our tool are negligible (the average percentage distance is less than 0.01\% in all the cases). 
As expected, PAPI on Python had the largest difference (with an average distance of 0.25\%) mainly due to an unsynchronized start that resulted in the loss of some instructions at the beginning of the execution. 
This can be an important problem if the application contains a small number of instructions.

The standard deviation of the 30 executions shows that our tool has the smallest variation on a run-to-run on most events. On the contrary, PAPI on Python shows a big variation compared to the others.

\subsubsection{Case of study} \label{subsec:ftv_case_of_study}

To cluster applications first we need to have a way to compare two programs, for that we define a new variable that tries to compute a fingerprint to the program, this variable has to look similar when we execute the same program with different conditions and inputs. Thinking in the simplest model of a program as a Turing machine as everything can be done with a tap of memory and set of rules we empirically define the variable input size as described on the equation \ref{eq:input_size}. On real computers this is not too far from reality, most of the computations and input and output operations somehow pass through the memory, so analyzing the relationship of the total number of instructions and memory instructions can give us a good fingerprint.

\begin{equation}
	I_{sz} = \frac{I}{I_{m}} \\
	\label{eq:input_size}
\end{equation}

Where $I_{sz}$ we called input size, $I$ the number of instructions executed, $I_m$ number of memory instructions execute. We observe that this variable demonstrate to have the proprieties that we are looking for, produce similar results to the same program with different input size, and environment. This can be used to identify programs but also to see the similarities between different applications.

To compute the distance between two programs we use the Canberra metric \cite{Jurman2009CanberraLists}, described on the equation \ref{eq:canberra}.

\begin{equation}
	d(p,q)= \sum_{i=1}^{n}{\frac{|p_i-q_i|}{|p_i|+|q_i|}}
	\label{eq:canberra}
\end{equation}

Where $p$ and $q$ are n-dimensional vectors.

We compute the input size for all 30 applications of the Polybench with 3 different inputs. Running each program 15 times, with a sampling rate of 0.01 seconds collecting the total number of instruction, number of loads and write instructions, number of floating point operations, besides some software counters. After applying the post-processing to the data collected we compute the input size and perform a Hierarchical clustering using the linkage method of Ward \cite{Murtagh2011WardsAlgorithm}, that minimizes the total within-cluster variance. The results of the clustering can be seen on the figure \ref{fig:sping_force_is} and on the dendrogram on figure \ref{fig:dendograma_input_size}.

%The exact HPCs was:
%PERF_COUNT_HW_INSTRUCTIONS,
%MEM_UOPS_RETIRED:ALL_LOADS,
%MEM_UOPS_RETIRED:ALL_STORES,
%FP_ARITH_INST_RETIRED:SCALAR

\begin{figure*}[h]
	\centering
	\includegraphics[width=\textwidth]{fingerprint/figures/dendograma_input_size.png}
	\caption{Dendrogram}
	\label{fig:dendograma_input_size}
\end{figure*}

From this dendrogram, we can have an idea of how close two applications are. We choose the number of clusters that maximize the number of hits of the same program with different input sizes on the same cluster, in this case, was 5 clusters.

It was observed that as input size grows the program behavior tends to a specific curve, but for small input sizes some have variation, so in some cases, the same program has been classified in more than one cluster. This can also happen if specific parts of the program are triggered with specific inputs, in which case it will also belong to more than one cluster.

In order to have an overall classification of each program, we can pick up the frequency in which each appeared in the clusters and classified it in the cluster in which it appeared more often. In this case, the clusters are:

\begin{itemize}
	\item Cluster 1: 2mm, 3mm, cholesky, correlation, covariance, floyd-warshall, gemm, gramschmidt, lu, ludcmp, nussinov, symm
	
	\item Cluster 2: deriche, doitgen, syrk
	
	\item Cluster 3: adi, fdtd-2d, jacobi-2d, syr2k
	
	\item Cluster 4: atax, bicg, durbin, gemver, gesummv, mvt, trisolv, trmm
	
	\item Cluster 5: heat-3d, seidel-2d
\end{itemize}

On the figure, \ref{fig:sping_force_is} we display the clusters using the spring force algorithm where each program with a specific input is a node and the edge weight is the Canberra distance. To a better visualization, the name of the inputs was replaced by numbers, the EXTRALARGE is 3, LARGE 2 and MEDIUM 1. 

\begin{figure}[H]
	\centering
	\includegraphics[width=\textwidth]{fingerprint/figures/graph_input_size.png}
	\caption{Spring force of Canberra distance}
	\label{fig:sping_force_is}
\end{figure}

The figure above shows a different way to observe in a reduced space the distance between clusters and applications and how they are organized. From this graph, we see that clustering it's well partitioned and we can clearly separate each cluster. It is also interesting to note that the applications of the cluster formed by the circle symbol are more separated, which may indicate that they were classified in this way because they did not fit into any other cluster.

% \begin{figure*}[t]
%     \begin{subfigure}{\textwidth}
%         \includegraphics[width=\textwidth]{fingerprint/figures/dendograma_input_size.png}
%     \end{subfigure}
%     \\
%     \begin{subfigure}{\textwidth}
%         \includegraphics[width=\textwidth]{fingerprint/figures/dendograma_floating.png}
%     \end{subfigure}
%     %and so on
% \end{figure*}

To have an idea of how the behavior of the variable input size for each clusters is, it was plotted for the applications of clusters 1 and 2 shown on the figures \ref{fig:c1_input_size_0}, \ref{fig:c1_input_size_1}.

\begin{figure}[H]
	\centering
	\includegraphics[width=\textwidth]{fingerprint/figures/cluster_input_0.png}
	\caption{Input size - Cluster 1}
	\label{fig:c1_input_size_0}
\end{figure}

From these figures, we can have an idea of what behavior was classified as the same class. On cluster 1 on the figure \ref{fig:c1_input_size_0} all applications showed the same behavior with approximately the same amplitude.

\begin{figure}[H]
	\centering
	\includegraphics[width=\textwidth]{fingerprint/figures/cluster_input_1.png}
	\caption{Input size - Cluster 2}
	\label{fig:c1_input_size_1}
\end{figure}

On figure \ref{fig:c1_input_size_1} we observe that curves that have similar shape regardless of scale on vertical and horizontal axis have also been classified as the same clusters. This is the wanted comportment for the classification because we are only interested in the overall shape of the curve.


% \begin{figure}[H]
%     \centering
%     \includegraphics[width=\textwidth]{fingerprint/figures/cluster_input_4.png}
%     \caption{Input size - Cluster 3}
%     \label{fig:c1_input_size_2}
% \end{figure}

% To conclude this study we also clustering applications by its floating point behavior. For that we only use the counter correspondent to floating point operations. The figure \ref{fig:dendogram_fp} shows the dendrogram and the \ref{fig:sping_force_fp} the spring force graph plot.

% \begin{figure*}[h]
%     \includegraphics[width=\textwidth]{fingerprint/figuresdendograma_floating.png}
%     \caption{Dendrogram}
%     \label{fig:dendogram_fp}
% \end{figure*}

% \begin{figure}[H]
%     \centering
%     \includegraphics[width=\textwidth]{fingerprint/figures/graph_floating2.png}
%     \caption{Spring force of Canberra distance}
%     \label{fig:sping_force_fp}
% \end{figure}

% The figures \ref{fig:cluster_fp1},\ref{fig:cluster_fp2},\ref{fig:cluster_fp3} show the behavior on the same cluster for the float point operations. For this classification the number of clusters was 8.

% \begin{figure}[H]
%     \centering
%     \includegraphics[width=\textwidth]{fingerprint/figures/cluster_fp_0.png}
%     \caption{Floating point - Cluster 1}
%     \label{fig:cluster_fp1}
% \end{figure}

% \begin{figure}[H]
%     \centering
%     \includegraphics[width=\textwidth]{fingerprint/figures/cluster_fp_4.png}
%     \caption{Floating point - Cluster 2}
%     \label{fig:cluster_fp2}
% \end{figure}

% \begin{figure}[H]
%     \centering
%     \includegraphics[width=\textwidth]{fingerprint/figures/cluster_fp_5.png}
%     \caption{Floating point - Cluster 3}
%     \label{fig:cluster_fp3}
% \end{figure}

% Here we can also observed again that applications with the same shape were classified as the same clusters.


	
	\chapter{Application energy and performance models} \label{chapter:models}
	Energy consumption is crucial in high-performance computing (HPC), especially to enable the next exascale generation. Hence, modern systems implement various hardware and software features for power management. Nonetheless, due to numerous different implementations, we can always push the limits of software to achieve the most efficient use of our hardware. To be energy efficient, the software relies on dynamic voltage and frequency scaling (DVFS), as well as dynamic power management (DPM). Yet, none have privileged information on the hardware architecture and application behavior, which may lead to energy-inefficient software operation. 
	This chapter proposes analytical modeling for architecture and application behavior that can be used to estimate energy-optimal software configurations and provide knowledgeable hints to improve DVFS and DPM techniques for single-node HPC applications.
	Additionally, model parameters, such as the level of parallelism and dynamic power, provide insights into how the modeled application consumes energy, which can be helpful for energy-efficient software development and operation.
	This novel analytical model takes the number of active cores, the operating frequencies, and the input size as inputs to provide energy consumption estimation.
	In this chapter we present the modeling of 13 parallel applications employed to determine energy-optimal configurations for several different input sizes.
	The results show that up to 70\% of energy could be saved in the best scenario compared to the default Linux choice and 14\% on average.
	We also compare the proposed model with standard machine-learning modeling concerning training overhead and accuracy. The results show that our approach generates about 10 times less energy overhead for the same level of accuracy.
	%\section{Power Model} \label{sec:powermodel}
%Some of the main factors that contribute to the CPU power consumption are the dynamic power consumption, the short-circuit power consumption, and the power loss due to the transistor leakage current, \cite{Rauber2014, Goel2016, Du2017, Gonzalez1997}. The complexity of the circuits of modern processors makes it very difficult to consider all the components and interconnections. A viable approach for modeling the CPU's power draw is to model their building components, which are mainly made out of logic gates. Thus, modeling the power consumption can be resumed to model the logic gates and multiplying this by the total number of gates, reducing the complexity of the modeling process. %Although this is a huge simplification, it can still provide sufficient accuracy, bellow 5\%
%
%The mature technology to manufacture logic gates is CMOS. Nowadays it has been replaced by FINFET. In general, in these technologies, there are three main components of power dissipation \cref{eq:power_breakdown}. Namely, static power $P_{\rm static}$, dynamic power $P_{\rm dynamic}$, and leakage power $P_{\rm leak}$, that accumulated compose the total power draw.
%\begin{equation}
%P_{total}=P_{\rm static}+P_{\rm leak}+P_{\rm dynamic}
%\label{eq:power_breakdown}
%\end{equation}
%
%The dynamic power and leakage power behavior can be approximated by \cref{eq:power_dyn} and \cref{eq:power_leak} respectively~\cite{Sarwar1997, Butzen2007}.
%\begin{equation}
%P_{dynamic}=CV^2f,
%\label{eq:power_dyn}
%\end{equation}
%\begin{equation}
%P_{leak} \propto V,
%\label{eq:power_leak}
%\end{equation}
%where $C$ is the transistor capacitance, $V$ the voltage applied to the circuit and $f$ the switching frequency.
%
%Another common approximation is to expect a linear relationship between the voltage and the applied frequency~\cite{Usman2013ANoC} such that \ref{eq:f_v}
%\begin{equation}
%f \propto V.
%\label{eq:f_v}
%\end{equation}
%Thus, the proposed model for one processing core of a multi-core processor is derived by using (\cref{eq:power_dyn}), (\cref{eq:power_leak}) and (\cref{eq:f_v}) to rewrite (\cref{eq:total_power}) as the following equation \ref{eq:total_power}.
%\begin{equation}
%P_{total}(f)= c_1f^3+c_2f+c_3,
%\label{eq:total_power}
%\end{equation}
%where $c_1$, $c_2$, and $c_3$ are the model's parameters. Including the number of active cores $p$, the estimation of the power consumption of the whole processor becomes \ref{eq:power_final}:
%\begin{equation}
%P_{total}(f,p)= p(c_1f^3+c_2f)+c_3.
%\label{eq:power_final}
%\end{equation}
%
%\section{Performance Model} \label{sec:performancemodel}
%To model the application execution time, we consider a program as a set of instructions that are executed on a mean frequency $f$ with $c_k$ instructions per cycle. The time $T_f$ that this program will take to complete in at a given frequency is devised as follows.
%\begin{equation}
%T_f=\frac{I}{c_kf},
%\label{eq:freqrel}
%\end{equation}
%where $I$ is the total number of instructions and $ck$ the ratio of instructions per unit of time.
%
%The next step is to include the number of cores in the equation. Amdahl's law, described in the equation \ref{eq:amdahl}, gives the theoretical background for that, it describes the speedup in latency of the execution of a task at a fixed workload.
%\begin{equation}
%S=\frac{T_s}{T_P}=\frac{1}{1-w+\frac{w}{p}},
%\label{eq:amdahl}
%\end{equation}
%where $S$ is the theoretical speedup of the execution of the whole task, $w$ s the proportion of execution time that the part benefiting from improved resources originally occupied, $p$ is the speedup of the part of the task that benefits from improved system resources. Combining this with \cref{eq:freqrel}, the parallel time at frequency $f$ can be written as:
%\begin{equation}
%T_p=\frac{T_s}{S}=\frac{T_f}{\frac{1}{1-w+\frac{w}{p}}}.
%\label{eq:parallel_time}
%\end{equation}
%
%Then, we write the execution-time equation of the program as a function of frequency, the number of cores and parallel proportion as \ref{eq:performance} and subsequently \ref{eq:performance_2}:
%\begin{equation}
%T(f,p)=\frac{I}{ \frac{c_kf}{1-w+\frac{w}{p}} },
%\label{eq:performance}
%\end{equation}
%\begin{equation}
%% T(f,p)=\frac{d_1 (1-w+\frac{w}{p}) }{f}
%T(f,p)=\frac{d_1(p-wp+w)}{fp}.
%\label{eq:performance_2}
%\end{equation}
%
%Finally, we include the input size so that the application is fully characterized, in \cite{Oliveira2018ApplicationCores} it was shown that generally this can be described with an exponential, as presented in \cref{eq:performance_final}.
%\begin{equation}
%% T(f,p)=\frac{d_1 (1-w+\frac{w}{p}) }{f}
%T(f,p,N)=\frac{d_1N^{d_2}(p-wp+w)}{fp},
%\label{eq:performance_final}
%\end{equation}
%where $d_1$ and $w$ are constants that depend on the application. This equation can describe the behavior of the execution time at any input $N$, frequency $f$ and active cores $p$.
%
%\section{Energy Model} \label{sec:energymodel}
%Combining the output of the power model described in~\cref{sec:powermodel} and the characterization of the application performance described in \cref{sec:performancemodel}, the total energy can be modeled as:
%\begin{equation}
%E(f,p,N)=P(f,p)\times{\rm T}(f,p,N),
%\label{eq:en_combination}
%\end{equation}
%where $P(f,p,s)$ is the total power modeled by~\cref{eq:power_final}, ${T}(f,p,N)$ is the execution time estimated by the \ref{eq:performance_final}, $f$ is the frequency, $p$ is the number of active cores, $s$ is the number of sockets, and $N$ is the input size. The final equation can be written as:
%\begin{equation}
%% E(f,p)=\frac{d_1((c_1f^3+c_2f)p+c_3)(1-w+\frac{w}{p}) }{f}
%E(f,p,N)=\frac{d_1N^{d_2}(p-wp+w)(p(c_1f^3+c_2f)+c_3)}{fp}.
%\label{eq:en_final}
%\end{equation}
%
%%%%%%%%%%%%%%%%%%%%%%%%%%%%%%%%%%%%%%%%%%%
%
%\section{Experimental validation} \label{sec:experimentalvalidation}
%In this section, the result of the models presented in~\cref{sec:powermodel} and~\cref{sec:performancemodel} were validated with a benchmark specific for multi-core architectures. The results were then compared with a machine learning approach, Support Vector Regression (SVR)~\cite{Smola2004}, to contrast the accuracy and overhead obtained with each case.
%
%\subsection{Case-Study Applications} \label{sec:casestudyapplication}
%The PARSEC parallel benchmark suite, version 3.0~\cite{Bienia2008}, OpenMC \cite{Romano2015OpenMC:Development} and LINPACK (HPL) \cite{Dongarra1988TheExplanation}, were chosen as cases of study. The PARSEC benchmark focus on emerging workloads and were designed to be representative of the next generation shared-memory programs for chip-multiprocessors. It covers an ample range of areas such as financial analysis, computer vision, engineering, enterprise storage, animation, similarity search, data mining, machine learning, and media processing. The OpenMC and the LINPACK applications are two classical HPC program.
%
%\subsection{Case-Study Architecture} \label{sec:casestudyarchitecture}
%The experiments were executed in one computer node equipped with two Intel Xeon E5-2698 v3 processors with sixteen cores each and two hardware threads for each core. The maximum non-turbo frequency is 2.3GHz, and the total physical memory of the node is 128GB (8$\times$16GB). Turbo frequency and hardware multi-threading were disabled during all experiments. The operating system used was Linux CentOS 6.5, kernel 4.16. The overview of the architecture is shown in figure \ref{fig:architecture}.
%\begin{figure}[h]
%	\centering
%	\includegraphics[width=\columnwidth]{models/models/figures/architecture.png}
%	\caption{Node architecture, the image was made with lstop application}
%	\label{fig:architecture}
%\end{figure}
%
%The Linux kernel has many different policies for power management depending on the driver. On the default driver the acpi-cpufreq, the options are:
%\begin{itemize}
%	\item Powersave
%	\item Performance
%	\item Ondemand
%	\item Conservative
%	\item Userspace
%\end{itemize}
%In this work, the frequency control was performed using the Userspace governor, and the core control was accomplished by modifying the appropriate system files with the default CPU-hotplug driver.
%
%This architecture is equipped with Intelligent Platform Management Interface (IPMI), which is a set of interfaces that allow out-of-band management of computer systems and platform-status monitoring through local network~\cite{Schwenkler2006IntelligentInterface}. It can monitor variables and resources such as the system's temperature, voltage, fans, and power supplies, with independent sensors attach to the hardware.
%
%\subsection{Fitting the Models} \label{sec:fitting}
%To find the parameters of the equation \ref{eq:en_final}, 10 random configurations of frequencies (f), cores (p) and inputs (N) were chosen from the range $1<=p<=32$, $1.2<=f<=2.2$ and $1<=N<=5$ respectively. The application ran with each configuration chosen and the results of energy and time were collected. The application's input was chosen in such a way that they increase workload linearly to the smallest workload to the biggest workload so that most of the input space is covered~\cite{Oliveira2018ApplicationCores}. For each configuration, samples of the power were collected using IPMI every 1 second. This sampling rate was chosen because the order of magnitude of the mean run time of the applications is minutes. Therefore, this rate provides enough samples to measure average power. Additionally, timestamps and the total run time were collected. The total energy spent on each configuration is estimated by integrating the power samples over time.
%
%From the sampled data, the arguments of the model can be estimated.  For the equation, this results in an optimization problem of finding the parameters that minimize the total distance between the estimated values and the measured values. To solve this minimization problem, non-linear least squared was applied.
%
%The python library scikit-learning was used as the SVR implementation~\cite{PedregosaF.VaroquauxG.GramfortA.2011Scikit-learn:Python}.The SVR was trained using the same data of the equation with a grid search used to find the best kernel function and the best values for the hyper-parameters penalty for the wrong ($C$) and ($\gamma$). For this data, the best function was the Radial Base Function (RBF) and the hyper-parameters were $C=10\times10^3$ and $\gamma=0.5$. 
%
%\subsection{Measured versus Modeled Energy} \label{sec:measuredversusmodeledenergy}
%To validate the proposed energy model, all possible configurations were tested varying the cores in a range of $1<=p<=32$, the frequency in $1.2<=f<=2.2$ and input in $1<=N<=5$. Then, the mean absolute error was computed as the difference between the estimated values and the measured values according to the following equation.
%\begin{equation}
%MAE = \frac{1}{N} \sum_i^N \frac{|y_{\rm estimated}-y_{\rm measured}|}{y_{\rm measured}}.
%\label{eq:mae}
%\end{equation}
%
%Figures \ref{fig:en_eq_black}, \ref{fig:en_eq_canneal}, \ref{fig:en_eq_dedup}, and~\ref{fig:en_eq_rtview} plot the measured and modeled energy consumption for some of the applications modeled. There show some of the possible shapes that the model can assume while varying the number of active cores, operating frequency, and input size.
%\begin{figure}[ht]
%	\centering
%	
%	\begin{subfigure}[b]{0.48\textwidth}
%		\centerline{\includegraphics[width=\columnwidth]{models/models/figures/energy/completo_black_5.png}}
%		\caption{Blackscholes}
%		\label{fig:en_eq_black}
%	\end{subfigure}
%	%
%	\begin{subfigure}[b]{0.48\textwidth}
%		\centerline{\includegraphics[width=\columnwidth]{models/models/figures/energy/completo_canneal_1.png}}
%		\caption{Canneal}
%		\label{fig:en_eq_canneal}
%	\end{subfigure}
%	
%\end{figure}
%
%\begin{figure}[ht]
%	\centering
%	
%	\begin{subfigure}[b]{0.48\textwidth}
%		\centerline{\includegraphics[width=\columnwidth]{models/models/figures/energy/completo_dedup_4.png}}
%		\caption{Dedup}
%		\label{fig:en_eq_dedup}
%	\end{subfigure}
%	%
%	\begin{subfigure}[b]{0.48\textwidth}
%		\centerline{\includegraphics[width=\columnwidth]{models/models/figures/energy/completo_rtview_4.png}}
%		\caption{Raytrace}
%		\label{fig:en_eq_rtview}
%	\end{subfigure}
%\end{figure}
%
%The validation results for each application trained with 10 random configurations are displayed in the figure \ref{fig:mae_svr_eq} and the raw MAE values in the table~\ref{tab:mae_svr_eq}.
%
%\begin{figure}[ht]
%	\includegraphics[width=\columnwidth]{models/models/figures/mae_svr_eq.png}
%	\caption{Comparison mean absolute error between the proposed model and SVR}
%	\label{fig:mae_svr_eq}
%\end{figure}
%
%\begin{table}[ht]
%	\centering
%	\begin{tabular}{|c|c|c|}
%		\hline
%		Application  & Model & SVR   \\ \hline
%		Ferret       & 5.25     & 12.49  \\ \hline
%		Raytrace     & 6.36     & 11.95  \\ \hline
%		Fluianimate  & 2.44     & 22.90  \\ \hline
%		x264         & 8.28     & 15.33  \\ \hline
%		Vips         & 7.54     & 10.80  \\ \hline
%		Swaptions    & 6.54     & 18.57  \\ \hline
%		Canneal      & 3.12     & 6.13   \\ \hline
%		Dedup        & 8.85     & 13.70  \\ \hline
%		Freqmine     & 2.44     & 3.24   \\ \hline
%		Blackscholes & 2.18     & 11.00  \\ \hline
%		HPL          & 7.47     & 12.75  \\ \hline
%		Bodytrack    & 16.98    & 34.12  \\ \hline
%		Openmc       & 11.15    & 24.34  \\ \hline
%	\end{tabular}
%	\caption{Comparison mean absolute error between the proposed model and SVR}
%	\label{tab:mae_svr_eq}
%\end{table}
%
%\subsection{Overhead on training}
%It is known that machine learning is data-driven, in that sense the results obtained using only 10 configurations could be improved, but how about the analytical model? To answer that question the proposed model and SVR were trained also with varying the number of configurations. Comparing the MAE and the amount of energy needed for each model. The figure \ref{fig:overhead_ferret}, \ref{fig:overhead_vips} show some of this comparisons for an individual application, while figure \ref{fig:overall_train} show the overall results for all applications.
%
%\begin{figure}[ht]
%	\centering
%	\begin{subfigure}[b]{0.45\textwidth}
%		\centerline{\includegraphics[width=\columnwidth]{models/models/figures/overhead/completo_ferret_4.png}}
%		\caption{MAE for Ferret}
%		\label{fig:overhead_ferret}
%	\end{subfigure}
%	%
%	\begin{subfigure}[b]{0.45\textwidth}
%		\centerline{\includegraphics[width=\columnwidth]{models/models/figures/overhead/completo_vips_4.png}}
%		\caption{MAE for Vips}
%		\label{fig:overhead_vips}
%	\end{subfigure}
%	\caption{MAE showing the that for the model is almost constant while for the SVR there are different crossing points for the lowest error}
%\end{figure}
%
%The analytical model demonstrates to be very stable, not changing a lot as more data are added, while the SVR keeps reshaping to adapt to the data. This reflects in the analytical model error being almost constant while the SVR drops meeting at some point.
%
%The figure \ref{fig:overall_train} present the overall results, computing the mean energy and MAE for all applications.
%
%\begin{figure}[ht]
%	\centering
%	\begin{subfigure}[b]{0.45\textwidth}
%		\centerline{\includegraphics[width=\columnwidth]{models/models/figures/overhead/overall_energy.png}}
%		\caption{Average energy spent over all applications}
%		\label{fig:overall_overhead}
%	\end{subfigure}
%	%
%	\begin{subfigure}[b]{0.45\textwidth}
%		\centerline{\includegraphics[width=\columnwidth]{models/models/figures/overhead/overall_mae.png}}
%		\caption{Average MAE over all applications}
%		\label{fig:overall_energy}
%	\end{subfigure}
%	\caption{Overall results for energy and MAE for each train size}
%	\label{fig:overall_train}
%\end{figure}
%
%The meeting point of the MAE for the SVR and the proposed model can be extracted from  \ref{fig:overall_overhead}. There it shows that approximately at 90 configurations the SVR starts to have a smaller error the cost of that is the linear increase in energy spent on training. The increase in energy of about $10\times$ can be observed in figure \ref{fig:overall_energy}.
%
%\subsection{Performance penalty}
%
%\subsection{Time to compensate energy spent on training}

\subsection{Power Model} \label{sec:powermodel}
The complexity of the modern processor's circuits makes it very difficult to consider all the components and interconnections. A viable approach for modeling the CPU's power draw is to model their building components, mainly made out of logic gates. Thus, modeling the power consumption can be resumed to model the logic gates and multiplying this by the total number of gates, reducing the complexity of the modeling process.

The mature technology to manufacture logic gates is CMOS. Nowadays, it has been replaced by FINFET. In general, in these technologies, there are three main components of power dissipation \cite{Rauber2014, Goel2016, Du2017, Gonzalez1997},  namely, static power $P_{\rm static}$, dynamic power $P_{\rm dynamic}$, and leakage power $P_{\rm leak}$, that accumulated compose the total power draw.
% \begin{equation}
% P=P_{\rm static}+P_{\rm leak}+P_{\rm dynamic}.
% \label{eq:power_breakdown}
% \end{equation}

The dynamic power and leakage power behavior can be approximated by \cref{eq:power_dyn} and \cref{eq:power_leak}, respectively, as shown by Sarwar et al. and Butzen et al~\cite{Sarwar1997, Butzen2007}.
\begin{equation}
P_{dynamic}=CV^2f,
\label{eq:power_dyn}
\end{equation}
\begin{equation}
P_{leak} \propto V,
\label{eq:power_leak}
\end{equation}
where $C$ is the load capacitance, $V$ the voltage applied to the circuit and $f$ the switching frequency.

Another common approximation is to expect a linear relationship between the voltage and the applied frequency~\cite{Usman2013ANoC} such that:
\begin{equation}
f \propto V,
\label{eq:f_v}
\end{equation}
Thus, the proposed model for one processing core of a multi-core processor is derived by using \cref{eq:power_dyn}, \cref{eq:power_leak} and \cref{eq:f_v} to write \cref{eq:total_power}.
\begin{equation}
P(f)= c_1f^3+c_2f+c_3,
\label{eq:total_power}
\end{equation}
where $c_1$ $c_2$, and $c_3$ are the model's parameters associated with the dynamic, leakage and static power aspects, respectively. Including the number of active cores $p$, the proposed estimation of the power consumption of the whole processor becomes \cref{eq:power_final}
\begin{equation}
P(f,p)= p(c_1f^3+c_2f)+c_3,
\label{eq:power_final}
\end{equation}

\subsection{Performance Model} \label{sec:performancemodel}
To model the application execution time, we consider a program as a set of instructions executed on a mean frequency $f$ with $c_k$ instructions per cycle. The time $T_f$ that this program will take to complete at a given frequency is devised as follows:
\begin{equation}
T_f=\frac{I}{c_kf},
\label{eq:freqrel}
\end{equation}
where $I$ is the total number of instructions and $c_k$ the ratio of instructions per unit of time.

The next step is to include the number of cores in the equation. Amdahl's law \cite{Amdahl1967ValidityCapabilities}, described in \cref{eq:amdahl}, gives the theoretical background for that. It describes the speedup in latency of the execution of a task at a fixed workload.
\begin{equation}
S=\frac{T_s}{T_p}=\frac{1}{1-w+\frac{w}{p}},
\label{eq:amdahl}
\end{equation}
where $T_s$ is the serial time, $T_p$ the parallel time, $S$ is the theoretical speedup of the execution of the whole task, $w$ is the proportion of the execution time that benefits from improving system resources, and $p$ is the part of the task that benefits from improved system resources. Combining this with \cref{eq:freqrel}, the parallel time at frequency $f$ can be written as:
\begin{equation}
T_p=\frac{T_s}{S}=\frac{T_f}{\frac{1}{1-w+\frac{w}{p}}},
\label{eq:parallel_time}
\end{equation}

We can then write the equation of the program execution time as a function of frequency, number of cores and parallelism  as \cref{eq:performance} and subsequently derive \cref{eq:performance_2}:
\begin{equation}
T(f,p)=\frac{I}{ \frac{c_kf}{1-w+\frac{w}{p}} },
\label{eq:performance}
\end{equation}
\begin{equation}
T(f,p)=\frac{d_1(p-wp+w)}{fp},
\label{eq:performance_2}
\end{equation}
where $d_1$ is a constant.

Finally, to fully characterize the application, a parameter representing the application's workload, called input size $N$, is introduced, representing the number of basic operations need to complete a problem \cite{Kumar1994AnalyzingArchitectures}. In Oliveira et al. \cite{Oliveira2018ApplicationCores}, they showed that this parameter could generally be described as exponential. Therefore the proposed performance model is presented in \cref{eq:performance_final}. This resulting equation can describe the behavior of the execution time of a program for an input $N$, frequency $f$, and active cores $p$:
\begin{equation}
T(f,p,N)=\frac{d_1N^{d_2}(p-wp+w)}{fp},
\label{eq:performance_final}
\end{equation}
where $d_1$, $d_2$ and $w$ are constants that depend on the application. 

\subsection{Energy Model} \label{sec:energymodel}
Combining the power model output described in~\cref{sec:powermodel} and the characterization of the application performance described in \cref{sec:performancemodel}, the total energy can be modeled as:
\begin{equation}
E(f,p,N)=P(f,p)\times{\rm T}(f,p,N),
\label{eq:en_combination}
\end{equation}
where $P(f,p)$ is the total power modeled by~\cref{eq:power_final}, ${T}(f,p,N)$ is the execution time estimated by the \cref{eq:performance_final}, $f$ is the frequency, $p$ is the number of active cores, and $N$ is the input size. The final equation can be written as:
\begin{equation}
E(f,p,N)=\frac{d_1N^{d_2}(p-wp+w)(p(c_1f^3+c_2f)+c_3)}{fp}.
\label{eq:en_final}
\end{equation}

%%%%%%%%%%%%%%%%%%%%%%%%%%%%%%%%%%%%%%%%%%
	%\section{Common approaches to the optimization problems}

To build the optimization problem it's necessary to model the system power and application, and controllable variables. Common approaches to the optimization problems are:

Minimize the energy function in function of our control variables, subject to constraints in the control variable.

\begin{equation}
\begin{aligned}
\textrm{min} \quad & P(s_1, s_2, ...)T(s_1,s_2,...)\\
\textrm{subject to} \quad & b_1<s_1<b_2\\
\quad & b_3<s_2<b_4\\
\quad & \vdots\\
\end{aligned}
\end{equation}

Another way is to minimize the total energy, with the constraint of to finish the work.

\begin{equation}
\begin{aligned}
\textrm{min} \quad & \sum{P_it_i}\\
\textrm{subject to} \quad & W_{tot} = \sum w_i\\
\quad & \vdots\\
\end{aligned}
\end{equation}

This kind of problem can be seen in multiple ways, considering an application as the total workload and choosing different speeds for each phase of the application, or could also treat each workload as a different application and create schedulers both CPU and cluster level. In fact the question of on each level of optimizer produces a better result is not well explored in the literature yet. The habitual approach it's to tackle each problem at once and combine the strategies resulting in a chain of schedulers.

The complexity of this problem also varies, depending on the choice of power function it can result in linear programming \cite{Kim2015RacingHeuristics}, quadratic programming \cite{Horyath2008Multi-mode}, until NP-hard problems \cite{Fu2018RaceMinimization}.
	\section{Model validation} \label{sec:model_validation}
In this section, the models presented in~Sections \ref{sec:power_model} and~\ref{sec:performance_model} were validated with a benchmark specific for multi-core architectures. Additionally, in order to assess the modeling overhead and accuracy, our proposal was then compared to machine learning approaches. We compared against support vector regression (SVR)~\cite{Smola2004ARegression}, decision tree \cite{Kitts2006RegressionLecture}, k-nearest neighbors \cite{Altman1992AnRegression}, multilayer perceptron \cite{Murtagh1991MultilayerRegression}, and some new methods, such as Gao et al. \cite{Gao2019DendriticPrediction}. However, SVR was chosen as the most representative because it performed best in our tests without aggressive fine-tuning, as shown in \cref{fig:ml_models}.

\begin{figure}[H]
	\centering
	\includegraphics[width=\columnwidth]{experiments/figures/ml_models.pdf}
	\caption{Average of the mean squared error for all applications of our study case Section \ref{sec:casestudyapplication}.}
	\label{fig:ml_models}
\end{figure}

\subsection{Verifying Hypothesis} \label{subsec:ev_verifying_hypothesis}

In this section, we validate whether the assumptions of our model are valid for the system used.

\subsection{Frequency and Voltage Relation} \label{subsec:ev_frequency_and_voltage Relation}
One of the assumptions was that the frequency and the voltage have a linear relationship, as indicated by ~\cref{eq:f_v}. To verify that, we build an experiment that sets the frequency to a specific value while sampling the voltage using the APERF and MPERF registers that provide feedback on the current CPU frequency. The average result of the sampling voltages is shown in  \cref{fig:freq_volt_rel}, where we can observe a near-perfect linear relation. This is because manufacturers implement this curve in the processors, using tables that relate ranges of frequencies to voltages so that they can precisely define any curve that will better suit their design.

\begin{figure}[H]
	\centering
	\captionsetup[subfigure]{justification=centering}
	\includegraphics[width=\columnwidth]{experiments/figures/freq_volt_rel.pdf}
	\caption{Frequency voltage relation.}
	\label{fig:freq_volt_rel}
\end{figure}

\subsection{Input Size and Instructions} \label{subsec:ev_input_size_and_instructions}
We ran the applications with different inputs assuming  linear growth in the amount of work for one input to the other when building our model. However, measuring and controlling the amount of work would require much instrumentation and tuning to find an input corresponding to a certain amount of work. Therefore, to build our models, we use the time to reference the amount of work, assuming that the work is proportional to the executing time. \cref{fig:time_input} corresponds to the verification of this supposition.


\cref{tab:corr_in_time} shows that the assumption was reasonable since the average correlation was 0.96 for all applications, indicating that growth in the number of instructions  will follow the time. This was the case for all applications that we ran in our benchmark and should hold for any data parallelism type of application.
\begin{figure}[H]
	\centering
	\captionsetup[subfigure]{justification=centering}
	\begin{subfigure}[b]{0.45\textwidth}
		\includegraphics[width=\columnwidth]{models/figures/hypothesis/input_instructions/input_time/blackscholes.pdf}
		\caption{Blackscholes.}
		\label{fig:time_input_blackshoels}
	\end{subfigure}
	%
	\begin{subfigure}[b]{0.45\textwidth}
		\includegraphics[width=\columnwidth]{models/figures/hypothesis/input_instructions/input_time/canneal.pdf}
		\caption{Canneal.}
		\label{fig:time_input_canneal}
	\end{subfigure}
	
	\hfill
	\caption{Relation between time and instructions for each input size. }
	\label{fig:time_input}
\end{figure}
\vspace{-6pt}

\begin{table}[H]
	\centering
	\caption{Correlation of time and instructions for all applications.}
	\begin{tabular}{ll}
		\multicolumn{1}{l|}{\textbf{Application}} & \textbf{Correlation} \\ \hline
		Blackscholes                                               & 0.99                                  \\
		Bodytrack                                                  & 0.99                                  \\
		Canneal                                                    & 0.99                                  \\
		Dedup                                                      & 0.99                                  \\
		Ferret                                                     & 0.96                                  \\
		Fluidanimate                                               & 0.99                                  \\
		Freqmineq                                                  & 0.99                                  \\
		Openmc                                                     & 0.94                                  \\
		Raytrace                                                   & 0.99                                  \\
		Swaptions                                                  & 0.99                                  \\
		Vips                                                       & 0.98                                  \\
		x264                                                       & 0.99                                  \\
		HPL                                                        & 0.79                                  \\ \hline
	\end{tabular}
	\label{tab:corr_in_time}
\end{table}

The next assumption was that the application's behavior was the same when varying the workload. This condition is necessary for using the model with an unknown input size because, if the behavior is the same, we can interpolate the known inputs. One way to verify this is to measure the rate of instructions per second normalized by the frequency, as shown in \cref{fig:fingerpritns}.
\begin{figure}[H]
	\centering
	\captionsetup[subfigure]{justification=centering}
	
	\begin{subfigure}[b]{0.45\textwidth}
		\includegraphics[width=\columnwidth]{models/figures/hypothesis/input_instructions/fp/blackscholes.pdf}
		\caption{Blacksholes.}
		\label{fig:fp_baclscholes}
	\end{subfigure}
	%
	\begin{subfigure}[b]{0.45\textwidth}
		\includegraphics[width=\columnwidth]{models/figures/hypothesis/input_instructions/fp/canneal.pdf}
		\caption{Canneal.}
		\label{fig:fp_canneal}
	\end{subfigure}
	
	\hfill
	\caption{Rate of instructions per second varying the input size normalized by the frequency.}
	\label{fig:fingerpritns}
\end{figure}

\cref{fig:fingerpritns} shows that the applications have roughly the same curve when normalized; this also happens for all other applications in our benchmark.

The final assumption is that the workload should also not vary depending on the number of cores or frequency. To verify, we measure the total number of executed instructions while varying the cores from 1 to 32. \cref{tab:cores_variation} shows the results.

\begin{table}[H]
	\centering
	\caption{Variation of the number of instructions when changing the number of cores for the same input.}
	%\begin{tabular}{c|c|c|c}
	\begin{tabular*}{\hsize}{@{\extracolsep{\fill}}cccc}
		\toprule
		& \textbf{Average} & \textbf{Standard}     & \textbf{Standard}\\
		\multirow{-2}{*}{\textbf{Application}}& \textbf{Number of Instructions} & \textbf{Deviation}     & \textbf{Deviation} \textbf{(\%) }\\ \midrule
		Vip          & $7.97 \times 10^{11}$     & $7.16 \times 10^{6}$  & 0.00 \\ %Please use Scientific notation. For example, “1.65e+04” should be changed to “1.65× 104”. Please confirm `%` whether to remove.
		Openmc       & $8.17 \times 10^{7}$     & $ 1.65 \times 10^{4}$ & 0.02    \\ 
		Rtview       & $9.91 \times 10^{12}$     & $ 1.55 \times 10^{9}$ & 0.02    \\ 
		X264         & $4.52 \times 10^{11}$     & $ 5.81 \times 10^{7}$ & 0.01    \\ 
		Bodytrack    & $1.86 \times 10^{12}$     & $ 3.95 \times 10^{10}$ & 2.13    \\ 
		Fluidanimate & $2.09 \times 10^{12}$     & $ 8.44 \times 10^{10}$ & 4.04    \\ 
		HPL          & $1.14 \times 10^{8}$      & $ 1.24 \times 10^{5}$ & 0.11    \\ 
		Blackschole  & $3.75 \times 10^{12}$     & $ 1.40 \times 10^{9}$ & 0.04     \\ 
		Dedup        & $1.02 \times 10^{11}$     & $ 5.74 \times 10^{7}$ & 0.06     \\ 
		Swapti       & $2.43 \times 10^{12}$     & $ 8.87 \times 10^{8}$ & 0.04     \\ 
		Canneal      & $1.19 \times 10^{11}$     & $ 4.46 \times 10^{7}$ & 0.04     \\ 
		Freqmine     & $1.27 \times 10^{12}$     & $ 4.78 \times 10^{8}$ & 0.04     \\ 
		Ferret       & $4.76 \times 10^{11}$     & $ 7.04 \times 10^{7}$ & 0.01     \\ \bottomrule
	\end{tabular*}
	\label{tab:cores_variation}
\end{table}
% Corrected

\cref{tab:cores_variation} shows the standard deviation and what that corresponds to in terms of the total number of instructions as a percentage. 

The same test was performed for the frequency, varying from 1.2 to 2.2 GHz with 100~MHz steps. The results are shown in \cref{tab:freq_variation}.

These results show that all the assumptions were reasonable, and we can safely move to the validation of the model's prediction.
\begin{table}[H]
	\caption{Variation of the number of instructions when changing the frequency for the same input.}
	\centering
	%\begin{tabular}{c|c|c|c}{\hsize}{@{\extracolsep{\fill}}cccc}
	\begin{tabular*}{\hsize}{@{\extracolsep{\fill}}cccc}
		\toprule
		& \textbf{Average} & \textbf{Standard}     & \textbf{Standard}\\
		\multirow{-2}{*}{\textbf{Application}}& \textbf{Number of Instructions} & \textbf{Deviation}     & \textbf{Deviation} \textbf{(\%)} \\ \midrule
		Vip          & $7.97 \times 10^{11}$      & $ 1.16 \times 10^{6}$  & 0.00      \\ 
		Openmc       & $8.17 \times 10^{7}$       & $ 4.52 \times 10^{3}$  & 0.01      \\ 
		Rtview       & $9.91 \times 10^{12}$      & $ 6.64 \times 10^{5}$  & 0.00      \\ 
		X264         & $4.52 \times 10^{11}$      & $ 1.54 \times 10^{5}$  & 0.00      \\ 
		Bodytrack    & $1.84 \times 10^{12}$      & $ 2.54 \times 10^{5}$  & 0.00      \\ 
		Fluidanimate & $2.38 \times 10^{12}$      & $ 1.70 \times 10^{9}$  & 0.07      \\ 
		HPL          & $1.14 \times 10^{8}$       & $ 5.95 \times 10^{3}$  & 0.01      \\ 
		Blackschole  & $3.75 \times 10^{12}$      & $ 4.36 \times 10^{5}$  & 0.00      \\ 
		Dedup        & $1.02 \times 10^{11}$      & $ 8.32 \times 10^{7}$  & 0.08      \\ 
		Swapti       & $2.43 \times 10^{12}$      & $ 1.48 \times 10^{5}$  & 0.00      \\ 
		Canneal      & $1.19 \times 10^{11}$      & $ 3.01 \times 10^{5}$  & 0.00      \\ 
		Freqmine     & $1.27 \times 10^{12}$      & $ 3.70 \times 10^{8}$  & 0.03      \\ 
		Ferret       & $4.76 \times 10^{11}$      & $ 5.63 \times 10^{7}$  & 0.01      \\\bottomrule
	\end{tabular*}
	\label{tab:freq_variation}
\end{table}


\section{Fitting the models} \label{subsec:fitting_the_models}
To find the parameters of  \cref{eq:en_final}, 10 uniformly random configurations of frequencies ($f$), cores ($p$) and inputs ($N$) were chosen from the range $1<=p<=32$, $1.2<=f<=2.2$ and $1<=N<=5$, respectively. The application was executed for each chosen configuration, and the measured energy and time values were collected. For the input size, if we assume that all CPU instructions take approximately the same time to execute, the number of basic operations will be directly correlated with the time. Thus, we can estimate the input size by looking at the execution time, allowing us to divide a large input size into several smaller ones, knowing their relationship, as performed in the work of Oliveira ~\cite{Oliveira2018ApplicationCharacterization}. The unity can also vary depending on the definition. For simplicity, we assign numbers from 1 to 10, increasing the problem linearly, so it is also possible to interpolate any input in between these values.

For each configuration, samples of the power were collected using IPMI every 1 second. This sampling rate was chosen based on the magnitude of the mean run time of the applications, which is in the order of minutes. Therefore, this rate provides enough samples to measure average power. Additionally, timestamps and the total run time were collected. The total energy spent on each configuration is estimated by first interpolating the power samples using the first-order method and then integrating this function in the time.

The model's parameters are calculated by solving an optimization problem of finding the values that minimize the squared error of the prediction to the measured values using the non-linear least-squares method.

The Python library Scikit-Learn was used to build the SVR model~\cite{Pedregosa2011Scikit-learn:Python}. The SVR was trained using the same data used for parameter estimation of Equation (\ref{eq:en_final}) %\cref{eq:en_final}
with a grid search used to find the best kernel function and the best values for the hyper-parameters penalty for the wrong ($C$) and ($\gamma$). For this data, the best function was the radial base function (RBF), and the hyper-parameters were $C=10^4$ and $\gamma=0.5$.


\section{Measured versus modeled energy} \label{sec:measured_versus_modeled_energy}

To validate the model, we ran all possible configurations in the tested machine, varying the cores in a range of $1<=p<=32$, the frequency in $1.2<=f<=2.2$, and the input in $1<=N<=5$. The total number of configurations varies from 400 to over 1000 depending on the application, as some applications have restrictions on the number of cores that they can run. Once the data was collected, we computed the mean percentage error (MPE) according to the following equation:
\begin{equation}
	MPE = \frac{1}{N} \sum_i^N \frac{|y_{\rm estimated}-y_{\rm measured}|}{y_{\rm measured}}.
	\label{eq:mpe}
\end{equation}

\subsection{Frequency $\times$ Cores} \label{subsec:mvme_frequency_x_cores}
\cref{fig:en_eq_freq_cores_fc} plots the measured and modeled energy consumption for some of the applications modeled. In addition,  some of the possible shapes that the model can take while varying the number of active cores, and operating frequency, are shown.%please confirm that the intended meaning has been retained.
% yes its the same

\begin{figure}[H]
	\centering
	\captionsetup[subfigure]{justification=centering}
	\begin{subfigure}[b]{0.45\textwidth}
		\centerline{\includegraphics[width=\columnwidth]{models/figures/energy/freq_cores/completo_black_5.pdf}}
		\caption{}
		\label{fig:en_eq_black_fc}
	\end{subfigure}
	%
	\begin{subfigure}[b]{0.45\textwidth}
		\centerline{\includegraphics[width=\columnwidth]{models/figures/energy/freq_cores/completo_canneal_1.pdf}}
		\caption{}
		\label{fig:en_eq_canneal_fc}
	\end{subfigure}
	
	\caption{Example fit for a specific input size: Blackscholes (\textbf{a}) and Canneal (\textbf{b}).  “measured values” are the sensor data, and “minimum energy” is the minimum energy model prediction.
	}
	\label{fig:en_eq_freq_cores_fc}
\end{figure}

\subsection{Frequency $\times$ Input} \label{subsec:mvme_frequency_x_input}
\cref{fig:en_eq_freq_inp_fi} plots the measured and modeled energy consumption for some of the applications modeled. The diagrams show some of the possible shapes that the model can take while varying the operating frequency, and input size.
\begin{figure}[H]
	\centering
	\captionsetup[subfigure]{justification=centering}
	\begin{subfigure}[b]{0.45\textwidth}
		\centerline{\includegraphics[width=\columnwidth]{models/figures/energy/freq_inps/completo_black_5.pdf}}
		\caption{}
		\label{fig:en_eq_black_fi}
	\end{subfigure}
	%
	\begin{subfigure}[b]{0.45\textwidth}
		\centerline{\includegraphics[width=\columnwidth]{models/figures/energy/freq_inps/completo_canneal_1.pdf}}
		\caption{}
		\label{fig:en_eq_canneal_fi}
	\end{subfigure}
	
	\caption{Example fit for a specific input size: Blackscholes (\textbf{a}) and Canneal (\textbf{b}).  “measured values” are the sensor data and “minimum energy” is the minimum energy model prediction.
	}
	\label{fig:en_eq_freq_inp_fi}
\end{figure}

\subsection{Cores $\times$ Input} \label{subsec:mvme_cores_x_input}
\cref{fig:en_eq_core_inp_ci} plots the measured and modeled energy consumption for some of the applications modeled. The diagrams  show some of the possible shapes that the model can take while varying the number of active cores, and input size.
\begin{figure}[H]
	\centering
	\captionsetup[subfigure]{justification=centering}
	\begin{subfigure}[b]{0.45\textwidth}
		\centerline{\includegraphics[width=\columnwidth]{models/figures/energy/cores_inps/completo_black_5.pdf}}
		\caption{}
		\label{fig:en_eq_black_ci}
	\end{subfigure}
	%
	\begin{subfigure}[b]{0.45\textwidth}
		\centerline{\includegraphics[width=\columnwidth]{models/figures/energy/cores_inps/completo_canneal_1.pdf}}
		\caption{}
		\label{fig:en_eq_canneal_ci}
	\end{subfigure}
	
	\caption{Example fit for a specific input size: Blackscholes (\textbf{a}) and Canneal (\textbf{b}).  “measured values” are the sensor data and “minimum energy” is the minimum energy model prediction.
	}
	\label{fig:en_eq_core_inp_ci}
\end{figure}


\section{Comparison} \label{sec:comparison}
The average results for each application were calculated using a model trained with only 10 configurations, and the comparison is displayed \cref{fig:mpe_svr_eq}. %We calculated the raw MPE values shown in ~\cref{tab:mpe_svr_eq}.
\begin{figure}[H]
	\centering
	\includegraphics[width=.8\columnwidth]{models/figures/mpe_svr_eq.pdf}
	\caption{Comparison of the mean percentage error between the proposed model and SVR. ``Model mean'' and ``SVR mean'' are the average of all MPE values for all applications.
	}
	\label{fig:mpe_svr_eq}
\end{figure}

\cref{fig:mpe_svr_eq} shows that the proposed model always performed better, with a lower MPE than SVR, when we were limited to 10 training points. This result is further explored in the next Section \ref{sec:overhead_on_training}, where we undertake a comparison with different training sizes. The exact values are shown in \cref{tab:mpe_svr_eq}.

\begin{table}[H]
\centering
\begin{tabular}{c|c|c}
\hline
Application  & Model & SVR   \\ \hline
Ferret       & 5.25     & 12.49  \\ 
Raytrace     & 6.36     & 11.95  \\
Fluianimate  & 2.44     & 22.90  \\
x264         & 8.28     & 15.33  \\
Vips         & 7.54     & 10.80  \\
Swaptions    & 6.54     & 18.57  \\
Canneal      & 3.12     & 6.13   \\
Dedup        & 8.85     & 13.70  \\
Freqmine     & 2.44     & 3.24   \\
Blackscholes & 2.18     & 11.00  \\
HPL          & 7.47     & 12.75  \\
Bodytrack    & 16.98    & 34.12  \\
Openmc       & 11.15    & 24.34  \\
\end{tabular}
\caption{Comparison of the Mean Percentage Error between the proposed model and SVR: raw values.}
\label{tab:mpe_svr_eq}
\end{table}

\section{Overheads on training} \label{subsec:overhead_on_training}
It is known that machine learning is data-driven; in that sense, the SVR model obtained using only 10 configurations could be improved, but what about the analytical model? 
To answer that question, the proposed model and the SVR were also trained with a varying number of configurations.
We then compared the MPE and the amount of energy spent to create each model. 
This accuracy-energy trade-off is crucial since building models' energy overhead defeats the primary goal of saving power when running applications.
\enlargethispage{.5cm}

\begin{figure}[H]
	\centering
	\captionsetup[subfigure]{justification=centering}
	\begin{subfigure}[b]{0.45\textwidth}
		\centerline{\includegraphics[width=\columnwidth]{models/figures/overhead/completo_ferret_4.pdf}}
		\caption{MPE for Ferret.}
		\label{fig:overhead_ferret}
	\end{subfigure}
	%
	\begin{subfigure}[b]{0.45\textwidth}
		\centerline{\includegraphics[width=\columnwidth]{models/figures/overhead/completo_black_6.pdf}}
		\caption{MPE for Blackscholes.}
		\label{fig:overhead_black}
	\end{subfigure}
	%
	\par\bigskip
	\begin{subfigure}[b]{0.45\textwidth}
		\centerline{\includegraphics[width=\columnwidth]{models/figures/overhead/completo_x264_4.pdf}}
		\caption{MPE for x264.}
		\label{fig:overhead_x264}
	\end{subfigure}
	%
	\begin{subfigure}[b]{0.45\textwidth}
		\centerline{\includegraphics[width=\columnwidth]{models/figures/overhead/completo_dedup_4.pdf}}
		\caption{MPE for Dedup.}
		\label{fig:overhead_dedup}
	\end{subfigure}
	
	\caption{MPE of the case studies versus training size, comparing how many training points is necessary to reach an acceptable result.}
	\label{fig:overheadapps}
\end{figure}

\cref{fig:overheadapps} shows the comparisons of MPE and energy spent to create each model for two selected applications. According to the results, the analytical model is very stable, not changing much as more data is added, while the SVR keeps reshaping to adapt to the data. 
The error of the analytical model is almost constant but that of the SVR, initially very high, drops as more data is used in the training process.
\begin{figure}[H]
	\centering
	\captionsetup[subfigure]{justification=centering}
	\begin{subfigure}[t]{0.45\textwidth}
		\includegraphics[width=\columnwidth]{models/figures/overhead/overall_energy_10pts.pdf}
		\caption{}
		\label{fig:overall_overhead}
	\end{subfigure}
	%
	%\par\bigskip
	\begin{subfigure}[t]{0.45\textwidth}
		\includegraphics[width=\columnwidth]{models/figures/overhead/overall_mpe_10pts.pdf}
		\caption{
		}
		\label{fig:overall_MPE}
	\end{subfigure}
	\caption{Overall results for energy and MPE for each training size. (\textbf{a}) Average energy spent on all applications during model creation. The two curves are identical because the same data were used to adjust the SVR and the model. (\textbf{b}) MPE of all applications: SVR needs 10 times more data to have an MPE lower than the proposed model.}
	\label{fig:overall_train}
\end{figure}

\cref{fig:overall_train} presents the overall results, with the mean energy overhead and MPE for all applications.
The meeting point of the MPE for the SVR and the proposed model can be extracted from \cref{fig:overall_train}b.
It shows that, in around 90 configurations, the SVR starts to have a smaller error. The cost of that is the linear increase in energy spent on training. The increase in energy, about 10 times more, can be observed in \cref{fig:overall_train}a.

\section{Analysis} \label{sec:analysis}
One of the most significant advantages of using an analytical model is the understanding of the problem that an equation provides, making many different kinds of analysis possible that are otherwise impossible with a machine learning model. In this section, we discuss one of the possible analyses. In the following figures, we try to understand the contribution of each parameter of the equation to the total energy consumption.

For this analysis, we took the model of one of the applications and, varying one parameter of the equation, we display the energy versus performance (time) for all configurations. After that, we computed the Pareto frontier, a set of all Pareto efficient allocations, i.e., all the configurations where resources cannot be reallocated to make one individual better off without making at least one individual worse off. This gives us all the configurations where we have an optimal trade-off of performance and energy to choose from.


\cref{fig:pareto_static} shows the Pareto frontier for several values for the static power parameter ($c_3$ in \cref{eq:en_final}) with configurations of frequency ranging from 1.2 to 5 GHz and cores from 1 to 64, so that we can also have an idea of what is the tendency when we increase the frequency and number of cores.

\begin{figure}[H]
	\includegraphics[width=\columnwidth]{models/figures/analisys/pareto_static_high.pdf}
	\caption{Pareto frontier for several values of static power parameter. The arrows with blue heads indicate the maximum energy, while the arrows with a red head the minimal energy for each corresponding curve.}% In four-digit numbers, no comma or spaces to indicate thousand, whatever the numbers are in the table or body text, e.g., 4500; More than five-digit numbers, should have comma, e.g. 59,400; 899,302. Please revise the figure.
	\label{fig:pareto_static}
\end{figure}
%Fixed

From this figure, we can see that when increasing the value of the static power parameter, the total energy consumption increases as expected. We can also observe that the values that minimize the total energy consumption tend to be high frequency and multiple cores. This is one of the consequences of increasing the static power factor. As the dynamic factor proportionally decreases, its variables tend to have less impact on total consumption, enabling configurations with high frequency and several cores. This also enables chip-level optimization for choosing components that change the ratio between static and dynamic power.


\cref{fig:pareto_w} shows the Pareto frontier in the same ranges described before but for the parameter corresponding to the level of parallelism of the application ($w$ in  \cref{eq:en_final}).

\begin{figure}[H]
	\includegraphics[width=\columnwidth]{models/figures/analisys/pareto_w_high.pdf}
	\caption{Pareto frontier for several values of static power parameter. The arrows with blue heads indicate the maximum energy, while the arrows with a red head, the minimal, for each corresponding curve.}% In four-digit numbers, no comma or spaces to indicate thousand, whatever the numbers are in the table or body text, e.g., 4500; More than five-digit numbers, should have comma, e.g. 59,400; 899,302. Please revise the figure.
	\label{fig:pareto_w}
\end{figure}
%Fixed

In  \cref{fig:pareto_w}, we observe that, as the parallelism level increases the total energy decreases. The number of cores tends to be higher with a higher level of parallelism as expected, and the frequency shows an inverse relation.


\section{DVFS and DPM optimization} \label{sec:dvfs_dpm_optminzation}
The effectiveness of the proposed approach during optimization was evaluated with a simple algorithm that finds the optimal frequency and number of active cores from the proposed equation. The results were then compared to the Linux default choices for power management.

With \cref{eq:en_final}, it is possible to calculate energy consumption estimates for each possible configuration since there is a finite range of possible values for the frequency and number of cores. It is also possible to apply constraints on the execution time, frequency, and the number of active cores. Then, the configuration that minimizes energy consumption for a given input can be selected. The complete workflow is shown in \cref{fig:optim_workflow}. We can see that any optimization problem can be structured with our model and the system's constraints. In the following examples, the optimization problem that we build is to minimize the energy equation given the constraints of possible frequencies and the number of cores that our system can run. The algorithm selected to minimize was the newton-CG~\cite{Royer2020AOptimization}.

Current HPC managers leave to the user the choice of how many cores to use. On this basis, three situations were analyzed in relation to the number of cores:

\begin{enumerate}
	\item Worst choice: number of cores that maximize the total energy consumed;
	\item Random choice: energy consumed for a random choice of the number of cores;
	\item Best choice: number of cores that minimize the total energy consumed (oracle).
\end{enumerate}
\begin{figure}[H]
	\centering
	\includegraphics[width=0.8\columnwidth]{models/figures/DVFS optim.pdf}
	\caption{Optimization workflow showing how DVFS and DPM optimization could be implemented from ou model.}
	\label{fig:optim_workflow}
\end{figure}


The default option for the Linux governor is Ondemand, and, by default, it has no DPM control for the number of active cores. As Ondemand only performs DVFS, \textls[-15]{for comparison, each application was executed with all available cores in the system, from 1 to 32.}

\textls[-15]{Figures \ref{fig:energy_worst_case}--\ref{fig:energy_best_case}, show the energy savings with respect to Ondemand, i.e., $\frac{Ondemand-Model_{min}}{Ondemand}$} for the three cases described above. The savings and losses for each case are:

\begin{enumerate}
	\item Worst choice: save 69.88\% on average;
	\item Random choice: save 12.04\% on average;
	\item Best choice: lost 14.06\% on average.
\end{enumerate}

\begin{figure}[H]
	\centering
	\includegraphics[width=\columnwidth]{models/figures/dvfs_cmp_max.pdf}
	\caption{Energy savings comparisons between the proposed model and the Worst case.}
	\label{fig:energy_worst_case}
\end{figure}

\begin{figure}[H]
	\centering
	\includegraphics[width=\columnwidth]{models/figures/dvfs_cmp_mean.pdf}
	\caption{Energy savings comparisons between the proposed model and the Random case.}
	\label{fig:energy_mean_case}
\end{figure}

By default, operating systems do not implement DPM at the core level, and, in HPC, the user usually explicitly chooses the number of cores to run their job. To give a better idea of the impact on the energy consumption of DPM at the core level, we analyzed the choices of the number of cores over a period of one year in the HPC center at UFRN. The result is plotted in~\cref{fig:cpu_requests}.


It is of note that the most common choice of many regular users is a single core requested per job, matching the worst-case choice for all applications analyzed in this investigation. The best choice was quite often 32 cores, which is the third most popular choice among users, but it is 72 times less frequent than 1 core. This led us to envision how much energy could be saved and encouraged us towards future research using the proposed model for DPM or more advanced optimization algorithms.

In practice, this approach can be implemented by allowing the resource manager to perform these changes for the user using pre-scripts and post-scripts for high energy consumption job submissions.
\begin{figure}[H]
	\centering
	\includegraphics[width=\columnwidth]{models/figures/dvfs_cmp_32.pdf}
	\caption{Energy savings comparisons between the proposed model and the Best case.}
	\label{fig:energy_best_case}
\end{figure}
\vspace{-12pt}

\begin{figure}[H]
	\centering
	\includegraphics[width=\columnwidth]{experiments/figures/cpu_requestes.pdf}
	\caption{Number of CPU requests during one year in HPC cluster, sorted by the number of cores requested per job.}
	\label{fig:cpu_requests}
\end{figure}

%\section{PMCs}
%
%\section{Energy per instruction}
%
%\begin{lstlisting}
%xor rcx, rcx
%mov rax, 1
%mov rdx, 0
%
%loop:
%	targ_inst(*arg)
%	targ_inst(*arg)
%	targ_inst(*arg)
%	targ_inst(*arg)
%	targ_inst(*arg)
%	targ_inst(*arg)
%	targ_inst(*arg)
%	targ_inst(*arg)
%	targ_inst(*arg)
%
%add rcx, 1
%cmp rcx, 9999999
%jne loop
%\end{lstlisting}
%
%Estimating the energy of the instruction from this benchmark
%
%$E_{total}=9999999(\frac{10}{13}inst+\frac{3}{13}loop)$
%
%$E_{total}=7692306*inst+constant$
%
%\subsection{Generic}
%
%\begin{figure}
%	\centering
%	\includegraphics[width=\textwidth]{experiments/figures/inst_en_args_generic.png}
%	\caption{Energy per instruction argument}
%	\label{fig:experiment_en1}
%\end{figure}
%
%\begin{figure}
%	\centering
%	\includegraphics[width=\textwidth]{experiments/figures/inst_mean_en_generic.png}
%	\caption{Mean energy per instruction over all arguments}
%	\label{fig:experiment_en2}
%\end{figure}
%
%\begin{figure}
%	\centering
%	\includegraphics[width=\textwidth]{experiments/figures/inst_std_en_generic.png}
%	\caption{Standard deviation energy per instruction over all arguments}
%	\label{fig:experiment_en3}
%\end{figure}
%
%\subsection{SSE}
%
%\begin{figure}
%	\centering
%	\includegraphics[width=\textwidth]{experiments/figures/inst_en_args_sse.png}
%	\caption{Energy per instruction argument (sse)}
%	\label{fig:experiment_en4}
%\end{figure}
%
%\begin{figure}
%	\centering
%	\includegraphics[width=\textwidth]{experiments/figures/inst_mean_en_sse.png}
%	\caption{Mean energy per instruction over all arguments (sse)}
%	\label{fig:experiment_en5}
%\end{figure}
%
%\begin{figure}
%	\centering
%	\includegraphics[width=\textwidth]{experiments/figures/inst_std_en_sse.png}
%	\caption{Standard deviation energy per instruction over all arguments (sse)}
%	\label{fig:experiment_en6}
%\end{figure}
%
%\subsection{SSE and Generic}
%
%\begin{figure}
%	\centering
%	\includegraphics[width=\textwidth]{experiments/figures/inst_en_cmp_sse_generic.png}
%	\caption{Comparing SSE and generic instructions energy consumption}
%	\label{fig:experiment_en7}
%\end{figure}
%
%\subsection{Power}
%
%\begin{figure}
%	\centering
%	\includegraphics[width=\textwidth]{experiments/figures/inst_pw_generic.png}
%	\caption{Instructions power draw}
%	\label{fig:experiment_pw1}
%\end{figure}
%\begin{figure}
%	\centering
%	\includegraphics[width=\textwidth]{experiments/figures/inst_pw_args.png}
%	\caption{Instructions power draw by argument}
%	\label{fig:experiment_pw2}
%\end{figure}
%
%\section{Tracing application methods}
%
%\begin{outline}
%	\1 Linux ptrace
%		\2 Breakpoint on each instruction	
%	\1 Intel pin
%		\2 JIT
%		\2 Not open source
%	\1 qemu user
%		\2 TCG accelerator (transform target to host instruction) similar to JIT
%		\2 Partial emulation
%\end{outline}
%
%We choose the less intrusive and open-source alternative qemu user, with an small modificaion we are able to trace an application instructions. Generating an table as follows:
%
%
%\begin{table}[H]
%	\centering
%	\begin{tabular}{|c|c|c|}
%		\hline
%		opcode         & mnemonic & args                          \\ \hline
%		48 89 e7       & movq     & \%rsp, \%rdi\textbackslash{}n \\ \hline
%		48 89 e7       & movq     & \%rsp, \%rdi\textbackslash{}n \\ \hline
%		e8 08 0e 00 00 & callq    & 0x4000a24ea0\textbackslash{}n \\ \hline
%		55             & pushq    & \%rbp\textbackslash{}n        \\ \hline
%		48 89 e5       & movq     & \%rsp, \%rbp\textbackslash{}n \\ \hline
%		41 57          & pushq    & \%r15\textbackslash{}n        \\ \hline
%		41 56          & pushq    & \%r14\textbackslash{}n        \\ \hline
%		41 55          & pushq    & \%r13\textbackslash{}n        \\ \hline
%		& \vdots & \\ \hline
%	\end{tabular}
%\end{table}
%
%
%\section{CPU load}
%
%\begin{figure}[H]
%	\centering
%	\begin{subfigure}[b]{0.45\textwidth}
%		\centering
%		\includegraphics[width=1\columnwidth]{experiments/figures/pw_freq_load.png}
%		\caption{Instructions power draw varing CPU load}
%		\label{fig:experiment_pw_load}
%	\end{subfigure}
%	%
%	\begin{subfigure}[b]{0.45\textwidth}
%		\centering
%		\includegraphics[width=1\columnwidth]{experiments/figures/pw_load.png}
%		\caption{Instructions power draw varing CPU load}
%		\label{fig:experiment_pw_load}
%	\end{subfigure}
%\end{figure}
%
%
%\section{Analysis}
%
%\subsection{Pareto points}

%
%\begin{figure}

%	\centering

%	\includegraphics[width=\columnwidth]{models/figures/analisys/pareto_static_high.png}

%	\caption{Pareto static energy}

%	\label{fig:pareto_static_h}

%\end{figure}

%
%
%\begin{figure}

%	\centering

%	\includegraphics[width=\columnwidth]{models/figures/analisys/pareto_static_low.png}

%	\caption{Pareto static energy}

%	\label{fig:pareto_static_l}

%\end{figure}

%
%
%\begin{figure}

%	\centering

%	\includegraphics[width=\columnwidth]{models/figures/analisys/pareto_w_high.png}

%	\caption{Pareto w energy}

%	\label{fig:pareto_w_h}

%\end{figure}

%
%\begin{figure}

%	\centering

%	\includegraphics[width=\columnwidth]{models/figures/analisys/pareto_static_low.png}

%	\caption{Pareto w energy}

%	\label{fig:pareto_w_l}

%\end{figure}

%
%\subsection{Optimization under constraints}

%
%\subsection{Gradient and Countorns}

%
%\begin{figure}[H]

%	\centering

%	\begin{subfigure}[b]{0.45\textwidth}

%		\includegraphics[width=\textwidth]{models/figures/analisys/pdyn0.png}

%	\end{subfigure}

%	%

%	\begin{subfigure}[b]{0.45\textwidth}

%		\includegraphics[width=\textwidth]{models/figures/analisys/pdyn0_3d.png}

%	\end{subfigure}

%\end{figure}

%
%\begin{figure}[H]

%	\centering

%	\begin{subfigure}[b]{0.45\textwidth}

%		\includegraphics[width=\textwidth]{models/figures/analisys/pdyn3.png}

%	\end{subfigure}

%	%

%	\begin{subfigure}[b]{0.45\textwidth}

%		\includegraphics[width=\textwidth]{models/figures/analisys/pdyn3_3d.png}

%	\end{subfigure}

%\end{figure}

%
%\begin{figure}[H]

%	\centering

%	\begin{subfigure}[b]{0.45\textwidth}

%		\includegraphics[width=\textwidth]{models/figures/analisys/pleak0.png}

%	\end{subfigure}

%	%

%	\begin{subfigure}[b]{0.45\textwidth}

%		\includegraphics[width=\textwidth]{models/figures/analisys/pleak0_3d.png}

%	\end{subfigure}

%\end{figure}

%
%\begin{figure}[H]

%	\centering

%	\begin{subfigure}[b]{0.45\textwidth}

%		\includegraphics[width=\textwidth]{models/figures/analisys/pleak10.png}

%	\end{subfigure}

%	%

%	\begin{subfigure}[b]{0.45\textwidth}

%		\includegraphics[width=\textwidth]{models/figures/analisys/pleak10_3d.png}

%	\end{subfigure}

%\end{figure}

%
%\begin{figure}[H]

%	\centering

%	\begin{subfigure}[b]{0.45\textwidth}

%		\includegraphics[width=\textwidth]{models/figures/analisys/pstatic0.png}

%	\end{subfigure}

%	%

%	\begin{subfigure}[b]{0.45\textwidth}

%		\includegraphics[width=\textwidth]{models/figures/analisys/pstatic0_3d.png}

%	\end{subfigure}

%\end{figure}

%
%\begin{figure}[H]

%	\centering

%	\begin{subfigure}[b]{0.45\textwidth}

%		\includegraphics[width=\textwidth]{models/figures/analisys/pstatic3000.png}

%	\end{subfigure}

%	%

%	\begin{subfigure}[b]{0.45\textwidth}

%		\includegraphics[width=\textwidth]{models/figures/analisys/pstatic3000_3d.png}

%	\end{subfigure}

%\end{figure}

%
%\begin{figure}[H]

%	\centering

%	\begin{subfigure}[b]{0.45\textwidth}

%		\includegraphics[width=\textwidth]{models/figures/analisys/w0.png}

%	\end{subfigure}

%	%

%	\begin{subfigure}[b]{0.45\textwidth}

%		\includegraphics[width=\textwidth]{models/figures/analisys/w0_3d.png}

%	\end{subfigure}

%\end{figure}

%
%\begin{figure}[H]

%	\centering

%	\begin{subfigure}[b]{0.45\textwidth}

%		\includegraphics[width=\textwidth]{models/figures/analisys/w1.png}

%	\end{subfigure}

%	%

%	\begin{subfigure}[b]{0.45\textwidth}

%		\includegraphics[width=\textwidth]{models/figures/analisys/w1_3d.png}

%	\end{subfigure}

%\end{figure}


	
	\chapter{Application-phase heuristic energy optimization} \label{chapter:phases}
	Dynamic voltage and frequency scaling (DVFS) and dynamic power management (DPM) are essential techniques for saving energy in modern computers. These techniques control hardware resources such as frequency and number of active processors. Although there is much work on the subject, there is an underexplored sub-topic: the impact of the choice of phase division on energy consumption.
	This chapter analyzes the impact of phase division on the energy efficiency of algorithms and proposes a methodology that combines measurement data with a heuristic to provide insights into choosing the best phase divisions.
	Our heuristic can reduce the scan space from $10^{7000}$ to $10^2$ with an average error of 10\% and up to 38\% reduction in energy consumption using optimal distribution compared to standard Linux DVFS. Moreover, we evaluated the trade-off of having too many divisions and the overhead caused. Finally, we have identified that there is usually a limit to the number of phases a running application can benefit from, giving a lower limit to the minimum number of phases.
	\section{Introduction} \label{sec:introduction}

Energy efficiency has been one of the main focuses in computing nowadays, both on mobile and HPC. It is essential to increase battery life and reduce heat generation, especially in high-performance computing (HPC), where a small percentage of the savings energy can significantly reduce maintenance costs or environmental impact due to its scale of consumption. For example, the leading Petaflop supercomputers consume in the range of 1 to 18 MW of electrical energy, with 1.5 MW on average, which we can estimate in millions of dollars in annual electricity cost \cite{Group2012HandbookSahni}. Furthermore, the International Energy Agency \cite{iea_2021} estimated that global data center electricity usage in 2020 was around 200-250 TWh, or around 1\% \cite{Corcoran2017EmergingICT} of global electricity demand, which could generate as much pollution as a nation like Argentina \cite{Mathew2012Energy-awareNetworks}.

Given the importance of this topic, modern desktops, cell phones, and HPC CPUs have built-in power management mechanisms. That is because the processor is one of the main components that drain energy and can reach 50\% of the total system consumption \cite{Fan2007, Barroso2007TheComputing, Malladi2012TowardsDRAM}. Among the mechanisms implemented by the CPU, the most impactful and controllable via software are Dynamic Frequency and Voltage Scaling (DVFS), and Dynamic Power Management (DPM) ~\cite{Rotem2012Power-managementBridge, Brown2005, Hackenberg2015}.

DPM encompasses a set of techniques that take advantage of different energy levels in the system, such as active, inactive, and disabled~\cite{CARDOSO201717, Shuja2012Energy-efficientCenters, Benini2000AManagement}. The basic idea is to turn off devices and turn them on when needed. However, it is not a trivial task, and several variables need to be considered, such as the cost of switching, in terms of time and energy, how long it will remain in that state, the energy consumption in each state, and much more.

The DPM technique has the most significant gains when in systems where the static power is very high or where the system is idle for a long time. In such cases, some papers report savings of up to 70\% in energy~\cite{Shuja2012Energy-efficientCenters, Benini2000AManagement}.

DVFS, on the other hand, allows real-time frequency and voltage control, depending on needs. This method is motivated by the fact that we can approximate the relationship of frequency to power as a cubic equation and frequency and performance as a linear equation \cite{Dayarathna2016DataSurvey, Group2012HandbookSahni} implying that a reduction in frequency has a cubic impact on power and linear performance. Still, it is not so simple because lowering the frequency leads to longer runtime, increasing power consumption. As a result, determining the best voltage and frequency to employ under all conditions is also a challenging task.

Although there are many studies on this topic, most works focus on implementing different optimization algorithms based on workload, and few studies pay attention to phase division. Instead, most studies use static phase division to simplify the optimization math and compensate by implementing lightweight algorithms. However, this leads to a sub-optimization, as another phase division could result in different frequency values. In addition, there is an overhead associated with the moment the algorithm takes action that could be reduced if we knew the ideal moment for it to take action, making it possible even to allow heavier algorithms. Therefore, this work proposes a methodology to study the division of time (phases) and analyzing possible gains.

\section{Related work} \label{sec:related_work}

One of the bases works in scheduling algorithms is the work of Irani et al. \cite{Irani2007}, where they formalized the problem of scheduling arriving jobs in a way that minimizes total energy use and so that each job is completed after its arrival time and before its deadline. Although each problem has been considered separately, this was the first theoretical analysis of systems that can use both mechanisms.

Past work has also used feedback-based approaches, as in Poellabauer et al. \cite{Poellabauer2005}, where applications' past CPU utilization's are used to predict future  CPU  requirements. However, mispredictions in these approaches can lead to missed deadlines,  sub-optimal energy savings,  or large overheads due to frequent changes to the chosen frequency or voltage. One shortcoming of previous approaches is that they ignore other 'indicators' of future CPU requirements, such as the frequency of I/O operations, memory accesses, or interrupts.

In the thesis work of Saha et al. \cite{Saha2012}, the timeliness and power consumption behavior of fourteen RT-DVFS schedulers through implementation and actual measurements. The schedulers include Static Earliest Deadline First (Static-EDF), Cycle Conserving Earliest Deadline First(CC-EDF), Look-Ahead Earliest Deadline First (LA-EDF), Snowdon-minimum (Snowdon-min), Resource-constrained Energy-Efficient Utility Accrual Algorithm (REUA), DynamicReclaiming Algorithm (DRA) and Aggressive Speed Reduction Algorithm(AGR) among the others. They draw attention to the fact that most of these algorithms are based on simulations, which can mislead some results. Broadly, RT-DVFS techniques have two objectives: (i) Reduce energy consumption through DVFS; and (ii) Optimize task timeliness behavior through real-time resource management (deadline).

Pietri et al. \cite{Pietri2014} used slack reclamation to achieve energy savings. The goal is to exploit the idle slots of the processors that occurred due to the earlier completion time of the tasks compared with the latest finish time, which is constrained by the deadline or data dependencies. In their model, the user submits a workflow for execution specifying a deadline for completion of the execution.

Mashayekhy et al. \cite{Mashayekhy2014} proposed a greedy algorithm, called Energy-aware  MapReduce  Scheduling  Algorithm  (EMRSA), that finds the assignments of the map and reduces tasks to the machine slots in order to minimize the energy consumed when executing the application. They consider an extensive data application consisting of a set of map and reduce tasks that need to be completed by deadline D.

Yousefi et al. \cite{Yousefi2018}  present a task-based greedy scheduling algorithm, TGSAVE. This algorithm selects a slot for each task to minimize the total energy consumption of the MapReduce job for big data applications in heterogeneous environments without significant performance loss while satisfying the service level agreement (SLA). TGSAVE  finds solutions under deadlines up to 74\% tighter than the tightest one feasible by EMRSA, and it can meet deadlines as tight as only12\%, on average, more significant than the energy-oblivious minimum makespan.

Crown scheduling is a static scheduling approach for sets of parallelizable tasks with a common deadline. Crown schedules are robust, i.e., the runtime prolongation of one task by a moderate percentage does not cause a deadline transgression by the same fraction. In addition, by speeding up some tasks scheduled after the lengthy task, the deadline can still be met at moderate additional energy consumption. In the work of Kessler et al. \cite{Kessler2021} they present a heuristic to perform this re-scaling online and explore the trade-off between additional energy consumption in normal execution and limitation of deadline transgression in delay cases.

%Typically target traditional real-time systems with static deadlines, resulting in conservative energy savings that cannot exploit additional energy optimizations due to dynamic deadlines. However, in the work of Yi et al. \cite{Yi2021} they present an adaptive system optimization and reconfiguration approach that dynamically adapts the scheduling parameters and processor speeds to satisfy dynamic deadlines while consuming as little energy as possible. As their work was for autonomous driving, they used velocities as deadlines.

In the work of \cite{Ajmal2021}, a green cloud computing algorithm named "Cost-based Energy Efficient Scheduling Technique for Dynamic Voltage Frequency Scaling (DVFS) Systems (CEEST)" is proposed. The proposed algorithm reduces energy consumption without compromising the quality of service (QoS). This algorithm aims to optimize and manage servers in the datacenters by utilizing maximum resources and powering off the underutilized servers. Furthermore, CEEST utilizes the scaling of virtual machines to finish jobs within the deadlines to reduce violations of service level agreement (SLA). 

All of these works at some point considered a deadline (defined by the user of the system) that defines the phases of the job; this means that choosing a different deadline could result in completed different energy savings, and because this is not the main focus of these works, this is not considered. In our paper, we study the impact of this phase division on the application's energy consumption. We estimate how much energy could be saved if an algorithm could magically give the ideal deadline.

Although most of the work is considered a deadline, some works include job division as part of the optimization problem. For example, in the paper of Agrawal et al. \cite{Agrawal2021}, they show that if the jobs can be divided into arbitrary parts, the minimum-energy schedule can be generated in linear time, giving exact scheduling algorithms. For the cases where jobs are non-divisible, they claimed a prove that the scheduling problems are NP-hard and also give approximation algorithms for the same along with their bounds. 

In this case, where no deadline is defined, the phase division is indirectly considered by solving the optimization problem of minimizing the energy. The problem, in this case, is that several constraints restrict the optimization problem while our approach works with measured data, meaning that we can find solutions independent of the constraints.

\section{Phase division approach} \label{sec:application_partition_method}
To find the optimal phase division, we would have to ensure that the one we choose is the most energy-efficient among all possible divisions and different combinations of power configurations.

This mathematical problem can be modeled in many ways, but always with some concessions to make it solvable. Another option would be by brute force, but there are infinite possibilities for splitting phases, although some hardware limitations make our analysis more accessible.

With that in mind, we decided to create a heuristic that comes the closest to brute force using hardware limitations to make it viable.

The first limitation is that the processor has a limited speed at which it can act. In other words, a division that can act would make no sense, limiting our analysis to discrete-time intervals at the processor's maximum performance speed.

This already limits our exploration space a lot, but even so, it is still unfeasible. For example, considering $d_t$ as the minimum processor action interval, $T$ the total application time, and $C$ the number of power settings. The number of possibilities could be estimated by:
\begin{equation}
	\left(\frac{T}{d_t}\right)^C,
\end{equation}
This number grows astronomically because $d_t$ is in a time range of microseconds, $T$ seconds or minutes, and C of hundreds. To give an idea, we can estimate this number to be in a range of $10^7000$.

Another thing that we can see is that each configuration gives the power profile for a given application, and phase division is just a combination of these configurations. This allows us to reduce further our exploration of space. For example, if we ran each power setting once and we had a way of combining them, our problem now would be feasible.

To make this possible, we assume that the configuration change only affects the program speed and that the power consumption is independent of the previous settings.

\iffalse %%%%%%%%%%%%%%%%%%%%%%%%%%%%%%%%%%%%%%%%%%%%%%% commented 
This means that in a certain part of the program $x_1$\% to $x_2$\% it will always execute the same instructions and the energy consumption would depend only on the configuration.

That said, if we treat the program as a percentage of execution, and we have the power consumption profile for all configurations, we can estimate the power consumption for any combination of phases.

For example if we have the phase division, 0-x_1\%, x_1\%-100\%, with the settings c_1 and c_2, the estimated energy would be:
P_{c_1}*u(0,x_1)*T_{c_1}+P_{c_2}*u(x_1, 100)*T_{c_2}

We can see that we just need a power profile P and the total runtime for each configuration and that we can divide the application into as many phases as we want.

For that we are going to assume that the number of instructions to complete a program is roughly the same independent on the frequency, thus we can write the number of instructions as:
\begin{equation}
	I=c_{k1}f_1T_1=c_{k2}f_2T_2,
\end{equation}
where $I$ is the number of instructions, $c_{k1}$ the rate of instructions per cycle, $f$ the frequency of the processor and $T$ the execution time. All the required parameters can be collected by simply running the application with the available frequencies. 

From this equation we can split the execution of the program into several different phases $i$ as follows:
% \begin{equation}
%     I=xI+(1-x)I=xc_{k1}f_1+(1-x)c_{k2}f_2
% \end{equation}
\begin{equation}
	I=\sum_{i}{a_iI} %=\sum_i{a_ic_{ki}f_i}
\end{equation}

with $a_i$ as the percentages of instructions executed with frequency $f_i$, we can compute any combination of phases. For convenience, we write this as a function of the time:
% \begin{equation}
% T=\frac{I}{c_{k1}f_1}
% \end{equation}
\begin{equation}
	I=Tc_{k}f
\end{equation}
\begin{equation}
	I=\sum_{i}{a_iI}=\sum_i{a_iTc_{ki}f_i}
\end{equation}
% \begin{equation}
% I=xI+(1-x)I=xTc_{k1}f_1+(1-x)Tc_{k2}f_2
% \end{equation}

the application being divided into phases of different duration, $a_i$  becomes the duration of the phase as a percentage of the total execution time. This general equation does not limit the number and duration of phases, and several phases can have the same frequency. 

In the event that we decide to include the number of cores as a context parameter, the equation becomes more complex. Indeed, the number of cores can completely change the application's behavior but, depending on the type of application, it's still possible to find similarities. Many HPC applications use data parallelism, and the Amdahl's law can model most of it. For this kind of applications, we can estimate that:
\begin{equation}
	T=\frac{T_s}{S}
\end{equation}
\begin{equation}
	T_s=\frac{I}{c_kf}
\end{equation}
\begin{equation}
	S=\frac{1}{1-w+\frac{w}{p}}
\end{equation}
\begin{equation}
	T=\frac{I(1-w+\frac{w}{p})}{c_kf}=\frac{I}{c_kf}-\frac{Iw}{c_kf}+\frac{Iw}{c_kfp}
\end{equation}

Looking at a fixed frequency, we arrive at a very similar structure as before:
\begin{equation}
	T=c_1I+\frac{c_2I}{p}
\end{equation}

As previously, we can use $I$ as the number of reference instructions, and compute $a$ as the percentage of single and multi-core instructions, or express it as a percentage of the total time T:
\begin{equation}
	aT=c_1aI+\frac{c_2aI}{p}
\end{equation}

The extent of the hypotheses will later be validated, however, by taking advantage of both principles, we can now, based on real measurements, estimate any phase division and possible context combination.
\fi %%%%%%%%%%%%%%%%%%%%%%%%%%%%%%%%%%%%%%%%%%%%%%%

This approach aims to bring the best of both worlds. We want simulation speed but with actual measurement data for greater accuracy. Our algorithm runs with measured data from a real system. The idea is to run the application with all possible configurations (e.g., frequency and number of active cores) of the machine without time division and estimate power consumption for different phase divisions.


\section{Building applications power profile} \label{sec:building_a_database}

%\subsection{Case-Study Architecture} \label{subsec:casestudyarchitecture}
%We executed the experiments in a system equipped with two Intel Xeon E5-2698 v3 processors with sixteen cores and two hardware threads for each core, with a total physical memory of the node is 128GB (8$\times$16GB). We disabled turbo frequency and hardware multi-threading during all experiments. The operating system used was Linux CentOS 6.5 and kernel 4.16. The overall view of the architecture is shown in \cref{fig:architecture}.
%
%\begin{figure}[H]
%	\centering
%	\includegraphics[width=0.8\columnwidth]{models/figures/architecture.png}
%	\caption{Node architecture (the image was made with the lstop application).}
%	\label{fig:architecture}
%\end{figure}
%
%To perform the frequency control, we used the acpi-cpufreq driver using the Userspace governor, which allows the user or any userspace program to set the CPU to a specific frequency. While the cores control was accomplished by modifying the appropriate system files with the default CPU-hotplug driver.
%
%The architecture is also equipped with the Intelligent Platform Management Interface (IPMI), a set of interfaces allowing out-of-band management of computer systems and platform-status monitoring via the local network~\cite{Schwenkler2006IntelligentInterface}. In addition, it can monitor variables and resources such as the system's temperature, voltage, fans, and power supplies, with independent sensors attached to the hardware.
%
%\subsection{Case-Study Applications} \label{subsec:casestudyapplication}
%The applications Black-Scholes, Bodytrack, Canneal, Dedup, Fluidanimate, Freqmine, Raytrace, Swaptions, Vips and x264 from the PARSEC \footnote{\url{https://parsec.cs.princeton.edu/download.htm}} parallel benchmark suite, version 3.0~\cite{Bienia2008TheSuite}, OpenMC \cite{Romano2015OpenMC:Development} and LINPACK (HPL) \cite{Dongarra1988TheExplanation}, were chosen as case studies. The PARSEC benchmark focused on emerging workloads and was designed to represent the next-generation shared-memory programs for chip-multiprocessors. It covers many areas such as financial analysis, computer vision, engineering, enterprise storage, animation, similarity search, data mining, machine learning, and media processing. The OpenMC and the LINPACK are two classic HPC programs.

\subsection{Data gathering} \label{sec:data_gathering}

Using the Pascal framework \cite{electronics11050689} the data was collected by running the application in all possible configurations. In this system, that is 32 cores and 13 frequencies, a total of 416 (32*13) possible configurations. Power samples were taken from dedicated sensors in the IPMI system with a sample rate of 0.5 seconds.

%\subsection{Application fingerprint}
%\textcolor{red}{Introduce the power profile as fingerprint briefly...}

\section{The phase optimization algorithm} \label{sec:heuristic}
To analyze the impact of configurations and identify the optimal phase division, we propose a zero-order integrator, which calculates the energy in an interval for a given configuration. The integrator uses the data previously described in \cref{sec:data_gathering}.

Figure \ref{fig:zero_order} shows an example energy computation for an application with four different power configurations. For that, first, the phases are defined in terms of the percentage of execution. Then, the integrator will run over each phase using the selected profile of power obtained from the measurements and accumulate the total energy consumption.


\begin{figure}[H]
	\centering
	\includegraphics[width=\columnwidth]{phases/figures/integrator.pdf}
	\caption{Power vs percentage of execution for a given application in several configurations of power. The energy is represented by the hatched area.}
	\label{fig:zero_order}
\end{figure}

The pseudo-algorithm for the integrator is described below:

\begin{lstlisting}[language=c++]
	integrator(total_time, nsamples,
	times, powers
	begin_phase, end_phase)
	{
		energy = 0;
		ti = total_time * begin_phase; // initial time
		tf = total_time * end_phase; // final time
		
		idx_i = 0, idx_f = 0;
		
		// find the index of first sample in the range
		for (idx_i= 0; idx_i<nsamples; idx_i++)
		if (times[idx_i] >= ti)
		break;
		idx_i = (idx_i != 0) ? idx_i - 1 : 0;
		
		// find the index of last sample in the range
		for (idx_f= idx_i; i<nsamples; idx_f++)
		if (times[idx_f] >= tf)
		break;
		idx_f = (idx_f != 0) ? idx_f - 1 : 0;
		
		// in case there is only one sample
		if (idx_i == idx_f) 
		return (tf-ti)*powers[idx_i];
		
		// handle the edges
		energy=(times[idx_i+1]-ti)*powers[idx_i]
		energy+=(tf-times[idx_f])*powers[idx_f];
		
		// handle sample in between
		for (int i= idx_i+1; i<idx_f; i++)
		energy += (times[i+1]-times[i])*powers[i];
		
		return energy;
	}
\end{lstlisting}

To obtain an optimal energy configuration for a given phase division, we can run all possible configurations and select the one that leads to a minimal consumption, as shown in the pseudo-algorithm below:

\begin{lstlisting}
	phase_optimization(configs, phases)
	{
		total_en = 0;
		for (i = 0; i<phases.size(); i++)
		{
			min_en = -1;
			for (j = 0; j<configs.size(); j++)
			{
				en = estimate_energy_phase(
				configs[j].total_time, 
				configs[j].nsamples,
				configs[j].time_samples,
				configs[j].power_samples,
				phases[i], phases[i + 1]);
				if (min_en == -1 || en < min_en)
				{
					min_en = en;
					min_conf = configs[j];
				}
			}
			total_en += min_en;
		}
		return total_en;
	}
\end{lstlisting}

This method can now obtain an optimal energy configuration for a given phase. The next step is to find the best phase, division.

\section{Results} \label{sec:results}

Now that we can evaluate phase divisions, we need to answer two questions, what is the best number of phases, and what is the best division within this number?

For that, we used the algorithm described previously in \cref{sec:heuristic} as the objective of a minimization problem where we want to find the vector of phases that minimizes the total energy. Moreover, by using a genetic algorithm, we can accelerate the process by controlling mutation and initial population generation since we already know some information about our problem.

For instance, we decide to start studying phases in the range of 3 to 100 divisions. We used a GA with a population of $10^3$ individuals. We kept the best 10\% individuals on each generation while having a 10\% of mutation rate, running for 300 generations. The reproduction function for the GA combines half-phases of one individual with another, and the mutation will randomly change a division.

We use the applications described in \cref{subsec:casestudyapplication}, to compute the optimal phase selection. The \cref{fig:relative_energy} shows the relative energy consumption compared to an optimized single phase.

\begin{figure}[H]
	\includegraphics[width=\columnwidth]{fingerprint/figures/energy_per_phase.pdf}
	\caption{Relative energy vs number of phases using applications from PARSEC 3.0, HPC, and Openmc benchmarks with different inputs. Energy compared to the single-phase optimal configuration.}
	\label{fig:relative_energy}
\end{figure}

This evaluation shows very interesting results, as shown in \cref{fig:relative_energy}. An ideal single-phase configuration provides the highest energy consumption for all cases. Furthermore, dividing the application into more than 35 phases is not necessary. On average, power consumption stopped after 35 phases.

% \begin{figure}[H]
% \includegraphics[width=\columnwidth]{figures/min_phases_distribution.pdf}
%     \caption{Histogram of minimal number of phases to optimal energy}
%     \label{fig:hist_optimal_phase}
% \end{figure}

The results meet our expectation that the optimum of n-divisions phases will always be better than or equal to n-1 divisions. This becomes clearer when we look at the number of cores, and frequency curve \cref{fig:freq_control} and \cref{fig:cores_control}, since starting from a certain number of divisions does not change the control signals since dividing an ideal phase into two would result in two phases with the same control signals. Therefore, the energy will remain the same, and there is no reason to increase the number of phases beyond that.

\begin{figure}[h]
	\centering
	\includegraphics[width=\columnwidth]{phases/figures/signals/completo_bodytrack_1_freq_signals_cmp.pdf}
	\caption{Frequency signal vs percentage of execution for the Bodytrack application, comparing the difference of 35 and 99 phases.}
\end{figure}%

\begin{figure}[h]
	\centering
	\includegraphics[width=\columnwidth]{phases/figures/signals/completo_freq_freq_signals_cmp.pdf}
	\caption{Frequency signal vs percentage of execution for the Freqmine application, comparing the difference of 35 and 99 phases.}
\end{figure}%

\begin{figure}[h]
	\centering
	\includegraphics[width=\columnwidth]{phases/figures/signals/completo_bodytrack_1_cores_signals_cmp.pdf}
	\caption{Cores signal vs percentage of execution for the Bodytrack application, comparing the difference of 35 and 99 phases.}
\end{figure}%
\begin{figure}[h]
	\centering
	\includegraphics[width=\columnwidth]{phases/figures/signals/completo_freq_cores_signals_cmp.pdf}
	\caption{Cores signal vs percentage of execution for the Freqmine application, comparing the difference of 35 and 99 phases.}
\end{figure}%

When we compare this method with the default DVFS algorithm in Linux, we observer an average saving of 38\% as shown in \cref{fig:cmp_ondemand} the relative energy per application.

\begin{figure}[H]
	\centering
	\includegraphics[width=\columnwidth]{phases/figures/comparison_ondemand.pdf}
	\caption{Relative energy comparison, optimal phase division energy compared to governor Ondemand on Linux.}
	\label{fig:cmp_ondemand}
\end{figure}

To illustrate the phases division, the \cref{fig:phase_division_cmap_35} and \cref{fig:phase_division_cmap_35} shows a heatmap with the divisions by application and the percentage of total energy spent in that phase. There we can see a pattern of 3 phases which for these applications are generally setup phase where it loads data from the disk, a computation phase where the data is processed, and a finalization phase where data is written back to the disk or displayed to the user.

\begin{figure}[H]
	\includegraphics[width=\columnwidth]{phases/figures/phase_division_cmap_3.pdf}
	\caption{Phase division heatmap (3 divisions) showing the energy consumption per phase for all applications.}
	\label{fig:phase_division_cmap_3}
\end{figure}

\begin{figure}[H]
	\includegraphics[width=\columnwidth]{phases/figures/phase_division_cmap_35.pdf}
	\caption{Phase division heatmap (35 divisions) showing the energy consumption per phase for all applications.}
	\label{fig:phase_division_cmap_35}
\end{figure}
	
	
	\chapter{Conclusions and future work} \label{chapter:conclusions}
	In this chapter, we deal with the conclusions about the effectiveness of the frameworks and the advantages and disadvantages of using models and algorithms with additional information on applications and architectures for energy optimization.
	We also discussed what could be improved in each approach and framework.
	\section{Conclusion} 

\subsection{Pascal Suite} \label{sec:conclusions_pascal}

This work introduces a practical and easy-to-use tool for measuring and analyzing the efficiency of parallel applications. The proposed tool focuses on observing scalability and energy consumption and implements features that enables analysis at hierarchical levels of the program's inner parts. It also simplifies the comparison of the application runs in different configurations, helping developers target software optimization efforts.

The tool has a low level of intrusion added to the program's performance measurement under analysis, which is a fundamental aspect of understanding the program's behavior and scalability capacity. 
Although it does not offer graphic elements to visualize the collected metrics, the tool itself makes it possible to examine the data from the command terminal. It also allows this data to be analyzed by an interface provided in Python or through .json files. 

\textls[-20]{In future work, we intend to evolve this tool to include the ability to predict speedup and efficiency from a few samples using state-of-the-art prediction models present in the \mbox{literature \cite{Alex2020WhenParallelSpeedups, Vitor2022AnalyticalEnergyModel}}. The idea is to present the general behavior of the program and its scalability trend and reduce the execution time necessary to compose a comprehensive analysis. Furthermore, we intend to include features that allow the observation of parallel applications in distributed environments that use the Message Passing Interface (MPI) standard.}

\subsubsection{fingerprint tool}

From the results, we observe that the API present overhead similar or lower to other low-level APIs, with the advantage of being in high abstraction and simplified configuration with a few lines of code, is possible to configure and gather counter data.

The tool developed provide also provided a way to fingerprint programs and compute similarities between different programs or the same program with different inputs. This can be useful to reduce applications spaces for benchmarks as was done in Polybench clustering but also to analyze the behavior of a parameter providing insights to the programmer to find a possible bottleneck.

We also provide a precise definition to input size that made possible fingerprint the programs, but it also can help programmers of benchmark applications to better create inputs with more precise growth of a particular parameter.

\subsection{Application model} \label{sec:conclusion_models}
The proposed energy model based on the operating frequency and the number of cores for a shared memory system can serve as a reference for DVFS and DPM optimization problems.

Results from three different HPC benchmarks demonstrate the potential of the proposed model while consuming 10 times less energy than a machine learning approach, such as SVR, to characterize applications. Moreover, it can provide knowledge-based hints to improve DVFS and DPM algorithms by enabling analysis of the contribution of each model parameter (e.g., level of parallelism) to the energy consumption. Indeed, as shown in Section \ref{sec:model_validation}, when no oracle is available to choose the frequency and the number of cores the application should use, the proposed model can save around 12\% of energy for a random choice and up to 70\% for the worse possible choice. Considering the job history of our own HPC center, which shows the prevalence of worse possible choices made by users, the potential energy savings are very significant and encourage further research.

Although the model is promising, it still has some limitations. The main one is related to the input size, which needs to be estimated to create the application model and optimize the application. Another limitation concerns the power model, which does not consider the load variation, so our model ends up using an average of the energy consumption, which is enough to obtain good results but limits its implementation in real-time optimization.
Future research is intended to solve both problems, first adapting the model to use the ratio of executed instructions as input size, something which is more tangible and easy to measure in modern systems without much overhead, and adding  new parameters to the power model to account for the load. This would allow us to develop more advanced DVFS models that could identify different phases of a target program with more subtle changes in frequency, and, perhaps, in the number of active cores to  further improve the results presented here.

The proposed energy model based on the frequency and number of cores for a full shared memory system can serve as a base for DVFS and DPM optimization problems that include both frequency and active cores. As well for analysis of the contribution of each parameter (ex: parallelism level) to the energy consumption.

Results from 3 HPC benchmarks running in one cluster demonstrate the potential of the proposed novel model. While consuming less energy than traditional machine learning approaches, it can serve as bases from DVFS and DPM algorithm as shown in the \ref{sec:model_validation} in an average case saving about 12\% up to 69\%. The previous knowledge of the application's performance can expose sufficiently relevant information, such as parallel speedups, that is harder to guess in run-time techniques based on DVFS.

A weakness of the proposed model is the need for information about the input size of the application that can be complex to derive. A possible solution would be to precisely define what is input size, given an definition in function of common variable for all applications like throughput for example. Future work will demonstrate all the possible analysis that is possible to archive with the equation as long as more advanced DVFS models that can be derived using the equation. For instance identification of different phases of the target program and thus, it will enable more fine-grained changes of the frequency and, perhaps, the number of active cores, to further improve the results presented here.

Another important aspect that is typically not taken into account is the number of processing cores to be used by a parallel program. This choice is left to the user, which often is not trivial as shown in this paper.


\subsection{Application-phase heuristic} \label{sec:phases_conclusion}
The main conclusion of this work was that in the HPC environment, in addition to the advantages of finding the optimal phase divisions, not many phases are needed to achieve the optimal energy consumption. Moreover, an average maximum of 35 phases was sufficient for the three benchmarks used, covering the most varied HPC applications.
It was also possible to notice that there is still a lot to improve in optimizing the DVFS algorithms since we obtained an average of 38\% energy savings compared to Ondemand on Linux.
We also noticed that there is a relation between the application behavior and the phases locations, independently of their number.

For future work, the final idea is to use the information from the best phase to find a performance metric that can provide us with this division without needing data. In future work, we intend to calculate these metrics using a combination of hardware performance counters that allow real-time optimal phase division. This metric can improve current DVFS algorithms by providing them with a workload metric that can improve energy savings.

	
	%\bibliographystyle{plain}
	%\bibliography{references.bib}
	\begin{thebibliography}{999}
	
	\bibitem[Ishfag(2012)]{Group2012HandbookSahni}
	Ishfag, A.; Sanjay, R. \textit{Handbook of Energy-Aware and Green Computing}; Chapman \& Hall/CRC: London, England, 2012; Volume~1, pp. 702--713.
	
	\bibitem[Dayarathna(2016)]{Dayarathna2016DataSurvey}
	Dayarathna, M.; Wen, Y.; Fan, R. Data Center Energy Consumption Modeling: A Survey. {\em IEEE Commun. Surv. Tutor.} {\bf 2016}, {\em 18}, 732--794. [\href{http://doi.org/10.1109/COMST.2015.2481183}{CrossRef}]
	
	\bibitem[Corcoran(2017)]{Corcoran2017EmergingICT}
	Corcoran, P.; Andrae, A. \emph{Emerging Trends in Electricity Consumption for Consumer ICT}; National University of Ireland: Galway, Ireland, 2013; pp. 1--56.
	
	\bibitem[Mathew(2012)]{Mathew2012Energy-awareNetworks}
	Mathew, V.; Sitaraman, R. K.; Shenoy, P. Energy-aware load balancing in content delivery networks. {In Proceedings of the 2012 Proceedings IEEE INFOCOM, Orlando, FL, USA, 25--30 March 2012}; pp. 954--962.
	
	\bibitem[Rivoire(2007)]{Rivoire2007ModelsOptimizations}
	Rivoire, S.; Shah, M.A.; Ranganathan, P.; Kozyrakis, C.; Meza, J. Models and Metrics to Enable Energy-Efficiency Optimizations. {\em Computer} {\bf 2007}, {\em 40}, 39--48. [\href{http://dx.doi.org/10.1109/MC.2007.436}{CrossRef}]
	
	\bibitem[Buyya(2013)]{Buyya2013Introduction}
	Buyya, R.; Vecchiola, C.; Selvi, S.T. \textit{Mastering Cloud Computing}; Morgan Kaufmann Publishers Inc.: San Francisco, CA, USA, 2013; pp. 3--27
	
	\bibitem[Poess(2008)]{Poess2008EnergyCenters}
	Poess, M.; Nambiar, R.O. Energy cost, the key challenge of today's data centers. {\em Proc. VLDB Endow.} {\bf 2008}, {\em 1}, 1229--1240. [\href{http://dx.doi.org/10.14778/1454159.1454162}{CrossRef}]
	
	\bibitem[Gao(2013)]{Gao2013QualityCenters}
	Gao, Y.; Guan, H.; Qi, Z.; Wang, B.; Liu, L. Quality of service aware power management for virtualized data centers. {\em J. Syst. Archit.} {\bf 2013}, {\em 59}, 245--259. [\href{http://dx.doi.org/10.1016/j.sysarc.2013.03.007}{CrossRef}]
	
	\bibitem[Fan(2007)]{Fan2007PowerComputer}
	Fan, X.; Weber, W.D.; Barroso, L.A. Power provisioning for a warehouse-sized computer. {\em ACM SIGARCH Comput. Archit. News} {\bf 2007}, {\em 35}, 13--23. [\href{http://dx.doi.org/10.1145/1273440.1250665}{CrossRef}]
	
	\bibitem[Barroso(2007)]{Barroso2007TheComputing}
	Barroso, L.A.; H{\"{o}}lzle, U. The Case for Energy-Proportional Computing. {\em Computer} {\bf 2007}, {\em 40}, 33--37. [\href{http://dx.doi.org/10.1109/MC.2007.443}{CrossRef}]
	
	\bibitem[Malladi(2012)]{Malladi2012TowardsDRAM}
	Malladi, K.T.; Nothaft, F.A.; Periyathambi, K.; Lee, B.C.; Kozyrakis, C.; Horowitz, M. Towards energy-proportional datacenter memory with mobile DRAM. {In  Proceedings of the 2012 39th Annual International Symposium on Computer Architecture (ISCA), Portland, OR, USA, 9--13 June 2012}; pp. 37--48.
	
	\bibitem[Rotem(2012)]{Rotem2012Power-managementBridge}
	Rotem, E.; Naveh, A.; Ananthakrishnan, A.; Weissmann, E.; Rajwan, D. Power-Management Architecture of the Intel Microarchitecture Code-Named Sandy Bridge. {\em IEEE Micro} {\bf 2012}, {\em 32} 20--27. [\href{http://dx.doi.org/10.1109/MM.2012.12}{CrossRef}]
	
	\bibitem[Brown(2005)]{Brown2005ACPILinux}
	Brown, L.; Moore, R.; Li, D.S.; Yu, L.; Keshavamurthy, A.; Pallipadi, V. ACPI in Linux. {\em Symposium } {\bf 2005}, {\em 51}, 1--51.
	
	\bibitem[Hackenberg(2015)]{Hackenberg2015AnProcessor}
	Hackenberg, D.; Schone, R.; Ilsche, T.; Molka, D.; Schuchart, J.; Geyer, R. An Energy Efficiency Feature Survey of the Intel Haswell Processor. {In  Proceedings of the  2015 IEEE International Parallel and Distributed Processing Symposium Workshop, Hyderabad, India, 25--29 May 2015}; pp. 896--904.
	
	\bibitem[{Intel}(2020)]{Intel20200thLake}
	Intel. \textit{12th Generation Intel {\textregistered} Core™ Processors}; Intel: Santa Clara, CA, USA, 2020; pp. 420--430
	
	\bibitem[Shuja(2012)]{Shuja2012Energy-efficientCenters}
	Shuja, J.; Madani, S.A.; Bilal, K.; Hayat, K.; Khan, S.U.; Sarwar, S. Energy-efficient data centers. {\em Computing} {\bf 2012}, {\em 94}, 973--994. [\href{http://dx.doi.org/10.1007/s00607-012-0211-2}{CrossRef}]
	
	\bibitem[Benini(2000)]{Benini2000AManagement}
	Benini, L.; Bogliolo, A.; De~Micheli, G. A survey of design techniques for system-level dynamic power management. {\em IEEE Trans. Very Large Scale Integr. (VLSI) Syst.} {\bf 2000}, {\em 8}, 299--316. [\href{http://dx.doi.org/10.1109/92.845896}{CrossRef}]
	
	\bibitem[Merkel(2006)]{Merkel2006BalancingSystems}
	Merkel, A.; Bellosa, F. Balancing power consumption in multiprocessor systems. {\em ACM SIGOPS/EuroSys Eur. Conf. Comput. Syst.} {\bf 2006}, \emph{40}, 403--4014.
	
	\bibitem[Roy(2013)]{Roy2013AnAlgorithms}
	Roy, S.; Rudra, A.; Verma, A. An energy complexity model for algorithms. In Proceedings of the 4th conference on Innovations in Theoretical Computer Science, New York, NY, USA, 9--12 January 2013.
	
	\bibitem[Weaver(2008)]{Weaver2008CanTrusted}
	Weaver, V.M.; McKee, S.A. Can hardware performance counters be trusted? {In Proceedings of the 2008 IEEE International Symposium on Workload Characterization, Seattle, WA, USA, 14--16 September 2008}; pp. 141--150.
	
	\bibitem[Weaver(2013)]{Weaver2013Non-determinismImplementations}
	Weaver, V.M.; Terpstra, D.; Moore, S. Non-determinism and overcount on modern hardware performance counter implementations. {In Proceedings of the 2013 IEEE International Symposium on Performance Analysis of Systems and Software (ISPASS), Austin, TX, USA, 21--23 April 2013}; pp. 215--224.
	
	\bibitem[Das(2019)]{Das2019SoK:Security}
	Das, S.; Werner, J.; Antonakakis, M.; Polychronakis, M.; Monrose, F. SoK: The Challenges, Pitfalls, and Perils of Using Hardware Performance Counters for Security. {In Proceedings of the 2019 IEEE Symposium on Security and Privacy (SP), San Francisco, CA, USA, 19--23 May 2019}; pp. 20--38.
	
	\bibitem[Guire(2009)]{McGuire2009AnalysisKernel}
	Mc~Guire, N.; Okech, P.; Schiesser, G. Analysis of Inherent Randomness of the Linux Kernel. {In Proceedings of the Eleventh RealTime Linux Workshop, Dresden, Germany, 28–30 September 2009}.
	
	\bibitem[Ramos(2019)]{Ramos2019AnCounters}
	Ramos, V.; Valderrama, C.; Xavier~de Souza, S.; Manneback, P. An Accurate Tool for Modeling, Fingerprinting, Comparison, and Clustering of Parallel Applications Based on Performance Counters. {In Proceedings of the  IEEE International Parallel and Distributed Processing, Rio de Janeiro, Brazil, 20--24 May 2019}; pp. 797--804.
	
	\bibitem[Silva-de Souza(2020)]{Silva-de-Souza2020ContainergyAWorkloads}
	Silva-de Souza, W.; Iranfar, A.; Br{\'{a}}ulio, A.; Zapater, M.; Xavier-de Souza, S.; Olcoz, K.; Atienza, D. Containergy—A Container-Based Energy and Performance Profiling Tool for Next Generation Workloads. {\em Energies} {\bf 2020}, {\em 13}, ~2162. [\href{http://dx.doi.org/10.3390/en13092162}{CrossRef}]
	
	\bibitem[Shao and Brooks(2013)]{Shao2013EnergyProcessor}
	Shao, Y.S.; Brooks, D. Energy characterization and instruction-level energy model of Intel's Xeon Phi processor. In Proceedings of the  International Symposium on Low Power Electronics and Design (ISLPED), Beijing, China, 4--6 September 2013; pp. 389--394.
	
	\bibitem[Lewis(2008)]{Lewis2008Run-timeSystems}
	Lewis, A.; Ghosh, S.; Tzeng, N.F. Run-time energy consumption estimation based on workload in server systems. {In Proceedings of the 2008 Conference on Power Aware Computing and Systems, San Diego, CA, USA, 8--10 December 2008}; pp. 3--4.
	
	\bibitem[Mills(2014)]{Mills2014EnergySystems}
	Mills, B.; Znati, T.; Melhem, R.; Ferreira, K.B.; Grant, R.E. Energy Consumption of Resilience Mechanisms in Large Scale Systems. In Proceedings of the 2014 22nd Euromicro International Conference on Parallel, Distributed, and Network-Based Processing, Turin, Italy, 12--14 February 2014;  pp. 528--535.
	
	\bibitem[Feng(2003)]{Feng2003MakingSupercomputing}
	Feng, W.c. Making a Case for Efficient Supercomputing. {\em Queue} {\bf 2003}, {\em 1}, 54--64. [\href{http://dx.doi.org/10.1145/957717.957772}{CrossRef}]
	
	\bibitem[Sarwar(1997)]{Sarwar1997CmosCalculation}
	Sarwar, A. Cmos power consumption and cpd calculation. In {\em Proceeding: Design Considerations for Logic Products};  Texas Instruments: Dallas, TX, USA, {1997}.
	
	\bibitem[Butzen and Ribas(2007)]{Butzen2007LeakageGates}
	Butzen, P.; Ribas, R. {\em Leakage Current in Sub-Micrometer CMOS Gates}; {Universidade Federal do Rio Grande do Sul}: Porto Alegre, Brazil, { 2007}; pp. 1--30.
	
	\bibitem[Amdahl(1967)]{Amdahl1967ValidityCapabilities}
	Amdahl, G.M. {Validity of the single processor approach to achieving large scale computing capabilities}. In Proceedings of the Spring Joint Computer Conference on---AFIPS '67 (Spring), New York, NY, USA, 18--20 April 1967.
	
	\bibitem[Eyerman(2010)]{Eyerman2010ModelingDesign}
	Eyerman, S.; Eeckhout, L. {Modeling critical sections in Amdahl's law and its implications for multicore design}. In Proceedings of the 37th Annual International Symposium on Computer Architecture---ISCA '10, New York, NY, USA, 19--23 June 2010.
	
	\bibitem[Gustafson(1988)]{Gustafson1988ReevaluatingLaw}
	Gustafson, J.L. Reevaluating Amdahl's law. {\em Commun. ACM} {\bf 1988}, {\em 31}, 532--533. [\href{http://dx.doi.org/10.1145/42411.42415}{CrossRef}]
	
	\bibitem[Seel(2012)]{Hypothesis2012EncyclopediaLearning}
	Seel, N.M. \emph{Encyclopedia of the Sciences of Learning}; {Springer:  {Berlin/Heidelberg, Germany,} %newly added information, please confirm
	} {1988}; pp. 223--242.
	
	\bibitem[Roy(2019)]{Roy2019ForecastingNetwork}
	Roy, P.; Mahapatra, G.S.; Dey, K.N. Forecasting of software reliability using neighborhood fuzzy particle swarm optimization based novel neural network. {\em IEEE/CAA J. Autom. Sin.} {\bf 2019}, {\em 6}, 1365--1383. [\href{http://dx.doi.org/10.1109/JAS.2019.1911753}{CrossRef}]
	
	\bibitem[Zhu(2019)]{Zhu2019PredictingLearning}
	Zhu, W.; Liu, X.; Xu, M.; Wu, H. Predicting the results of RNA molecular specific hybridization using machine learning. {\em IEEE/CAA J. Autom. Sin.} {\bf 2019}, {\em 6}, 1384--1396. [\href{http://dx.doi.org/10.1109/JAS.2019.1911756}{CrossRef}]
	
	\bibitem[Rivoire(2008)]{Rivoire2008AModels}
	Rivoire, S.; Ranganathan, P.; Kozyrakis, C. A comparison of high-level full-system power models. {In Proceedings of the 2008 Conference on Power Aware Computing and Systems, San Diego, CA, USA, 8--10 December 2008}; pp. 1--5.
	
	\bibitem[Usman(2013)]{Usman2013ANoC}
	Usman, S.; Khan, S.U.; Khan, S. A comparative study of voltage/frequency scaling in NoC. {In Proceedings of the  IEEE International Conference on Electro-Information Technology, Rapid City, SD, USA, 9--11 May 2013}; pp. 1--5.
	
	\bibitem[Paolillo(2018)]{Paolillo2018OptimisationParallelism}
	Paolillo, A. Optimisation of Performance Metrics of Embedded Hard Real-Time
	Systems using Software/Hardware Parallelism, Ph.D. Thesis, Université libre de Bruxelles, Brussels, Belgium, 2018.
	
	\bibitem[Kim(2015)]{Kim2015RacingHeuristics}
	Kim, D.H.; Imes, C.; Hoffmann, H. Racing and Pacing to Idle: Theoretical and Empirical Analysis of
	Energy Optimization Heuristics. {In Proceedings of the  2015 IEEE 3rd International Conference on Cyber-Physical Systems, Networks, and Applications, Hong Kong, China, 19--21 August 2015}; pp. 78--85.
	
	\bibitem[Fu(2018)]{Fu2018RaceMinimization}
	Fu, C.; Chau, V.; Li, M.; Xue, C.J. Race to idle or not: Balancing the memory sleep time with DVS for energy minimization. {\em J. Comb. Optim.} {\bf 2018}, {\em 35}, 860--894. [\href{http://dx.doi.org/10.1007/s10878-017-0229-7}{CrossRef}]
	
	\bibitem[Rauber(2014)]{Rauber2014EnergyScaling}
	Rauber, T.; R{\"{u}}nger, G.; Schwind, M.; Xu, H.; Melzner, S. Energy measurement, modeling, and prediction for processors with frequency scaling. {\em J. Supercomput.} {\bf 2014}, {\em 70}, 1451--1476. [\href{http://dx.doi.org/10.1007/s11227-014-1236-4}{CrossRef}]
	
	\bibitem[Goel(2016)]{Goel2016AProcessors}
	Goel, B.; McKee, S.A. A Methodology for Modeling Dynamic and Static Power Consumption for Multicore Processors. {In Proceedings of the IEEE International Parallel and Distributed Processing Symposium, Chicago, IL, USA, 23--27 May 2016}; pp. 273--282.
	
	\bibitem[Du(2017)]{Du2017ModelingSystems}
	Du, Z.; Ge, R.; Lee, V.W.; Vuduc, R.; Bader, D.A.; He, L. Modeling the Power Variability of Core Speed Scaling on Homogeneous Multicore Systems. {\em Sci. Program.} {\bf 2017}, {\em 2017}, 1--13. [\href{http://dx.doi.org/10.1155/2017/8686971}{CrossRef}]
	
	\bibitem[Gonzalez(1997)]{Gonzalez1997SupplyCMOS}
	Gonzalez, R.; Gordon, B.; Horowitz, M. Supply and threshold voltage scaling for low power CMOS. {\em IEEE J. Solid-State Circuits} {\bf 1997}, {\em 32}, 1210--1216. [\href{http://dx.doi.org/10.1109/4.604077}{CrossRef}]
	
	\bibitem[Silva(2019)]{Silva2019Energy-OptimalApplications}
	Silva, V.R.G.; Furtunato, A.F.A.; Georgiou, K.; Sakuyama, C.A.V.; Eder, K.; Xavier-de Souza, S. Energy-Optimal Configurations for Single-Node HPC Applications. {In Proceedings of the  2019 International Conference on High Performance Computing \& Simulation (HPCS), Dublin, Ireland, 15--19 July 2019}; pp. 448--454.
	
	\bibitem[Kumar(1994)]{Kumar1994AnalyzingArchitectures}
	Kumar, V.; Gupta, A. Analyzing Scalability of Parallel Algorithms and Architectures. {\em J. Parallel  Distrib. Comput.} {\bf 1994}, {\em 22}, 379--391. [\href{http://dx.doi.org/10.1006/jpdc.1994.1099}{CrossRef}]
	
	\bibitem[Oliveira(2018)]{Oliveira2018ApplicationCharacterization}
	Oliveira, V.H.F.; Furtunato, A.F.A.; Silveira, L.F.; Georgiou, K.; Eder, K.; Xavier-de Souza, S. Application Speedup Characterization. {In Proceedings of the ACM/SPEC International Conference on Performance Engineering, Berlin, Germany, 9--13 April 2018}; pp. 43--44.
	
	\bibitem[Smola(2004)]{Smola2004ARegression}
	Smola, A.J.; Sch{\"{o}}lkopf, B. A tutorial on support vector regression. {\em Stat. Comput.} {\bf 2004}, {\em 14}, 199--222. [\href{http://dx.doi.org/10.1023/B:STCO.0000035301.49549.88}{CrossRef}]
	
	\bibitem[Kitts(2006)]{Kitts2006RegressionLecture}
	Kitts, B. Regression Trees Lecture. {\em Data Min.} {\bf 2006}, 6--7.
	
	\bibitem[Altman(1992)]{Altman1992AnRegression}
	Altman, N.S. An Introduction to Kernel and Nearest-Neighbor Nonparametric Regression. {\em  Am. Stat.} {\bf 1992}, {\em 46}, 175--185.
	
	\bibitem[Murtagh(1991)]{Murtagh1991MultilayerRegression}
	Murtagh, F. Multilayer perceptrons for classification and regression. {\em Neurocomputing} {\bf 1991}, {\em 2}, 183--197. [\href{http://dx.doi.org/10.1016/0925-2312(91)90023-5}{CrossRef}]
	
	\bibitem[Gao(2019)]{Gao2019DendriticPrediction}
	Gao, S.; Zhou, M.; Wang, Y.; Cheng, J.; Yachi, H.; Wang, J. Dendritic Neuron Model With Effective Learning Algorithms for Classification, Approximation, and Prediction. {\em IEEE Trans. Neural Netw. Learn. Syst.} {\bf 2019}, {\em 30}, 601--614. [\href{http://dx.doi.org/10.1109/TNNLS.2018.2846646}{CrossRef}]
	
	\bibitem[Schwenkler(2006)]{Schwenkler2006IntelligentInterface}
	Schwenkler, T.; Deutschland, S. Intelligent Platform Management Interface. In \emph{Sicheres Netzwerkmanagement}; Springer: Berlin/Heidelberg, Germany, 2006; pp. 169--207.
	
	\bibitem[Bienia(2008)]{Bienia2008TheSuite}
	Bienia, C.; Kumar, S.; Singh, J.P.; Li, K. The PARSEC benchmark suite. In Proceedings of the 17th international conference on Parallel architectures and compilation techniques---PACT '08, New York, NY, USA, 25--29 October 2008.
	
	\bibitem[Romano(2015)]{Romano2015OpenMC:Development}
	Romano, P.K.; Horelik, N.E.; Herman, B.R.; Nelson, A.G.; Forget, B.; Smith, K. OpenMC: A state-of-the-art Monte Carlo code for research and development. {\em Ann. Nucl. Energy} {\bf 2015}, {\em 82}, 90--97. [\href{http://dx.doi.org/10.1016/j.anucene.2014.07.048}{CrossRef}]
	
	\bibitem[Dongarra(1988)]{Dongarra1988TheExplanation}
	Dongarra, J.J. The LINPACK Benchmark: An explanation. In Proceedings of the 1st International Conference on Supercomputing, Athens, Greece, 8--12 June 1987.
	
	\bibitem[Pedregosa(2011)]{Pedregosa2011Scikit-learn:Python}
	Pedregosa, F; Varoquaux, G.; Gramfort, A.; Michel, V.; Thirion, B.; Grisel, O.; Blondel, M.;  Prettenhofer, P.; Weiss, R.; Dubourg, V.; et al. Scikit-learn: Machine Learning in {\{}P{\}}ython. {\em J. Mach. Learn. Res.} {\bf 2011}, {\em 12}, 2825--2830.
	
	\bibitem[Royer(2020)]{Royer2020AOptimization}
	Royer, C.W.; O’Neill, M.; Wright, S.J. A Newton-CG algorithm with complexity guarantees for smooth
	unconstrained optimization. {\em Math. Programm.} {\bf 2020}, {\em 180}, 451--488.
	
	\bibitem[Horyath(2008)]{Horyath2008Multi-mode}
	Tibor Horyath and Kevin Skadron.
	\newblock {Multi-mode energy management for multi-tier server clusters}.
	\newblock {\em Parallel Architectures and Compilation Techniques - Conference Proceedings, PACT}, 270--279, 2008.

	\bibitem[Demme(2013)]{Demme2013OnCounters}
	John Demme, Matthew Maycock, Jared Schmitz, Adrian Tang, Adam Waksman, Simha
	Sethumadhavan, and Salvatore Stolfo.
	\newblock {On the feasibility of online malware detection with performance
	counters}.
	\newblock {\em ACM SIGARCH Computer Architecture News}, 41(3):559, 2013.
	
	\bibitem[Eranian(2008)]{Eranian2008Perfmon2}
	Stephane Eranian.
	\newblock {Perfmon2: a standard performance monitoring interface for Linux}.
	\newblock {\em Slides, perfmon2 overview}, 2008.

	\bibitem[Hahnel(2012)]{Hahnel2012RAPL}
	Marcus H{\"{a}}hnel, Bj{\"{o}}rn D{\"{o}}bel, Marcus V{\"{o}}lp, and Hermann
	H{\"{a}}rtig.
	\newblock {Measuring energy consumption for short code paths using RAPL}.
	\newblock {\em ACM SIGMETRICS Performance Evaluation Review}, 40(3):13, 2012.

	\bibitem[Zamani(2012)]{Zamani2012ASystems}
	Reza Zamani and Ahmad Afsahi.
	\newblock {A study of hardware performance monitoring counter selection in
	power modeling of computing systems}.
	\newblock {\em 2012 International Green Computing Conference, IGCC 2012}, 2012.

	\bibitem[IPMI(2013)]{IPMI2013ConfigurationGuide}
	Published November.
	\newblock {IPMI Configuration User Guide}.
	\newblock 2017(November), 2013.

	\bibitem[Murtagh(2011)]{Murtagh2011WardsAlgorithm}
	Fionn Murtagh and Pierre Legendre.
	\newblock {Ward's Hierarchical Clustering Method: Clustering Criterion and
	Agglomerative Algorithm}.
	\newblock (June):1--20, 2011.

	\bibitem[Mucci(1999)]{Mucci1999PAPI}
	PJ~Mucci, Shirley Browne, Christine Deane, and George Ho.
	\newblock {PAPI: A portable interface to hardware performance counters}.
	\newblock {\em Proceedings of the department of defense HPCMP users group
	conference}, 32:7–10, 1999.

	\bibitem[Luo(2005)]{Luo2005PropertiesDifferentiators}
	Jianwen Luo, Kui Ying, Ping He, and Jing Bai.
	\newblock {Properties of Savitzky-Golay digital differentiators}.
	\newblock {\em Digital Signal Processing: A Review Journal}, 15(2):122--136,
	2005.

	\bibitem[Kufrin(2005)]{Kufrin2005Perfsuite}
	Rick Kufrin.
	\newblock {Perfsuite: An accessible, open source performance analysis
	environment for linux}.
	\newblock {\em Dans Presented at The 6th International Conference on Linux
	Clusters: The HPC Revolution}, 151(April):5, 2005.

	\bibitem[Knupfer(2011)]{Knupfer2011Scorep}
	Andreas Kn{\"{u}}pfer and Christian R{\"{o}}ssel.
	\newblock {Score-P – A Joint Performance Measurement Run-Time Infrastructure
	for}.
	\newblock 1--12, 2011.

	\bibitem[Jurman(2009)]{Jurman2009CanberraLists}
	Giuseppe Jurman, Samantha Riccadonna, Roberto Visintainer, and Cesare
	Furlanello.
	\newblock {Canberra distance on ranked lists}.
	\newblock {\em Proceedings, Advances in Ranking-NIPS 09 Workshop}, pages
	22--27, 2009.

	\bibitem[Hang(2017)]{Hang2017CubicApplications}
	Houjun Hang, Xing Yao, Qingqing Li, and Michel Artiles.
	\newblock {Cubic B-Spline Curves with Shape Parameter and Their Applications}.
	\newblock {\em Mathematical Problems in Engineering}, 2017:1--8, 2017.

	\bibitem[Intel(2013)]{Intel2013IntelGuide}
	{Intel}.
	\newblock {Intel{\textregistered} 64 and IA-32 Architectures Software
	Developer’s Manual, Volume 3 (3A, 3B {\&} 3C): System Programming Guide}.
	\newblock 3(253665):1--1386, 2013.
	
	
	%%%%%%%%%%%%%%
	
	\bibitem[Huck(2007)]{Huck2007}
	Huck, K.; Malony, A.; Shende, S.; Morris, A.
	\newblock Scalable, Automated Performance Analysis with TAU and PerfExplorer.
	\newblock In Proceedings of the PARCO, Aachen, Germany, 4--7 September 
	2007; Volume~15, pp. 629--636.
	
	\bibitem[Islam(2019)]{Islam2019}
	Islam, T.; Ayala, A.; Jensen, Q.; Ibrahim, K.
	\newblock Toward a Programmable Analysis and Visualization Framework for
	Interactive Performance Analytics.
	\newblock In Proceedings of the IEEE/ACM International Workshop on Programming and Performance
	Visualization Tools (ProTools), Denver, CO, USA, 17 November 2019; pp. 70--77.
	\newblock [\href{http://doi.org/10.1109/ProTools49597.2019.00015}{CrossRef}]
	
	
	\bibitem[Weber(2019)]{Weber2019}
	\textls[-20]{Weber, M.; Ziegenbalg, J.; Wesarg, B.
		\newblock Online Performance Analysis with the Vampir Tool Set. In {\em Tools
			for High Performance Computing 2017, Proceedings of the 11th International Workshop on Parallel Tools for High Performance Computing, Dresden, Germany, 11--12 September, 2017
		}; Springer International Publishing: Cham, Switzerland,
		2019; pp. 129--143.
		\newblock
	} [\href{http://dx.doi.org/10.1007/978-3-030-11987-4_8}{CrossRef}]
	
	\bibitem[Bergel(2019)]{Bergel2019}
	\textls[-30]{Bergel, A.; Bhatele, A.; Boehme, D.; Gralka, P.; Griffin, K.; Hermanns, M.A.;
		Okanović, D.; Pearce, O.; Vierjahn, T. \emph{Visual Analytics Challenges in
			Analyzing Calling Context Trees}; Springer: Cham, Switzerland, 2019; pp. 233--249.
		\newblock
	} [\href{http://dx.doi.org/10.1007/978-3-030-17872-7_14}{CrossRef}]
	
	\bibitem[Malony(2005)]{Huck2005}
	Huck, K.; Malony, A.
	\newblock PerfExplorer: A Performance Data Mining Framework For Large-Scale
	Parallel Computing.
	\newblock In Proceedings of the SC '05: Proceedings of the 2005 ACM/IEEE Conference on
	Supercomputing, Seattle, WA, USA, 12--18 November 2005; p.~41.
	\newblock [\href{http://dx.doi.org/10.1109/SC.2005.55}{CrossRef}]
	
	
	\bibitem[Geimer(2010)]{Geimer2010}
	Geimer, M.; Wolf, F.; Wylie, B.; {\'A}brah{\'a}m, E.; Becker, D.; Mohr, B.
	\newblock The Scalasca performance toolset architecture.
	\newblock {\em Concurr. Comput. Pract. Exp.} {\bf
		2010}, {\em 22}, 702--719. [\href{http://dx.doi.org/10.1002/cpe.1556}{CrossRef}]
	
	\bibitem[Malony(2006)]{Shende2006}
	Shende, S.S.; Malony, A.D.
	\newblock The Tau Parallel Performance System.
	\newblock {\em Int. J. High Perform. Comput. Appl.} {\bf 2006}, {\em 20},~287--311.
	\newblock [\href{http://dx.doi.org/10.1177/1094342006064482}{CrossRef}]
	
	
	\bibitem[Adhianto(2010)]{Adhianto2010}
	Adhianto, L.; Banerjee, S.; Fagan, M.; Krentel, M.; Marin, G.; Mellor-Crummey,
	J.; Tallent, N.R.
	\newblock HPCTOOLKIT: Tools for performance analysis of optimized parallel
	programs.
	\newblock {\em Concurr. Comput. Pract. Exp.} {\bf 2010},
	{\em 22},~685--701.
	\newblock [\href{http://dx.doi.org/10.1002/cpe.1553}{CrossRef}]
	
	
	\bibitem[Miller(1995)]{Miller1995}
	Miller, B.; Callaghan, M.; Cargille, J.; Hollingsworth, J.; Irvin, R.;
	Karavanic, K.; Kunchithapadam, K.; Newhall, T.
	\newblock The Paradyn parallel performance measurement tool.
	\newblock {\em Computer} {\bf 1995}, {\em 28},~37--46.
	\newblock [\href{http://dx.doi.org/10.1109/2.471178}{CrossRef}]
	
	
	\bibitem[Galobardes(2015)]{Galobardes2015}
	Galobardes, E.C.
	\newblock \emph{Automatic Tuning of HPC Applications. The Periscope Tuning Framework};
	\newblock Shaker: Herzogenrath, Germany, 2015.
	
	\bibitem[Pillet(2007)]{Pillet2007}
	Pillet, V.; Labarta, J.; Cortes, T.; Girona, S.
	\newblock PARAVER: A Tool to Visualize and Analyze Parallel Code. In Proceedings of the WoTUG-18: Transputer and Occam Developments, Manchester, UK, 9--13 April 2007.
	
	
	\bibitem[Brink(2020)]{Brink2020}
	Brink, S.; Lumsden, I.; Scully-Allison, C.; Williams, K.; Pearce, O.; Gamblin,
	T.; Taufer, M.; Isaacs, K.E.; Bhatele, A.
	\newblock Usability and Performance Improvements in Hatchet.
	\newblock In Proceedings of the IEEE/ACM International Workshop on HPC User Support Tools
	(HUST) and Workshop on Programming and Performance Visualization Tools
	(ProTools), Atlanta, GA, USA, 18 November 2020; pp. 49--58.
	\newblock [\href{http://dx.doi.org/10.1109/HUSTProtools51951.2020.00013}{CrossRef}]
	
	
	\bibitem[Silva(2018)]{Silva2018}
	Silva, A.B.N.; Cunha, D.A.M.; Silva, V.R.G.; Furtunato, A.F.A.; Souza, S.X.-d.-S.
	\newblock PaScal Viewer: A Tool for the Visualization of Parallel Scalability
	Trends.
	\newblock In Proceedings of the ESPT/VPA@SC, Dallas, TX, USA, 11--16 November 2018.
	
	\bibitem[Eriksson(2016)]{Eriksson2016}
	Eriksson, J.; Ojeda-may, P.; Ponweiser, T.; Steinreiter, T.
	\newblock \emph{Profiling and Tracing Tools for Performance Analysis of Large Scale Applications};
	\newblock PRACE---Partnership for Advanced Computing in Europe: Brussels, Belgium, 2016; 
	pp. 1--30.
	\newblock Available online: \url{https://prace-ri.eu/wp-content/uploads/WP237.pdf} (accessed on 12 January 2020).
	
	
	\bibitem[Roberts(2017)]{10.1007/978-3-319-58667-0_22}
	Roberts, S.I.; Wright, S.A.; Fahmy, S.A.; Jarvis, S.A.
	\newblock Metrics for Energy-Aware Software Optimisation.
	\newblock In \emph{High Performance Computing, Proceedings of the 32nd International Conference, ISC High Performance 2017, Frankfurt, Germany, 18--22 June 2017}; Kunkel, J.M., Yokota, R., Balaji, P.,
	Keyes, D., Eds.; Springer International Publishing: Cham, Switzerland, 2017; pp.
	413--430.
	
	\bibitem[Eastep(2017)]{10.1007/978-3-319-58667-0_21}
	Eastep, J.; Sylvester, S.; Cantalupo, C.; Geltz, B.; Ardanaz, F.; Al-Rawi, A.;
	Livingston, K.; Keceli, F.; Maiterth, M.; Jana, S.
	\newblock Global Extensible Open Power Manager: A Vehicle for HPC Community
	Collaboration on Co-Designed Energy Management Solutions.
	\newblock In \emph{High Performance Computing, Proceedings of the 32nd International Conference, ISC High Performance 2017, Frankfurt, Germany, 18--22 June 2017}; Kunkel, J.M., Yokota, R., Balaji, P.,
	Keyes, D., Eds.; Springer International Publishing: Cham, Switzerland, 2017; pp.
	394--412.
	
	\bibitem[Hackenberg(2014)]{7016382}
	Hackenberg, D.; Ilsche, T.; Schuchart, J.; Schöne, R.; Nagel, W.E.; Simon, M.;
	Georgiou, Y.
	\newblock HDEEM: High Definition Energy Efficiency Monitoring.
	\newblock In Proceedings of the Energy Efficient Supercomputing Workshop, New Orleans, LA, USA, 16 November 2014; pp. 1--10.
	\newblock [\href{http://dx.doi.org/10.1109/E2SC.2014.13}{CrossRef}]
	
	
	\bibitem[Roberts(2019)]{10.1145/3321551}
	Roberts, S.I.; Wright, S.A.; Fahmy, S.A.; Jarvis, S.A.
	\newblock The Power-Optimised Software Envelope.
	\newblock {\em ACM Trans. Archit. Code Optim.} {\bf 2019}, {\em 16}, 1--27.
	\newblock [\href{http://dx.doi.org/10.1145/3321551}{CrossRef}]
	
	
	\bibitem[Boehme(2016)]{Boehme2016}
	Boehme, D.; Gamblin, T.; Beckingsale, D.; Bremer, P.T.; Gimenez, A.; LeGendre,
	M.; Pearce, O.; Schulz, M.
	\newblock Caliper: Performance Introspection for HPC Software Stacks.
	\newblock In Proceedings of the SC '16: Proceedings of the International Conference for High
	Performance Computing, Networking, Storage and Analysis, Salt Lake City, UT, USA, 13--18 November 2016; pp. 550--560.
	\newblock [\href{http://dx.doi.org/10.1109/SC.2016.46}{CrossRef}]
	
	
	\bibitem[Intel(2021)]{Intel2021Vtune}
	Corporation, I.
	\newblock \text{Intel VTune}.
	\newblock Available online: \url{https://software.intel.com/vtune} (accessed on 15 February 2020). 
	
	
	\bibitem[Houstis(1997)]{Pantazopoulos1997}
	Pantazopoulos, K.N.; Houstis, E.
	\newblock {Performance Analysis and Visualization Tools for Parallel Computing}.
	\newblock 1997.
	\newblock Available online: \url{https://docs.lib.purdue.edu/cstech/1346} (accessed on 20 May 2020).
	%MDPI: Please add the publisher and location.
	
	
	\bibitem[Ott(2010)]{Gerndt2010}
	Gerndt, M.; Ott, M.
	\newblock Automatic Performance Analysis with Periscope.
	\newblock {\em Concurr. Comput. Pract. Exp.} {\bf
		2010}, {\em 22},~736--748. [\href{http://dx.doi.org/10.1002/cpe.1551}{CrossRef}]
	
	\bibitem[Labarta(2005)]{Labarta2005}
	Labarta, J.; Gimenez, J.; Martínez, E.; González, P.; Servat, H.; Llort, G.;
	Aguilar, X.
	\newblock Scalability of Tracing and Visualization Tools. In Proceedings of the {International Conference ParCo}, Prague, Czech Republic, 10--13 September 2005;
	\newblock pp. 869--876.
	
	\bibitem[Bienia(2008)]{Bienia2008}
	Bienia, C.; Kumar, S.; Singh, J.P.; Li, K.
	\newblock {The PARSEC benchmark suite: Characterization and architectural
		implications}.
	\newblock {In Proceedings of the International Conference on Parallel
		Architectures and Compilation Techniques}, Toronto, ON, Canada, {25--29 October 2008}; pp. 72--81.
	\newblock [\href{http://dx.doi.org/10.1145/1454115.1454128}{CrossRef}]
	
	
	\bibitem[Alex(2020)]{Alex2020WhenParallelSpeedups}
	\newblock Furtunato, A.F.A.; Georgiou, K.; Eder, K.; Xavier-De-Souza, S.
	\newblock When Parallel Speedups Hit the Memory Wall.
	\newblock {\em IEEE Access} {\bf 2020}, {\em  8},  79225--79238. [\href{http://dx.doi.org/10.1109/ACCESS.2020.2990418}{CrossRef}]
	
	\bibitem[Vitor(2020)]{Vitor2022AnalyticalEnergyModel}
	\newblock Silva, V.R.G.; Valderrama, C.; Manneback, P.; Xavier-de-Souza, S.
	\newblock Analytical Energy Model Parametrized by Workload, Clock Frequency and Number of Active Cores for Share-Memory High-Performance Computing Applications.
	\newblock {\em  Energies} {\bf 2022}, {\em  15},  1213. [\href{http://dx.doi.org/10.3390/en15031213}{CrossRef}]

	%%%%%%

	\bibitem[Melo(2010)]{Melo2010Perf}
	A.~C. de~Melo, ``{The New Linux 'perf' Tools},'' {\em Linux Kongress}, 2010.
	
	\bibitem[Gonzalez(2021)]{Gonzalez2021PolyBench}Abella-Gonzalez, M., Carollo-Fernandez, P., Pouchet, L., Rastello, F. \& Rodrıiguez, G. PolyBench/Python: Benchmarking Python Environments with Polyhedral Optimizations. {\em Proceedings Of The 30th ACM SIGPLAN International Conference On Compiler Construction}. pp. 59-70 (2021)

	\bibitem[Nikolaev(2011)]{Nikolaev2011Perfctr}Nikolaev, R. \& Back, G. Perfctr-Xen: A Framework for Performance Counter Virtualization. {\em Proceedings Of The 7th ACM SIGPLAN/SIGOPS International Conference On Virtual Execution Environments}. pp. 15-26 (2011)


\end{thebibliography}
	
	\backmatter
	
\end{document}