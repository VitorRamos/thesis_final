
\section{Theoretical Background} \label{sec:theoretical_background}

A model is a formal representation of a natural system. The representation of computer system models includes equations, graphical models, rules, decision trees, representative collections of examples, and neural networks. The choice of representation affects the model's accuracy, as well as its interpretability by people~\cite{Hypothesis2012EncyclopediaLearning, Roy2019ForecastingNetwork, Zhu2019PredictingLearning}. Accurate energy and power consumption models are essential for many energy efficiency schemes employed in computing equipment \cite{Rivoire2007ModelsOptimizations}, and they can have multiple uses, including the design, forecasting, and optimization of data center systems. This study focuses on analytical models that could aid energy optimization and analyses of crucial factors in the total energy draw.

The desirable properties of a full-system model of energy consumption include accuracy, speed, generality and portability, inexpensiveness, and simplicity \cite{Rivoire2008AModels}. However, modeling an HPC system's exact energy consumption behavior is not straightforward, either at the whole-system level or at the level of individual components. Data centers' energy consumption patterns depend on multiple factors, such as hardware specifications, workload, cooling requirements, or the type of the applications. Some of these factors cannot be measured easily. Furthermore, it is impractical to perform detailed measurements of the energy consumption of lower-level components without additional overhead.

Several proposed models have already been classified concerning their input parameters, as shown by Dayarathna et al.~\cite{Dayarathna2016DataSurvey}, who analyzed more than 200 models according to their characteristics and limitations and classified them into categories where the model is more suited to its objectives:
\begin{itemize}
	%\item Temperature
	\item System utilization or workload
	\item Frequency
	\item Other system states, such as cache miss, branch prediction, number of instructions executed, and more
	\label{tab:input_type}
\end{itemize}

Often, energy models are described as a combination of two main parts, the power model of the system and the performance model of the application. This is because the concept of energy (E) is the total amount of work performed by a system over a period of time (T), while power (P) is the rate at which the system performs the work. The relation between these three amounts can be expressed as:
\begin{equation}
E = \int_{0}^{T}P(t)dt.
\label{eq:energy_definition_cont}
\end{equation}

\subsection{Power Models}

The modeling of system parameters is becoming popular nowadays with the advantage of performance counters provided by the CPU or the operating system. These counters can measure micro-architectural events, such as instructions executed, cache hits, miss-predicted branches, and more; thus, providing a base for many different estimations of power usage. This makes this type of model very suitable for power estimation because it can use information about several internal states of the computer.

Frequency-based models are the most common kind of model. They serve as a base for many power models~\cite{Sarwar1997CmosCalculation, Butzen2007LeakageGates, Usman2013ANoC}. These models utilize the fact that every digital circuit (including modern processors) is composed of transistors. Thus, modeling one transistor's interaction and scaling this to the chip can give a reasonable estimate of the entire system's energy. One of the most common frequency-based %please confirm that the intended meaning has been retained.
model approximations is defined as follows: 
\begin{equation}
P = \alpha+\beta f^3,
\label{eq:power_simplified}
\end{equation}
where $\alpha$ and $\beta$ are model parameters, and $f$ is the operating frequency (details of this equation are covered in (\cref{chapter:models}). This type of model is suitable for optimization problems since these are a function of the operating frequency, which can be easily controlled.

% We confirm that the meaning has been retained

\subsection{Performance Models}
The most common way to model the application performance is using the workload. The workload is an abstract representation of the amount of work done for a given time and speed. The workload ($W$) can be defined in many different ways. One common way, used in many studies, such as Paolillo et al. ~\cite{Paolillo2018OptimisationParallelism}, Francis et al. \cite{ Group2012HandbookSahni}, and Kim et al. \cite{Kim2015RacingHeuristics}, is the following:
\begin{equation}
W = \int_{0}^{\tau}s(t)dt = s\tau,
\label{eq:workload_definition}
\end{equation}
where $\tau$ is total active time, and $s$ is the execution speed in instructions/second.

Utilization models \cite{Fu2018RaceMinimization, Group2012HandbookSahni} are also found in the literature, defined as the ratio between the time that the system is active and the total time (idle and active). These models are present in many DVFS algorithms present in Linux. They can be viewed as a good alternative to the workload since it is impossible to measure workload in real-time.  \cref{eq:utilization_definition} defines workload in terms of CPU utilization ($u$):
\begin{equation}
u = \frac{\tau}{T} = \frac{W/s}{T},
\label{eq:utilization_definition}
\end{equation}
where $T$ is the total execution time (idle and active), and $\tau$ is the active time, meaning when the processor was executing instructions.
Models based on CPU utilization are the basis for DVFS algorithms. Even though this is not a controllable parameter, it is straightforward to measure system utilization with almost no overhead, and it is also very portable in terms of operating systems and architectures.

\section{Related work} \label{sec:related_work}

Merkel et al.~\cite{Merkel2006BalancingSystems} developed an energy model for processors based on events. Their model assumes a fixed energy consumption $\alpha_i$ for each activity, and by counting the number of occurrences $c_i$ of every activity, they estimate the total energy as:
\begin{equation}
E = \sum_{i=1}^{n}\alpha_ic_i.
\label{equation:rw_merkel}
\end{equation}

Another event-based model, introduced by Roy et al. \cite{Roy2013AnAlgorithms}, described the computational energy consumed by a CPU for an algorithm $A$ as the Equation (\ref{equation:rw_event_based})%\cref{equation:rw_event_based}:
\begin{equation}
E(A) = P_{clk}T(A) + P_wW(A),
\label{equation:rw_event_based}
\end{equation}
where $P_{clk}$ is a processor clock leakage power, $T(A)$ is the total execution time, $W(A)$ is the total time taken by  non-I/O operations, and $P_w$ is used to capture the power consumption per operation performed by the CPU. $T(A)$ and $W(A)$ are estimated using performance features.

Models based on events present some drawbacks, they are highly dependent on the operating system and its architecture, making them problematic to port for other platforms. There are also limitations regarding the number of simultaneous events that can coexist without adding a non-negligible overhead. Additionally, there are cases where events need multiplexing, for example, when using more hardware events than %please confirm that the intended meaning has been retained.
the CPU can provide. There are also some well know problems regarding the precision of some events, as shown in many studies~\cite{Weaver2008CanTrusted, Weaver2013Non-determinismImplementations, Das2019SoK:Security, McGuire2009AnalysisKernel, Ramos2019AnCounters, Silva-de-Souza2020ContainergyAWorkloads}. Some events that should be exact and deterministic (such as the number of executed instructions) show run-to-run variations and over-count on various architectures, even when running in strictly controlled environments. Because of that, our proposed model is not dependent on events and, therefore, not vulnerable to those drawbacks.
% We confirm that the meaning is the same

An instruction-level energy model was also proposed in \cite{Shao2013EnergyProcessor} by Yakun et. al. Where they proposed an energy per instruction (EPI) characterization made on Xeon Phi. Their model is expressed as:
\begin{equation}
E(f) = \frac{(p_1 - p_0)(c_1 - c_0)/f}{N}, 
\label{eq:rw_instruction_level}
\end{equation}
where $N$ is the total number of dynamic instructions, $p_0$ is the initial idle power, $p_1$ is the average dynamic power, and ($c_1$ $-$ $c_0$) refers to the cumulative number of cycles the micro-benchmark performs. This model is suitable for estimating the energy after the application finishes executing when it is possible to count the total cycles. However, it is challenging to use for optimization or forecasting since it does not have an application model to predict the cycles. Our model integrates the behavior of the application, taking into account the execution time.


Lewis et al.~\cite{Lewis2008Run-timeSystems} described the overall system energy consumption using the following equation:
\begin{equation}
E = A_0(E_{proc} + E_{mem}) + A_1E_{em} + A_2E_{board} + A_3E_{hdd},
\label{eq:lewvis}
\end{equation}
where, $A_0$, $A_1$, $A_2$, and $A_3$ are unknown constants that are calculated via linear regression analysis and those remain constant for a specific server architecture. This model, as the previous one, relies on knowledge of energy spent on each component, being  a suitable option for estimation after the application has already run, but not for optimization of the run itself, which is the aim of our model.

In another energy consumption model  based on system utilization, Mills et al. \cite{Mills2014EnergySystems} modeled the energy consumed by a compute node with CPU (single) executing at speed $\sigma$ as Equation (\ref{eq:mills})%\cref{eq:mills},
\begin{equation}
E(\sigma,[t_1,t_2]) = \int_{t_1}^{t_2} \sigma^3 + \rho \sigma_max^3 dt,
\label{eq:mills}
\end{equation}
where $\rho$ stands for the overhead power consumed regardless of the processor speed, $t_1$ and $t_2$ are the application's initial and final execution times. The overhead includes power consumption by all other system components, such as memory, network, and more. For this reason, although the authors mentioned the energy consumption of a socket, their power model is generalized to the entire server. This model lacks a closed-form, i.e., it depends on the definition of $\alpha(t)$ to be complete. Our model has a closed-form which facilitates analyses.

\section{Power Model} \label{sec:power_model}

The developed power model is based on the developed frequency models \cite{Rauber2014EnergyScaling, Goel2016AProcessors, Du2017ModelingSystems, Gonzalez1997SupplyCMOS}. In this approach, the idea is to reduce the complexity of the processor dynamics by looking only at the main element that it is composed of, the transistor. Thus, modeling the power consumption can be resumed to model the logic gates and multiply this by the total number of gates, reducing the complexity of the modeling process.

FINFET and MOSFET comprise the main techniques to manufacture transistors. However, FINFET is the more recent,  and has gradually replaced the mature technology MOSFET. Despite having different characteristics, they have aspects in common that can be modeled  \cite{Rauber2014EnergyScaling, Goel2016AProcessors, Du2017ModelingSystems, Gonzalez1997SupplyCMOS}. These are static power $P_{\rm static}$, dynamic power $P_{\rm dynamic}$, and leakage power $P_{\rm leak}$, which, in combination, comprise  and approximate the total power draw.

The dynamic power and leakage power behavior can be approximated by the following equations, respectively, as shown by Sarwar et al.~\cite{Sarwar1997CmosCalculation} and Butzen et al.~\cite{Butzen2007LeakageGates}.
\begin{equation}
	P_{dynamic}=CV^2f,
	\label{eq:power_dyn}
\end{equation}
\begin{equation}
	P_{leak} \underset{\sim}{\propto} V,
	\label{eq:power_leak}
\end{equation}
where $C$ is the load capacitance, $V$ is the voltage applied to the circuit, and $f$ is the switching frequency.

Another common approximation is to assume a linear relationship between the voltage and the applied frequency~\cite{Usman2013ANoC}, such that:
\begin{equation}
	f \underset{\sim}{\propto} V,
	\label{eq:f_v}
\end{equation}

These approximations have been demonstrated to be very precise. In the work of Silva et al., the mean percentage error was calculated to be 0.75\% \cite{Silva2019Energy-OptimalApplications}.

Thus, the proposed model for one processing core of a multi-core processor is derived by using Equations (10)--(12) to write \cref{eq:total_power}.
\begin{equation}
	P(f)= c_1f^3+c_2f+c_3,
	\label{eq:total_power}
\end{equation}
where $c_1$ $c_2$, and $c_3$ are the model's parameters associated with the dynamic, leakage and static power aspects, respectively. Including the number of active cores $p$, the proposed estimation of the power consumption of the whole processor becomes \cref{eq:power_final}
\begin{equation}
	P(f,p)= p(c_1f^3+c_2f)+c_3,
	\label{eq:power_final}
\end{equation}

\section{Performance Model} \label{sec:performance_model}
We consider a program as a set of instructions executed on a mean frequency $f$ with $c_k$ instructions per cycle to model the application execution time. The time $T_f$ that this program will take to complete at a given frequency is devised as follows:
\begin{equation}
	T_f=\frac{I}{c_kf},
	\label{eq:freqrel}
\end{equation}
where $I$ is the total number of instructions and $c_k$ the ratio of instructions per unit of time.

The next step is to include the number of cores in the equation. Amdahl's law~\cite{Amdahl1967ValidityCapabilities}, gives the theoretical background for that. It describes the speedup in latency of the execution of a task at a fixed workload.
\begin{equation}
	S=\frac{T_s}{T_p}=\frac{1}{1-w+\frac{w}{p}},
	\label{eq:amdahl}
\end{equation}
where $T_s$ is the serial time, $T_p$ the parallel time, $S$ is the theoretical speedup of the execution of the whole task, $w$ is the proportion of the execution time that benefits from improving system resources, and $p$ is the speedup part of the task that benefits from improved system resources. Combining this with \cref{eq:freqrel}, the parallel time at frequency $f$ can be written as:
\begin{equation}
	T_p=\frac{T_s}{S}=\frac{T_f}{\frac{1}{1-w+\frac{w}{p}}},
	\label{eq:parallel_time}
\end{equation}

We can then write the equation of the program execution time as a function of frequency, the number of cores, and parallelism  as \cref{eq:performance} and subsequently derive \cref{eq:performance_2}:
\begin{equation}
	T(f,p)=\frac{I}{ \frac{c_kf}{1-w+\frac{w}{p}} },
	\label{eq:performance}
\end{equation}
\begin{equation}
	T(f,p)=\frac{d_1(p-wp+w)}{fp},
	\label{eq:performance_2}
\end{equation}
where $d_1$ is a constant.

Finally, to fully characterize the application, a parameter representing the application's workload, called input size $N$, is introduced, representing the number of basic operations needed to complete a problem \cite{Kumar1994AnalyzingArchitectures}. In Oliveira et al. \cite{Oliveira2018ApplicationCharacterization}, they showed that this parameter could generally be described as exponential. Therefore the proposed performance model is presented in \cref{eq:performance_final}. This resulting equation describes the behavior of the execution time of a program for an input $N$, frequency $f$, and active cores $p$:
\begin{equation}
	T(f,p,N)=\frac{d_1N^{d_2}(p-wp+w)}{fp},
	\label{eq:performance_final}
\end{equation}
where $d_1$, $d_2$ and $w$ are constants that depend on the application.

\section{Energy Model} \label{sec:energy_model}
Combining the power model output described in~Section \ref{sec:power_model} and the characterization of the application performance described in Section \ref{sec:performance_model}, the total energy can be modeled as:
\begin{equation}
	E(f,p,N)=P(f,p)\times{\rm T}(f,p,N),
	\label{eq:en_combination}
\end{equation}
where $P(f,p)$ is the total power modeled by~\cref{eq:power_final}, ${T}(f,p,N)$ is the execution time estimated by the \cref{eq:performance_final}, $f$ is the frequency, $p$ is the number of active cores, and $N$ is the input size. The final equation can be written as:
\begin{equation}
	E(f,p,N)=\frac{d_1N^{d_2}(p-wp+w)(p(c_1f^3+c_2f)+c_3)}{fp}.
	\label{eq:en_final}
\end{equation}