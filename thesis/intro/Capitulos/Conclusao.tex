% Considerações finais
\chapter{Conclusão}

Neste trabalho, foi proposta uma nova abordagem para otimizar a eficiência de energia de aplicativos HPC em lote de nó único. Em contraste com os algoritmos de DVFS existentes, nossa técnica utiliza o perfil de tempo de execução do aplicativo e um modelo de energia do nó de cálculo para prever a frequência ideal e o número de núcleos a serem usados. Isso se mostrou eficaz na redução do consumo de energia das aplicações, possivelmente devido ao fato de que o uso do conhecimento prévio do desempenho do aplicativo na arquitetura de destino pode expor informações suficientemente relevantes, como acelerações paralelas, que é mais difícil de encontrar em técnicas de tempo de execução baseadas em DVFS.

O uso de uma modelagem de energia agnóstica do aplicativo para a arquitetura de destino ajuda a tornar a técnica portátil para outros aplicativos. Ou seja, para estimar a frequência ótima de energia e o número de núcleos ativos para uma nova aplicação, apenas uma caracterização de desempenho é necessária.

As principais fraquezas da técnica proposta são, a necessidade de informações sobre o tamanho de entrada do aplicativo antes da execução, o modelo de potencia não leva em consideração a variação ao longo do tempo e o treinamento da SVR consume muita energia.

Para trabalhos futuros, possíveis soluções para os problemas encontrados são, utilizar contadores de desempenho, presentes em todos os processadores HPC modernos, para descobrir o tamanho da entrada com base em dados previamente coletados, incluir a porcentagem de utilização do processador no modelo de potência e encontrar uma formulação matemática para o modelo de desempenho que além de resolver o problema do consumo de energia no treinamento, ainda simplifica a forma de encontrar o mínimo global, podendo ser calculado analiticamente.