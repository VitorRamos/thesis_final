% Introdução
\chapter{Introdução} \label{cap:introducao}

Consumo de energia é uma preocupação na computação moderna, tanto em dispositivos pessoais, smartphones e notebooks, visando economia da bateria do dispositivo, como em grandes servidores, nuvens e sistemas computacionais de alto desempenho devido aos custos de manutenção e impacto ambiental que eles podem gerar.

Em computação de alto desempenho \abrv[HPC --High Performance Computing]{(HPC)} isso é mais complicado, pois, facilmente um sistema chega a consumir centenas de KiloWatts e sistemas com maior poder computacional podem chegar até a dezenas de MegaWatts \cite{Top500}. Isso pode inviabilizar projetos ou até mesmo descontinuar trabalhos existentes por falta de recursos. Além disso, a próxima geração de sistemas HPC com capacidade de processamento de exaFLOPS, isto é, bilhões de bilhões de cálculos por segundo, terão seu uso restrito pelo seu consumo de energia \cite{Alfonso2013}.

Investir em novas maneiras de reduzir o consumo de potência é crucial para esses sistemas se tornarem viáveis economicamente, além da redução do impacto ambiental causado pelo alto consumo deles. Há muitas pesquisas nesta área, e portanto, diversas formas de reduzir o consumo de energia. Uma das principais é através do consumo de energia do processador. Pesquisas do Google \cite{Fan2007} mostraram que ele é o principal componente responsável pelo consumo de energia em servidores, chegando no pico a 57\% da energia total gasta pelo sistema \cite{Barroso2007}.

Por isso, processadores atuais incluem diversos componentes para minimizar a potência utilizada, como, unidades independentes que podem ser desabilitadas \cite{Rotem2012}, clock gating: técnica utilizada em circuitos síncronos para diminuir o consumo de energia \cite{Srinivasan2015}, Dynamic Voltage and Frequency Scaling \cite{Mittal2014} \abrv[DVFS -- Dynamic Voltage and Frequency Scaling]{(DVFS)}, entre outras.

A técnica de DVFS \cite{Hackenberg2015, Dzhagaryan2014, Hahnel2012, Basmadjian2012, Travers2015, Miyoshi2002, Anghel2011, Pietri2014, Spiga2006} é umas das mais eficientes, pois, o consumo do processador está altamente correlacionado com a sua frequência de operação. Esta consiste em controlar a frequência e a voltagem de acordo com a utilização do dispositivo. Este controle pode ser feito tanto em software através do sistema operacional, quanto em hardware.

Este trabalho propõe uma metodologia para encontrar a frequência operacional e o número de núcleos ativos que minimizam a energia total usada para executar uma aplicação HPC em um único nó com memória compartilhada. Essa metodologia é diferente dos existentes no Linux, pois utiliza uma modelagem do consumo de potência junto com uma caracterização do tempo de execução da aplicação para melhor encontrar a configuração que minimiza a energia total utilizada.

O modelo de potência baseia-se na modelagem da potência do circuito lógico CMOS em função da frequência de operação \cite{Sarwar1997}, incluindo a potência dinâmica, estática e de fuga. Além da frequência de operação, o modelo de energia também é parametrizado para o número de soquetes ativos e de núcleos ativos por soquete.

A modelagem do desempenho é feita pela caracterização da aplicação na arquitetura de destino. A ideia é prever o desempenho da aplicação em qualquer configuração específica. O modelo utiliza a frequência de operação, o número de núcleos ativos e o tamanho da entrada. Essa modelagem utiliza o método de aprendizado supervisionado para regressão chamado Support Vector Regression \abrv[SVR -- Support Vector Regression]{(SVR)} \cite{Ventura2009, Smola2004}.

Para encontrar as configurações de energia ideal, o algoritmo minimiza o produto dos resultados dos modelos de potência e desempenho. Essa abordagem foi validada em quatro aplicações do PARSEC, que é um benchmark para arquiteturas paralelas, \cite{Bienia2008} e comparada a politica \emph{Ondemand}, que é o esquema padrão do DVFS para o sistema operacional Linux e a politica do P-state que é o padrão nos processadores da Intel. Os resultados mostram que a abordagem proposta foi capaz de encontrar configurações que consumiram até 30 \% menos energia quando comparado aos métodos citados anteriormente. A média geral de economia de energia foi de 5,5\%.

%\section{Organização do trabalho}

%No capitulo \ref{cap:referencial_teorico} será dado um referencial teórico das técnicas utilizadas neste trabalho será discutido sobre o controle de frequência na seção \ref{sec:controle_freq} e também sobre o monitoramento de energia nos sistemas em \ref{sec:monitoramento}. No próximo capitulo \ref{cap:modelos} será abordado os modelos de potencia \ref{sec:potencia}, performance  \ref{sec:performance} e energia \ref{sec:energia} detalhando seus desenvolvimentos. A metodologia utilizada para obter os ajustes dos modelos e encontrasse no capitulo \ref{cap:metodologia} junto com a configuração utilizada \ref{sec:config} e as aplicações \ref{sec:apps}. Os resultados estão no capitulo \ref{cap:resultados} onde é mostrado o ajuste dos modelos \ref{sec:ajuste_pw}, \ref{sec:ajuste_perf} e \ref{sec:ajuste_en}. E também onde é feita a comparação \ref{sec:comp}.