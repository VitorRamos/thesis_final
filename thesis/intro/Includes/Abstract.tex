% Resumo em língua estrangeira (em inglês Abstract, em espanhol Resumen, em francês Résumé)
\begin{center}
	{\Large{\textbf{Energy-Optimal Configurations for high performance computer applications}}}
\end{center}

\vspace{1cm}

\begin{flushright}
	Author: Vitor Ramos Gomes da Silva\\
	Advisor: Prof. Dr. Samuel Xavier de Souza
\end{flushright}

\vspace{1cm}

\begin{center}
	\Large{\textsc{\textbf{Abstract}}}
\end{center}

\noindent This work proposes a methodology to minimize energy consumption
in high-performance computing by finding the ideal frequency and active cores parameters in the processor. Based on a model of the power draw of the system, developed based on the power consumption of the CMOS circuit, and an estimation of the execution time of the application using artificial intelligence is calculated the energy equation with frequency, number of active cores and input of the application as variables and from this equation is found the configuration that minimizes the total power consumption. The results of this method in four applications show that the proposed approach achieved savings of more than 30 \% when compared to the current energy saving scheme in Linux.

\noindent\textit{Keywords}: 
Energy Efficient Software, Energy Modeling, Frequency Scaling.