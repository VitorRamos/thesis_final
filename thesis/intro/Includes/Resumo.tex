% Resumo em língua vernácula
\begin{center}
	{\Large{\textbf{Configurações otimizadas de energia para aplicações em computação de alto desempenho}}}
\end{center}

\vspace{1cm}

\begin{flushright}
	Autor: Vitor Ramos Gomes da Silva\
	Orientador: Prof. Dr. Samuel Xavier de Souza
\end{flushright}

\vspace{1cm}

\begin{center}
	\Large{\textsc{\textbf{Resumo}}}
\end{center}

\noindent 
Este trabalho propõe uma metodologia para minimizar o consumo de energia 
em computação de alta performance, encontrando os parâmetros ideais de frequência e núcleos ativos no processador. A partir de uma modelagem da potência do sistema, desenvolvida com base no consumo de potência do circuito CMOS e uma estimativa do tempo de execução da aplicação utilizando inteligência artificial, é calculada a equação da energia em função da frequência, número de núcleos e entrada da aplicação. Com esta equação, é encontrada a configuração que minimiza o consumo total de energia. Foram feitos testes em quatro aplicações e os resultados desse método mostraram que essa abordagem chegou à economia de até 30\% quando comparado ao esquema de economia de energia existente atualmente no Linux.


\noindent\textit{Palavras-chave}: Software energeticamente eficiente, Modelagem do consumo de potência, Escalonamento de frequência