\section{Common approaches to the optimization problems}

To build the optimization problem it's necessary to model the system power and application, and controllable variables. Common approaches to the optimization problems are:

Minimize the energy function in function of our control variables, subject to constraints in the control variable.

\begin{equation}
\begin{aligned}
\textrm{min} \quad & P(s_1, s_2, ...)T(s_1,s_2,...)\\
\textrm{subject to} \quad & b_1<s_1<b_2\\
\quad & b_3<s_2<b_4\\
\quad & \vdots\\
\end{aligned}
\end{equation}

Another way is to minimize the total energy, with the constraint of to finish the work.

\begin{equation}
\begin{aligned}
\textrm{min} \quad & \sum{P_it_i}\\
\textrm{subject to} \quad & W_{tot} = \sum w_i\\
\quad & \vdots\\
\end{aligned}
\end{equation}

This kind of problem can be seen in multiple ways, considering an application as the total workload and choosing different speeds for each phase of the application, or could also treat each workload as a different application and create schedulers both CPU and cluster level. In fact the question of on each level of optimizer produces a better result is not well explored in the literature yet. The habitual approach it's to tackle each problem at once and combine the strategies resulting in a chain of schedulers.

The complexity of this problem also varies, depending on the choice of power function it can result in linear programming \cite{Kim2015RacingHeuristics}, quadratic programming \cite{Horyath2008Multi-mode}, until NP-hard problems \cite{Fu2018RaceMinimization}.