\begin{thebibliography}{999}
	
	\bibitem[Ishfag(2012)]{Group2012HandbookSahni}
	Ishfag, A.; Sanjay, R. \textit{Handbook of Energy-Aware and Green Computing}; Chapman \& Hall/CRC: London, England, 2012; Volume~1, pp. 702--713.
	
	\bibitem[Dayarathna(2016)]{Dayarathna2016DataSurvey}
	Dayarathna, M.; Wen, Y.; Fan, R. Data Center Energy Consumption Modeling: A Survey. {\em IEEE Commun. Surv. Tutor.} {\bf 2016}, {\em 18}, 732--794. [\href{http://doi.org/10.1109/COMST.2015.2481183}{CrossRef}]
	
	\bibitem[Corcoran(2017)]{Corcoran2017EmergingICT}
	Corcoran, P.; Andrae, A. \emph{Emerging Trends in Electricity Consumption for Consumer ICT}; National University of Ireland: Galway, Ireland, 2013; pp. 1--56.
	
	\bibitem[Mathew(2012)]{Mathew2012Energy-awareNetworks}
	Mathew, V.; Sitaraman, R. K.; Shenoy, P. Energy-aware load balancing in content delivery networks. {In Proceedings of the 2012 Proceedings IEEE INFOCOM, Orlando, FL, USA, 25--30 March 2012}; pp. 954--962.
	
	\bibitem[Rivoire(2007)]{Rivoire2007ModelsOptimizations}
	Rivoire, S.; Shah, M.A.; Ranganathan, P.; Kozyrakis, C.; Meza, J. Models and Metrics to Enable Energy-Efficiency Optimizations. {\em Computer} {\bf 2007}, {\em 40}, 39--48. [\href{http://dx.doi.org/10.1109/MC.2007.436}{CrossRef}]
	
	\bibitem[Buyya(2013)]{Buyya2013Introduction}
	Buyya, R.; Vecchiola, C.; Selvi, S.T. \textit{Mastering Cloud Computing}; Morgan Kaufmann Publishers Inc.: San Francisco, CA, USA, 2013; pp. 3--27
	
	\bibitem[Poess(2008)]{Poess2008EnergyCenters}
	Poess, M.; Nambiar, R.O. Energy cost, the key challenge of today's data centers. {\em Proc. VLDB Endow.} {\bf 2008}, {\em 1}, 1229--1240. [\href{http://dx.doi.org/10.14778/1454159.1454162}{CrossRef}]
	
	\bibitem[Gao(2013)]{Gao2013QualityCenters}
	Gao, Y.; Guan, H.; Qi, Z.; Wang, B.; Liu, L. Quality of service aware power management for virtualized data centers. {\em J. Syst. Archit.} {\bf 2013}, {\em 59}, 245--259. [\href{http://dx.doi.org/10.1016/j.sysarc.2013.03.007}{CrossRef}]
	
	\bibitem[Fan(2007)]{Fan2007PowerComputer}
	Fan, X.; Weber, W.D.; Barroso, L.A. Power provisioning for a warehouse-sized computer. {\em ACM SIGARCH Comput. Archit. News} {\bf 2007}, {\em 35}, 13--23. [\href{http://dx.doi.org/10.1145/1273440.1250665}{CrossRef}]
	
	\bibitem[Barroso(2007)]{Barroso2007TheComputing}
	Barroso, L.A.; H{\"{o}}lzle, U. The Case for Energy-Proportional Computing. {\em Computer} {\bf 2007}, {\em 40}, 33--37. [\href{http://dx.doi.org/10.1109/MC.2007.443}{CrossRef}]
	
	\bibitem[Malladi(2012)]{Malladi2012TowardsDRAM}
	Malladi, K.T.; Nothaft, F.A.; Periyathambi, K.; Lee, B.C.; Kozyrakis, C.; Horowitz, M. Towards energy-proportional datacenter memory with mobile DRAM. {In  Proceedings of the 2012 39th Annual International Symposium on Computer Architecture (ISCA), Portland, OR, USA, 9--13 June 2012}; pp. 37--48.
	
	\bibitem[Rotem(2012)]{Rotem2012Power-managementBridge}
	Rotem, E.; Naveh, A.; Ananthakrishnan, A.; Weissmann, E.; Rajwan, D. Power-Management Architecture of the Intel Microarchitecture Code-Named Sandy Bridge. {\em IEEE Micro} {\bf 2012}, {\em 32} 20--27. [\href{http://dx.doi.org/10.1109/MM.2012.12}{CrossRef}]
	
	\bibitem[Brown(2005)]{Brown2005ACPILinux}
	Brown, L.; Moore, R.; Li, D.S.; Yu, L.; Keshavamurthy, A.; Pallipadi, V. ACPI in Linux. {\em Symposium } {\bf 2005}, {\em 51}, 1--51.
	
	\bibitem[Hackenberg(2015)]{Hackenberg2015AnProcessor}
	Hackenberg, D.; Schone, R.; Ilsche, T.; Molka, D.; Schuchart, J.; Geyer, R. An Energy Efficiency Feature Survey of the Intel Haswell Processor. {In  Proceedings of the  2015 IEEE International Parallel and Distributed Processing Symposium Workshop, Hyderabad, India, 25--29 May 2015}; pp. 896--904.
	
	\bibitem[{Intel}(2020)]{Intel20200thLake}
	Intel. \textit{12th Generation Intel {\textregistered} Core™ Processors}; Intel: Santa Clara, CA, USA, 2020; pp. 420--430
	
	\bibitem[Shuja(2012)]{Shuja2012Energy-efficientCenters}
	Shuja, J.; Madani, S.A.; Bilal, K.; Hayat, K.; Khan, S.U.; Sarwar, S. Energy-efficient data centers. {\em Computing} {\bf 2012}, {\em 94}, 973--994. [\href{http://dx.doi.org/10.1007/s00607-012-0211-2}{CrossRef}]
	
	\bibitem[Benini(2000)]{Benini2000AManagement}
	Benini, L.; Bogliolo, A.; De~Micheli, G. A survey of design techniques for system-level dynamic power management. {\em IEEE Trans. Very Large Scale Integr. (VLSI) Syst.} {\bf 2000}, {\em 8}, 299--316. [\href{http://dx.doi.org/10.1109/92.845896}{CrossRef}]
	
	\bibitem[Merkel(2006)]{Merkel2006BalancingSystems}
	Merkel, A.; Bellosa, F. Balancing power consumption in multiprocessor systems. {\em ACM SIGOPS/EuroSys Eur. Conf. Comput. Syst.} {\bf 2006}, \emph{40}, 403--4014.
	
	\bibitem[Roy(2013)]{Roy2013AnAlgorithms}
	Roy, S.; Rudra, A.; Verma, A. An energy complexity model for algorithms. In Proceedings of the 4th conference on Innovations in Theoretical Computer Science, New York, NY, USA, 9--12 January 2013.
	
	\bibitem[Weaver(2008)]{Weaver2008CanTrusted}
	Weaver, V.M.; McKee, S.A. Can hardware performance counters be trusted? {In Proceedings of the 2008 IEEE International Symposium on Workload Characterization, Seattle, WA, USA, 14--16 September 2008}; pp. 141--150.
	
	\bibitem[Weaver(2013)]{Weaver2013Non-determinismImplementations}
	Weaver, V.M.; Terpstra, D.; Moore, S. Non-determinism and overcount on modern hardware performance counter implementations. {In Proceedings of the 2013 IEEE International Symposium on Performance Analysis of Systems and Software (ISPASS), Austin, TX, USA, 21--23 April 2013}; pp. 215--224.
	
	\bibitem[Das(2019)]{Das2019SoK:Security}
	Das, S.; Werner, J.; Antonakakis, M.; Polychronakis, M.; Monrose, F. SoK: The Challenges, Pitfalls, and Perils of Using Hardware Performance Counters for Security. {In Proceedings of the 2019 IEEE Symposium on Security and Privacy (SP), San Francisco, CA, USA, 19--23 May 2019}; pp. 20--38.
	
	\bibitem[Guire(2009)]{McGuire2009AnalysisKernel}
	Mc~Guire, N.; Okech, P.; Schiesser, G. Analysis of Inherent Randomness of the Linux Kernel. {In Proceedings of the Eleventh RealTime Linux Workshop, Dresden, Germany, 28–30 September 2009}.
	
	\bibitem[Ramos(2019)]{Ramos2019AnCounters}
	Ramos, V.; Valderrama, C.; Xavier~de Souza, S.; Manneback, P. An Accurate Tool for Modeling, Fingerprinting, Comparison, and Clustering of Parallel Applications Based on Performance Counters. {In Proceedings of the  IEEE International Parallel and Distributed Processing, Rio de Janeiro, Brazil, 20--24 May 2019}; pp. 797--804.
	
	\bibitem[Silva-de Souza(2020)]{Silva-de-Souza2020ContainergyAWorkloads}
	Silva-de Souza, W.; Iranfar, A.; Br{\'{a}}ulio, A.; Zapater, M.; Xavier-de Souza, S.; Olcoz, K.; Atienza, D. Containergy—A Container-Based Energy and Performance Profiling Tool for Next Generation Workloads. {\em Energies} {\bf 2020}, {\em 13}, ~2162. [\href{http://dx.doi.org/10.3390/en13092162}{CrossRef}]
	
	\bibitem[Shao and Brooks(2013)]{Shao2013EnergyProcessor}
	Shao, Y.S.; Brooks, D. Energy characterization and instruction-level energy model of Intel's Xeon Phi processor. In Proceedings of the  International Symposium on Low Power Electronics and Design (ISLPED), Beijing, China, 4--6 September 2013; pp. 389--394.
	
	\bibitem[Lewis(2008)]{Lewis2008Run-timeSystems}
	Lewis, A.; Ghosh, S.; Tzeng, N.F. Run-time energy consumption estimation based on workload in server systems. {In Proceedings of the 2008 Conference on Power Aware Computing and Systems, San Diego, CA, USA, 8--10 December 2008}; pp. 3--4.
	
	\bibitem[Mills(2014)]{Mills2014EnergySystems}
	Mills, B.; Znati, T.; Melhem, R.; Ferreira, K.B.; Grant, R.E. Energy Consumption of Resilience Mechanisms in Large Scale Systems. In Proceedings of the 2014 22nd Euromicro International Conference on Parallel, Distributed, and Network-Based Processing, Turin, Italy, 12--14 February 2014;  pp. 528--535.
	
	\bibitem[Feng(2003)]{Feng2003MakingSupercomputing}
	Feng, W.c. Making a Case for Efficient Supercomputing. {\em Queue} {\bf 2003}, {\em 1}, 54--64. [\href{http://dx.doi.org/10.1145/957717.957772}{CrossRef}]
	
	\bibitem[Sarwar(1997)]{Sarwar1997CmosCalculation}
	Sarwar, A. Cmos power consumption and cpd calculation. In {\em Proceeding: Design Considerations for Logic Products};  Texas Instruments: Dallas, TX, USA, {1997}.
	
	\bibitem[Butzen and Ribas(2007)]{Butzen2007LeakageGates}
	Butzen, P.; Ribas, R. {\em Leakage Current in Sub-Micrometer CMOS Gates}; {Universidade Federal do Rio Grande do Sul}: Porto Alegre, Brazil, { 2007}; pp. 1--30.
	
	\bibitem[Amdahl(1967)]{Amdahl1967ValidityCapabilities}
	Amdahl, G.M. {Validity of the single processor approach to achieving large scale computing capabilities}. In Proceedings of the Spring Joint Computer Conference on---AFIPS '67 (Spring), New York, NY, USA, 18--20 April 1967.
	
	\bibitem[Eyerman(2010)]{Eyerman2010ModelingDesign}
	Eyerman, S.; Eeckhout, L. {Modeling critical sections in Amdahl's law and its implications for multicore design}. In Proceedings of the 37th Annual International Symposium on Computer Architecture---ISCA '10, New York, NY, USA, 19--23 June 2010.
	
	\bibitem[Gustafson(1988)]{Gustafson1988ReevaluatingLaw}
	Gustafson, J.L. Reevaluating Amdahl's law. {\em Commun. ACM} {\bf 1988}, {\em 31}, 532--533. [\href{http://dx.doi.org/10.1145/42411.42415}{CrossRef}]
	
	\bibitem[Seel(2012)]{Hypothesis2012EncyclopediaLearning}
	Seel, N.M. \emph{Encyclopedia of the Sciences of Learning}; {Springer:  {Berlin/Heidelberg, Germany,} %newly added information, please confirm
	} {1988}; pp. 223--242.
	
	\bibitem[Roy(2019)]{Roy2019ForecastingNetwork}
	Roy, P.; Mahapatra, G.S.; Dey, K.N. Forecasting of software reliability using neighborhood fuzzy particle swarm optimization based novel neural network. {\em IEEE/CAA J. Autom. Sin.} {\bf 2019}, {\em 6}, 1365--1383. [\href{http://dx.doi.org/10.1109/JAS.2019.1911753}{CrossRef}]
	
	\bibitem[Zhu(2019)]{Zhu2019PredictingLearning}
	Zhu, W.; Liu, X.; Xu, M.; Wu, H. Predicting the results of RNA molecular specific hybridization using machine learning. {\em IEEE/CAA J. Autom. Sin.} {\bf 2019}, {\em 6}, 1384--1396. [\href{http://dx.doi.org/10.1109/JAS.2019.1911756}{CrossRef}]
	
	\bibitem[Rivoire(2008)]{Rivoire2008AModels}
	Rivoire, S.; Ranganathan, P.; Kozyrakis, C. A comparison of high-level full-system power models. {In Proceedings of the 2008 Conference on Power Aware Computing and Systems, San Diego, CA, USA, 8--10 December 2008}; pp. 1--5.
	
	\bibitem[Usman(2013)]{Usman2013ANoC}
	Usman, S.; Khan, S.U.; Khan, S. A comparative study of voltage/frequency scaling in NoC. {In Proceedings of the  IEEE International Conference on Electro-Information Technology, Rapid City, SD, USA, 9--11 May 2013}; pp. 1--5.
	
	\bibitem[Paolillo(2018)]{Paolillo2018OptimisationParallelism}
	Paolillo, A. Optimisation of Performance Metrics of Embedded Hard Real-Time
	Systems using Software/Hardware Parallelism, Ph.D. Thesis, Université libre de Bruxelles, Brussels, Belgium, 2018.
	
	\bibitem[Kim(2015)]{Kim2015RacingHeuristics}
	Kim, D.H.; Imes, C.; Hoffmann, H. Racing and Pacing to Idle: Theoretical and Empirical Analysis of
	Energy Optimization Heuristics. {In Proceedings of the  2015 IEEE 3rd International Conference on Cyber-Physical Systems, Networks, and Applications, Hong Kong, China, 19--21 August 2015}; pp. 78--85.
	
	\bibitem[Fu(2018)]{Fu2018RaceMinimization}
	Fu, C.; Chau, V.; Li, M.; Xue, C.J. Race to idle or not: Balancing the memory sleep time with DVS for energy minimization. {\em J. Comb. Optim.} {\bf 2018}, {\em 35}, 860--894. [\href{http://dx.doi.org/10.1007/s10878-017-0229-7}{CrossRef}]
	
	\bibitem[Rauber(2014)]{Rauber2014EnergyScaling}
	Rauber, T.; R{\"{u}}nger, G.; Schwind, M.; Xu, H.; Melzner, S. Energy measurement, modeling, and prediction for processors with frequency scaling. {\em J. Supercomput.} {\bf 2014}, {\em 70}, 1451--1476. [\href{http://dx.doi.org/10.1007/s11227-014-1236-4}{CrossRef}]
	
	\bibitem[Goel(2016)]{Goel2016AProcessors}
	Goel, B.; McKee, S.A. A Methodology for Modeling Dynamic and Static Power Consumption for Multicore Processors. {In Proceedings of the IEEE International Parallel and Distributed Processing Symposium, Chicago, IL, USA, 23--27 May 2016}; pp. 273--282.
	
	\bibitem[Du(2017)]{Du2017ModelingSystems}
	Du, Z.; Ge, R.; Lee, V.W.; Vuduc, R.; Bader, D.A.; He, L. Modeling the Power Variability of Core Speed Scaling on Homogeneous Multicore Systems. {\em Sci. Program.} {\bf 2017}, {\em 2017}, 1--13. [\href{http://dx.doi.org/10.1155/2017/8686971}{CrossRef}]
	
	\bibitem[Gonzalez(1997)]{Gonzalez1997SupplyCMOS}
	Gonzalez, R.; Gordon, B.; Horowitz, M. Supply and threshold voltage scaling for low power CMOS. {\em IEEE J. Solid-State Circuits} {\bf 1997}, {\em 32}, 1210--1216. [\href{http://dx.doi.org/10.1109/4.604077}{CrossRef}]
	
	\bibitem[Silva(2019)]{Silva2019Energy-OptimalApplications}
	Silva, V.R.G.; Furtunato, A.F.A.; Georgiou, K.; Sakuyama, C.A.V.; Eder, K.; Xavier-de Souza, S. Energy-Optimal Configurations for Single-Node HPC Applications. {In Proceedings of the  2019 International Conference on High Performance Computing \& Simulation (HPCS), Dublin, Ireland, 15--19 July 2019}; pp. 448--454.
	
	\bibitem[Kumar(1994)]{Kumar1994AnalyzingArchitectures}
	Kumar, V.; Gupta, A. Analyzing Scalability of Parallel Algorithms and Architectures. {\em J. Parallel  Distrib. Comput.} {\bf 1994}, {\em 22}, 379--391. [\href{http://dx.doi.org/10.1006/jpdc.1994.1099}{CrossRef}]
	
	\bibitem[Oliveira(2018)]{Oliveira2018ApplicationCharacterization}
	Oliveira, V.H.F.; Furtunato, A.F.A.; Silveira, L.F.; Georgiou, K.; Eder, K.; Xavier-de Souza, S. Application Speedup Characterization. {In Proceedings of the ACM/SPEC International Conference on Performance Engineering, Berlin, Germany, 9--13 April 2018}; pp. 43--44.
	
	\bibitem[Smola(2004)]{Smola2004ARegression}
	Smola, A.J.; Sch{\"{o}}lkopf, B. A tutorial on support vector regression. {\em Stat. Comput.} {\bf 2004}, {\em 14}, 199--222. [\href{http://dx.doi.org/10.1023/B:STCO.0000035301.49549.88}{CrossRef}]
	
	\bibitem[Kitts(2006)]{Kitts2006RegressionLecture}
	Kitts, B. Regression Trees Lecture. {\em Data Min.} {\bf 2006}, 6--7.
	
	\bibitem[Altman(1992)]{Altman1992AnRegression}
	Altman, N.S. An Introduction to Kernel and Nearest-Neighbor Nonparametric Regression. {\em  Am. Stat.} {\bf 1992}, {\em 46}, 175--185.
	
	\bibitem[Murtagh(1991)]{Murtagh1991MultilayerRegression}
	Murtagh, F. Multilayer perceptrons for classification and regression. {\em Neurocomputing} {\bf 1991}, {\em 2}, 183--197. [\href{http://dx.doi.org/10.1016/0925-2312(91)90023-5}{CrossRef}]
	
	\bibitem[Gao(2019)]{Gao2019DendriticPrediction}
	Gao, S.; Zhou, M.; Wang, Y.; Cheng, J.; Yachi, H.; Wang, J. Dendritic Neuron Model With Effective Learning Algorithms for Classification, Approximation, and Prediction. {\em IEEE Trans. Neural Netw. Learn. Syst.} {\bf 2019}, {\em 30}, 601--614. [\href{http://dx.doi.org/10.1109/TNNLS.2018.2846646}{CrossRef}]
	
	\bibitem[Schwenkler(2006)]{Schwenkler2006IntelligentInterface}
	Schwenkler, T.; Deutschland, S. Intelligent Platform Management Interface. In \emph{Sicheres Netzwerkmanagement}; Springer: Berlin/Heidelberg, Germany, 2006; pp. 169--207.
	
	\bibitem[Bienia(2008)]{Bienia2008TheSuite}
	Bienia, C.; Kumar, S.; Singh, J.P.; Li, K. The PARSEC benchmark suite. In Proceedings of the 17th international conference on Parallel architectures and compilation techniques---PACT '08, New York, NY, USA, 25--29 October 2008.
	
	\bibitem[Romano(2015)]{Romano2015OpenMC:Development}
	Romano, P.K.; Horelik, N.E.; Herman, B.R.; Nelson, A.G.; Forget, B.; Smith, K. OpenMC: A state-of-the-art Monte Carlo code for research and development. {\em Ann. Nucl. Energy} {\bf 2015}, {\em 82}, 90--97. [\href{http://dx.doi.org/10.1016/j.anucene.2014.07.048}{CrossRef}]
	
	\bibitem[Dongarra(1988)]{Dongarra1988TheExplanation}
	Dongarra, J.J. The LINPACK Benchmark: An explanation. In Proceedings of the 1st International Conference on Supercomputing, Athens, Greece, 8--12 June 1987.
	
	\bibitem[Pedregosa(2011)]{Pedregosa2011Scikit-learn:Python}
	Pedregosa, F; Varoquaux, G.; Gramfort, A.; Michel, V.; Thirion, B.; Grisel, O.; Blondel, M.;  Prettenhofer, P.; Weiss, R.; Dubourg, V.; et al. Scikit-learn: Machine Learning in {\{}P{\}}ython. {\em J. Mach. Learn. Res.} {\bf 2011}, {\em 12}, 2825--2830.
	
	\bibitem[Royer(2020)]{Royer2020AOptimization}
	Royer, C.W.; O’Neill, M.; Wright, S.J. A Newton-CG algorithm with complexity guarantees for smooth
	unconstrained optimization. {\em Math. Programm.} {\bf 2020}, {\em 180}, 451--488.
	
	\bibitem[Horyath(2008)]{Horyath2008Multi-mode}
	Tibor Horyath and Kevin Skadron.
	\newblock {Multi-mode energy management for multi-tier server clusters}.
	\newblock {\em Parallel Architectures and Compilation Techniques - Conference Proceedings, PACT}, 270--279, 2008.

	\bibitem[Demme(2013)]{Demme2013OnCounters}
	John Demme, Matthew Maycock, Jared Schmitz, Adrian Tang, Adam Waksman, Simha
	Sethumadhavan, and Salvatore Stolfo.
	\newblock {On the feasibility of online malware detection with performance
	counters}.
	\newblock {\em ACM SIGARCH Computer Architecture News}, 41(3):559, 2013.
	
	\bibitem[Eranian(2008)]{Eranian2008Perfmon2}
	Stephane Eranian.
	\newblock {Perfmon2: a standard performance monitoring interface for Linux}.
	\newblock {\em Slides, perfmon2 overview}, 2008.

	\bibitem[Hahnel(2012)]{Hahnel2012RAPL}
	Marcus H{\"{a}}hnel, Bj{\"{o}}rn D{\"{o}}bel, Marcus V{\"{o}}lp, and Hermann
	H{\"{a}}rtig.
	\newblock {Measuring energy consumption for short code paths using RAPL}.
	\newblock {\em ACM SIGMETRICS Performance Evaluation Review}, 40(3):13, 2012.

	\bibitem[Zamani(2012)]{Zamani2012ASystems}
	Reza Zamani and Ahmad Afsahi.
	\newblock {A study of hardware performance monitoring counter selection in
	power modeling of computing systems}.
	\newblock {\em 2012 International Green Computing Conference, IGCC 2012}, 2012.

	\bibitem[IPMI(2013)]{IPMI2013ConfigurationGuide}
	Published November.
	\newblock {IPMI Configuration User Guide}.
	\newblock 2017(November), 2013.

	\bibitem[Murtagh(2011)]{Murtagh2011WardsAlgorithm}
	Fionn Murtagh and Pierre Legendre.
	\newblock {Ward's Hierarchical Clustering Method: Clustering Criterion and
	Agglomerative Algorithm}.
	\newblock (June):1--20, 2011.

	\bibitem[Mucci(1999)]{Mucci1999PAPI}
	PJ~Mucci, Shirley Browne, Christine Deane, and George Ho.
	\newblock {PAPI: A portable interface to hardware performance counters}.
	\newblock {\em Proceedings of the department of defense HPCMP users group
	conference}, 32:7–10, 1999.

	\bibitem[Luo(2005)]{Luo2005PropertiesDifferentiators}
	Jianwen Luo, Kui Ying, Ping He, and Jing Bai.
	\newblock {Properties of Savitzky-Golay digital differentiators}.
	\newblock {\em Digital Signal Processing: A Review Journal}, 15(2):122--136,
	2005.

	\bibitem[Kufrin(2005)]{Kufrin2005Perfsuite}
	Rick Kufrin.
	\newblock {Perfsuite: An accessible, open source performance analysis
	environment for linux}.
	\newblock {\em Dans Presented at The 6th International Conference on Linux
	Clusters: The HPC Revolution}, 151(April):5, 2005.

	\bibitem[Knupfer(2011)]{Knupfer2011Scorep}
	Andreas Kn{\"{u}}pfer and Christian R{\"{o}}ssel.
	\newblock {Score-P – A Joint Performance Measurement Run-Time Infrastructure
	for}.
	\newblock 1--12, 2011.

	\bibitem[Jurman(2009)]{Jurman2009CanberraLists}
	Giuseppe Jurman, Samantha Riccadonna, Roberto Visintainer, and Cesare
	Furlanello.
	\newblock {Canberra distance on ranked lists}.
	\newblock {\em Proceedings, Advances in Ranking-NIPS 09 Workshop}, pages
	22--27, 2009.

	\bibitem[Hang(2017)]{Hang2017CubicApplications}
	Houjun Hang, Xing Yao, Qingqing Li, and Michel Artiles.
	\newblock {Cubic B-Spline Curves with Shape Parameter and Their Applications}.
	\newblock {\em Mathematical Problems in Engineering}, 2017:1--8, 2017.

	\bibitem[Intel(2013)]{Intel2013IntelGuide}
	{Intel}.
	\newblock {Intel{\textregistered} 64 and IA-32 Architectures Software
	Developer’s Manual, Volume 3 (3A, 3B {\&} 3C): System Programming Guide}.
	\newblock 3(253665):1--1386, 2013.
	
	
	%%%%%%%%%%%
	
	\bibitem[Huck(2007)]{Huck2007}
	Huck, K.; Malony, A.; Shende, S.; Morris, A.
	\newblock Scalable, Automated Performance Analysis with TAU and PerfExplorer.
	\newblock In Proceedings of the PARCO, Aachen, Germany, 4--7 September 
	2007; Volume~15, pp. 629--636.
	
	\bibitem[Islam(2019)]{Islam2019}
	Islam, T.; Ayala, A.; Jensen, Q.; Ibrahim, K.
	\newblock Toward a Programmable Analysis and Visualization Framework for
	Interactive Performance Analytics.
	\newblock In Proceedings of the IEEE/ACM International Workshop on Programming and Performance
	Visualization Tools (ProTools), Denver, CO, USA, 17 November 2019; pp. 70--77.
	\newblock [\href{http://doi.org/10.1109/ProTools49597.2019.00015}{CrossRef}]
	
	
	\bibitem[Weber(2019)]{Weber2019}
	\textls[-20]{Weber, M.; Ziegenbalg, J.; Wesarg, B.
		\newblock Online Performance Analysis with the Vampir Tool Set. In {\em Tools
			for High Performance Computing 2017, Proceedings of the 11th International Workshop on Parallel Tools for High Performance Computing, Dresden, Germany, 11--12 September, 2017
		}; Springer International Publishing: Cham, Switzerland,
		2019; pp. 129--143.
		\newblock
	} [\href{http://dx.doi.org/10.1007/978-3-030-11987-4_8}{CrossRef}]
	
	\bibitem[Bergel(2019)]{Bergel2019}
	\textls[-30]{Bergel, A.; Bhatele, A.; Boehme, D.; Gralka, P.; Griffin, K.; Hermanns, M.A.;
		Okanović, D.; Pearce, O.; Vierjahn, T. \emph{Visual Analytics Challenges in
			Analyzing Calling Context Trees}; Springer: Cham, Switzerland, 2019; pp. 233--249.
		\newblock
	} [\href{http://dx.doi.org/10.1007/978-3-030-17872-7_14}{CrossRef}]
	
	\bibitem[Malony(2005)]{Huck2005}
	Huck, K.; Malony, A.
	\newblock PerfExplorer: A Performance Data Mining Framework For Large-Scale
	Parallel Computing.
	\newblock In Proceedings of the SC '05: Proceedings of the 2005 ACM/IEEE Conference on
	Supercomputing, Seattle, WA, USA, 12--18 November 2005; p.~41.
	\newblock [\href{http://dx.doi.org/10.1109/SC.2005.55}{CrossRef}]
	
	
	\bibitem[Geimer(2010)]{Geimer2010}
	Geimer, M.; Wolf, F.; Wylie, B.; {\'A}brah{\'a}m, E.; Becker, D.; Mohr, B.
	\newblock The Scalasca performance toolset architecture.
	\newblock {\em Concurr. Comput. Pract. Exp.} {\bf
		2010}, {\em 22}, 702--719. [\href{http://dx.doi.org/10.1002/cpe.1556}{CrossRef}]
	
	\bibitem[Malony(2006)]{Shende2006}
	Shende, S.S.; Malony, A.D.
	\newblock The Tau Parallel Performance System.
	\newblock {\em Int. J. High Perform. Comput. Appl.} {\bf 2006}, {\em 20},~287--311.
	\newblock [\href{http://dx.doi.org/10.1177/1094342006064482}{CrossRef}]
	
	
	\bibitem[Adhianto(2010)]{Adhianto2010}
	Adhianto, L.; Banerjee, S.; Fagan, M.; Krentel, M.; Marin, G.; Mellor-Crummey,
	J.; Tallent, N.R.
	\newblock HPCTOOLKIT: Tools for performance analysis of optimized parallel
	programs.
	\newblock {\em Concurr. Comput. Pract. Exp.} {\bf 2010},
	{\em 22},~685--701.
	\newblock [\href{http://dx.doi.org/10.1002/cpe.1553}{CrossRef}]
	
	
	\bibitem[Miller(1995)]{Miller1995}
	Miller, B.; Callaghan, M.; Cargille, J.; Hollingsworth, J.; Irvin, R.;
	Karavanic, K.; Kunchithapadam, K.; Newhall, T.
	\newblock The Paradyn parallel performance measurement tool.
	\newblock {\em Computer} {\bf 1995}, {\em 28},~37--46.
	\newblock [\href{http://dx.doi.org/10.1109/2.471178}{CrossRef}]
	
	
	\bibitem[Galobardes(2015)]{Galobardes2015}
	Galobardes, E.C.
	\newblock \emph{Automatic Tuning of HPC Applications. The Periscope Tuning Framework};
	\newblock Shaker: Herzogenrath, Germany, 2015.
	
	\bibitem[Pillet(2007)]{Pillet2007}
	Pillet, V.; Labarta, J.; Cortes, T.; Girona, S.
	\newblock PARAVER: A Tool to Visualize and Analyze Parallel Code. In Proceedings of the WoTUG-18: Transputer and Occam Developments, Manchester, UK, 9--13 April 2007.
	
	
	\bibitem[Brink(2020)]{Brink2020}
	Brink, S.; Lumsden, I.; Scully-Allison, C.; Williams, K.; Pearce, O.; Gamblin,
	T.; Taufer, M.; Isaacs, K.E.; Bhatele, A.
	\newblock Usability and Performance Improvements in Hatchet.
	\newblock In Proceedings of the IEEE/ACM International Workshop on HPC User Support Tools
	(HUST) and Workshop on Programming and Performance Visualization Tools
	(ProTools), Atlanta, GA, USA, 18 November 2020; pp. 49--58.
	\newblock [\href{http://dx.doi.org/10.1109/HUSTProtools51951.2020.00013}{CrossRef}]
	
	
	\bibitem[Silva(2018)]{Silva2018}
	Silva, A.B.N.; Cunha, D.A.M.; Silva, V.R.G.; Furtunato, A.F.A.; Souza, S.X.-d.-S.
	\newblock PaScal Viewer: A Tool for the Visualization of Parallel Scalability
	Trends.
	\newblock In Proceedings of the ESPT/VPA@SC, Dallas, TX, USA, 11--16 November 2018.
	
	\bibitem[Eriksson(2016)]{Eriksson2016}
	Eriksson, J.; Ojeda-may, P.; Ponweiser, T.; Steinreiter, T.
	\newblock \emph{Profiling and Tracing Tools for Performance Analysis of Large Scale Applications};
	\newblock PRACE---Partnership for Advanced Computing in Europe: Brussels, Belgium, 2016; 
	pp. 1--30.
	\newblock Available online: \url{https://prace-ri.eu/wp-content/uploads/WP237.pdf} (accessed on 12 January 2020).
	
	
	\bibitem[Roberts(2017)]{10.1007/978-3-319-58667-0_22}
	Roberts, S.I.; Wright, S.A.; Fahmy, S.A.; Jarvis, S.A.
	\newblock Metrics for Energy-Aware Software Optimisation.
	\newblock In \emph{High Performance Computing, Proceedings of the 32nd International Conference, ISC High Performance 2017, Frankfurt, Germany, 18--22 June 2017}; Kunkel, J.M., Yokota, R., Balaji, P.,
	Keyes, D., Eds.; Springer International Publishing: Cham, Switzerland, 2017; pp.
	413--430.
	
	\bibitem[Eastep(2017)]{10.1007/978-3-319-58667-0_21}
	Eastep, J.; Sylvester, S.; Cantalupo, C.; Geltz, B.; Ardanaz, F.; Al-Rawi, A.;
	Livingston, K.; Keceli, F.; Maiterth, M.; Jana, S.
	\newblock Global Extensible Open Power Manager: A Vehicle for HPC Community
	Collaboration on Co-Designed Energy Management Solutions.
	\newblock In \emph{High Performance Computing, Proceedings of the 32nd International Conference, ISC High Performance 2017, Frankfurt, Germany, 18--22 June 2017}; Kunkel, J.M., Yokota, R., Balaji, P.,
	Keyes, D., Eds.; Springer International Publishing: Cham, Switzerland, 2017; pp.
	394--412.
	
	\bibitem[Hackenberg(2014)]{7016382}
	Hackenberg, D.; Ilsche, T.; Schuchart, J.; Schöne, R.; Nagel, W.E.; Simon, M.;
	Georgiou, Y.
	\newblock HDEEM: High Definition Energy Efficiency Monitoring.
	\newblock In Proceedings of the Energy Efficient Supercomputing Workshop, New Orleans, LA, USA, 16 November 2014; pp. 1--10.
	\newblock [\href{http://dx.doi.org/10.1109/E2SC.2014.13}{CrossRef}]
	
	
	\bibitem[Roberts(2019)]{10.1145/3321551}
	Roberts, S.I.; Wright, S.A.; Fahmy, S.A.; Jarvis, S.A.
	\newblock The Power-Optimised Software Envelope.
	\newblock {\em ACM Trans. Archit. Code Optim.} {\bf 2019}, {\em 16}, 1--27.
	\newblock [\href{http://dx.doi.org/10.1145/3321551}{CrossRef}]
	
	
	\bibitem[Boehme(2016)]{Boehme2016}
	Boehme, D.; Gamblin, T.; Beckingsale, D.; Bremer, P.T.; Gimenez, A.; LeGendre,
	M.; Pearce, O.; Schulz, M.
	\newblock Caliper: Performance Introspection for HPC Software Stacks.
	\newblock In Proceedings of the SC '16: Proceedings of the International Conference for High
	Performance Computing, Networking, Storage and Analysis, Salt Lake City, UT, USA, 13--18 November 2016; pp. 550--560.
	\newblock [\href{http://dx.doi.org/10.1109/SC.2016.46}{CrossRef}]
	
	
	\bibitem[Intel(2021)]{Intel2021Vtune}
	Corporation, I.
	\newblock \text{Intel VTune}.
	\newblock Available online: \url{https://software.intel.com/vtune} (accessed on 15 February 2020). 
	
	
	\bibitem[Houstis(1997)]{Pantazopoulos1997}
	Pantazopoulos, K.N.; Houstis, E.
	\newblock {Performance Analysis and Visualization Tools for Parallel Computing}.
	\newblock 1997.
	\newblock Available online: \url{https://docs.lib.purdue.edu/cstech/1346} (accessed on 20 May 2020).
	%MDPI: Please add the publisher and location.
	
	
	\bibitem[Ott(2010)]{Gerndt2010}
	Gerndt, M.; Ott, M.
	\newblock Automatic Performance Analysis with Periscope.
	\newblock {\em Concurr. Comput. Pract. Exp.} {\bf
		2010}, {\em 22},~736--748. [\href{http://dx.doi.org/10.1002/cpe.1551}{CrossRef}]
	
	\bibitem[Labarta(2005)]{Labarta2005}
	Labarta, J.; Gimenez, J.; Martínez, E.; González, P.; Servat, H.; Llort, G.;
	Aguilar, X.
	\newblock Scalability of Tracing and Visualization Tools. In Proceedings of the {International Conference ParCo}, Prague, Czech Republic, 10--13 September 2005;
	\newblock pp. 869--876.
	
	\bibitem[Bienia(2008)]{Bienia2008}
	Bienia, C.; Kumar, S.; Singh, J.P.; Li, K.
	\newblock {The PARSEC benchmark suite: Characterization and architectural
		implications}.
	\newblock {In Proceedings of the International Conference on Parallel
		Architectures and Compilation Techniques}, Toronto, ON, Canada, {25--29 October 2008}; pp. 72--81.
	\newblock [\href{http://dx.doi.org/10.1145/1454115.1454128}{CrossRef}]
	
	
	\bibitem[Alex(2020)]{Alex2020WhenParallelSpeedups}
	\newblock Furtunato, A.F.A.; Georgiou, K.; Eder, K.; Xavier-De-Souza, S.
	\newblock When Parallel Speedups Hit the Memory Wall.
	\newblock {\em IEEE Access} {\bf 2020}, {\em  8},  79225--79238. [\href{http://dx.doi.org/10.1109/ACCESS.2020.2990418}{CrossRef}]
	
	\bibitem[Vitor(2020)]{Vitor2022AnalyticalEnergyModel}
	\newblock Silva, V.R.G.; Valderrama, C.; Manneback, P.; Xavier-de-Souza, S.
	\newblock Analytical Energy Model Parametrized by Workload, Clock Frequency and Number of Active Cores for Share-Memory High-Performance Computing Applications.
	\newblock {\em  Energies} {\bf 2022}, {\em  15},  1213. [\href{http://dx.doi.org/10.3390/en15031213}{CrossRef}]

	%%%%%%

	\bibitem[Melo(2010)]{Melo2010Perf}
	A.~C. de~Melo, ``{The New Linux 'perf' Tools},'' {\em Linux Kongress}, 2010.
	
	\bibitem[Gonzalez(2021)]{Gonzalez2021PolyBench}Abella-Gonzalez, M., Carollo-Fernandez, P., Pouchet, L., Rastello, F. \& Rodrıiguez, G. PolyBench/Python: Benchmarking Python Environments with Polyhedral Optimizations. {\em Proceedings Of The 30th ACM SIGPLAN International Conference On Compiler Construction}. pp. 59-70 (2021)

	\bibitem[Nikolaev(2011)]{Nikolaev2011Perfctr}Nikolaev, R. \& Back, G. Perfctr-Xen: A Framework for Performance Counter Virtualization. {\em Proceedings Of The 7th ACM SIGPLAN/SIGOPS International Conference On Virtual Execution Environments}. pp. 15-26 (2011)

	%%%%%%%%%%% PHASES %%%%%%%%%%%
	
	\bibitem{Group2012HandbookSahni}
	I.~Ahmad and S.~Ranka, \emph{Handbook of Energy-Aware and Green
		Computing}.\hskip 1em plus 0.5em minus 0.4em\relax Chapman \& Hall/CRC, 2012.
	
	\bibitem{iea_2021}
	Iea, ``Data centres and data transmission networks – analysis,'' Nov 2021.
	[Online]. Available:
	\url{https://www.iea.org/reports/data-centres-and-data-transmission-networks}
	
	\bibitem{Corcoran2017EmergingICT}
	P.~Corcoran and A.~Andrae, ``{Emerging Trends in Electricity Consumption for
		Consumer ICT},'' no. July 2013, pp. 1--56, 2017.
	
	\bibitem{Mathew2012Energy-awareNetworks}
	V.~Mathew, R.~K. Sitaraman, and P.~Shenoy, ``{Energy-aware load balancing in
		content delivery networks},'' \emph{Proceedings - IEEE INFOCOM}, no.
	September 2011, pp. 954--962, 2012.
	
	\bibitem{Fan2007}
	X.~Fan, W.-D. Weber, and L.~A. Barr¨oso, ``{Power provisioning for a
		warehouse-sized computer},'' \emph{ACM SIGARCH Computer Architecture News},
	vol.~35, no.~2, p.~13, 2007.
	
	\bibitem{Barroso2007TheComputing}
	L.~A. Barroso and U.~H{\"{o}}lzle, ``{The case for energy-proportional
		computing},'' \emph{Computer}, vol.~40, no.~12, pp. 33--37, 2007.
	
	\bibitem{Malladi2012TowardsDRAM}
	K.~T. Malladi, F.~A. Nothaft, K.~Periyathambi, B.~C. Lee, C.~Kozyrakis, and
	M.~Horowitz, ``{Towards energy-proportional datacenter memory with mobile
		DRAM},'' \emph{Proceedings - International Symposium on Computer
		Architecture}, no. Figure 1, pp. 37--48, 2012.
	
	\bibitem{Rotem2012Power-managementBridge}
	E.~Rotem, A.~Naveh, A.~Ananthakrishnan, E.~Weissmann, and D.~Rajwan,
	``{Power-management architecture of the intel microarchitecture code-named
		Sandy Bridge},'' \emph{IEEE Micro}, vol.~32, no.~2, pp. 20--27, 2012.
	
	\bibitem{Brown2005}
	L.~Brown, R.~Moore, D.~S. Li, L.~Yu, A.~Keshavamurthy, and V.~Pallipadi,
	``{ACPI in Linux},'' \emph{Symposium A Quarterly Journal In Modern Foreign
		Literatures}, vol.~51, p.~51, 2005.
	
	\bibitem{Hackenberg2015}
	D.~Hackenberg, R.~Sch{\"{o}}ne, T.~Ilsche, D.~Molka, J.~Schuchart, and
	R.~Geyer, ``{An Energy Efficiency Feature Survey of the Intel Haswell
		Processor},'' \emph{Proceedings - 2015 IEEE 29th International Parallel and
		Distributed Processing Symposium Workshops, IPDPSW 2015}, pp. 896--904, 2015.
	
	\bibitem{CARDOSO201717}
	J.~M. Cardoso, J.~G.~F. Coutinho, and P.~C. Diniz, ``Chapter 2 -
	high-performance embedded computing,'' in \emph{Embedded Computing for High
		Performance}, J.~M. Cardoso, J.~G.~F. Coutinho, and P.~C. Diniz, Eds.\hskip
	1em plus 0.5em minus 0.4em\relax Boston: Morgan Kaufmann, 2017, pp. 17--56.
	[Online]. Available:
	\url{https://www.sciencedirect.com/science/article/pii/B9780128041895000028}
	
	\bibitem{Shuja2012Energy-efficientCenters}
	J.~Shuja, S.~A. Madani, K.~Bilal, K.~Hayat, S.~U. Khan, and S.~Sarwar,
	``{Energy-efficient data centers},'' \emph{Computing}, vol.~94, no.~12, pp.
	973--994, 2012.
	
	\bibitem{Benini2000AManagement}
	L.~Benini, A.~Bogliolo, and G.~De~Micheli, ``{A survey of design techniques for
		system-level dynamic power management},'' \emph{IEEE Transactions on Very
		Large Scale Integration (VLSI) Systems}, vol.~8, no.~3, pp. 299--316, 2000.
	
	\bibitem{Dayarathna2016DataSurvey}
	M.~Dayarathna, Y.~Wen, and R.~Fan, ``{Data center energy consumption modeling:
		A survey},'' \emph{IEEE Communications Surveys and Tutorials}, vol.~18,
	no.~1, pp. 732--794, 2016.
	
	\bibitem{Irani2007}
	S.~Irani, S.~Shukla, and R.~Gupta, ``Algorithms for power savings,'' \emph{ACM
		Transactions on Algorithms}, vol.~3, p.~41, 11 2007. [Online]. Available:
	\url{https://dl.acm.org/doi/10.1145/1290672.1290678}
	
	\bibitem{Poellabauer2005}
	C.~Poellabauer, L.~Singleton, and K.~Schwan, ``Feedback-based dynamic voltage
	and frequency scaling for memory-bound real-time applications,'' 2005, pp.
	234--243.
	
	\bibitem{Saha2012}
	S.~Saha and B.~Ravindran, ``An experimental evaluation of real-time dvfs
	scheduling algorithms.''\hskip 1em plus 0.5em minus 0.4em\relax ACM Press,
	2012, pp. 1--12. [Online]. Available:
	\url{http://dl.acm.org/citation.cfm?doid=2367589.2367604}
	
	\bibitem{Pietri2014}
	I.~Pietri and R.~Sakellariou, ``Energy-aware workflow scheduling using
	frequency scaling.''\hskip 1em plus 0.5em minus 0.4em\relax IEEE, 9 2014, pp.
	104--113. [Online]. Available:
	\url{http://ieeexplore.ieee.org/document/7103444/}
	
	\bibitem{Mashayekhy2014}
	L.~Mashayekhy, M.~M. Nejad, D.~Grosu, D.~Lu, and W.~Shi, ``Energy-aware
	scheduling of mapreduce jobs.''\hskip 1em plus 0.5em minus 0.4em\relax
	Institute of Electrical and Electronics Engineers Inc., 9 2014, pp. 32--39.
	
	\bibitem{Yousefi2018}
	M.~H.~N. Yousefi and M.~Goudarzi, ``A task-based greedy scheduling algorithm
	for minimizing energy of mapreduce jobs,'' \emph{Journal of Grid Computing},
	vol.~16, pp. 535--551, 12 2018.
	
	\bibitem{Kessler2021}
	C.~Kessler, S.~Litzinger, and J.~Keller, ``Crown-scheduling of sets of
	parallelizable tasks for robustness and energy-elasticity on many-core
	systems with discrete dynamic voltage and frequency scaling,'' \emph{Journal
		of Systems Architecture}, vol. 115, p. 101999, 5 2021. [Online]. Available:
	\url{https://linkinghub.elsevier.com/retrieve/pii/S1383762121000175}
	
	\bibitem{Ajmal2021}
	M.~S. Ajmal, Z.~Iqbal, F.~Z. Khan, M.~Bilal, and R.~M. Mehmood, ``Cost-based
	energy efficient scheduling technique for dynamic voltage and frequency
	scaling system in cloud computing,'' \emph{Sustainable Energy Technologies
		and Assessments}, vol.~45, p. 101210, 6 2021. [Online]. Available:
	\url{https://linkinghub.elsevier.com/retrieve/pii/S2213138821002204}
	
	\bibitem{Agrawal2021}
	P.~Agrawal and S.~Rao, ``Energy-efficient scheduling: classification, bounds,
	and algorithms,'' \emph{Sādhanā}, vol.~46, p.~46, 12 2021. [Online].
	Available: \url{http://link.springer.com/10.1007/s12046-021-01564-w}
	
	\bibitem{Schwenkler2006IntelligentInterface}
	T.~Schwenkler, ``{Intelligent Platform Management Interface},'' pp. 169--207,
	2006.
	
	\bibitem{Bienia2008TheSuite}
	C.~Bienia, S.~Kumar, J.~P. Singh, and K.~Li, ``{The PARSEC benchmark suite},''
	in \emph{Proceedings of the 17th international conference on Parallel
		architectures and compilation techniques - PACT '08}, no. January.\hskip 1em
	plus 0.5em minus 0.4em\relax New York, New York, USA: ACM Press, 2008, p.~72.
	[Online]. Available:
	\url{http://portal.acm.org/citation.cfm?doid=1454115.1454128}
	
	\bibitem{Romano2015OpenMC:Development}
	P.~K. Romano, N.~E. Horelik, B.~R. Herman, A.~G. Nelson, B.~Forget, and
	K.~Smith, ``{OpenMC: A state-of-the-art Monte Carlo code for research and
		development},'' \emph{Annals of Nuclear Energy}, vol.~82, pp. 90--97, 8 2015.
	[Online]. Available:
	\url{https://linkinghub.elsevier.com/retrieve/pii/S030645491400379X}
	
	\bibitem{Dongarra1988TheExplanation}
	J.~J. Dongarra, ``{The LINPACK Benchmark: An explanation},'' in \emph{Lecture
		Notes in Computer Science (including subseries Lecture Notes in Artificial
		Intelligence and Lecture Notes in Bioinformatics)}, 1988, vol. 297 LNCS, pp.
	456--474. [Online]. Available:
	\url{http://link.springer.com/10.1007/3-540-18991-2_27}
	
	\bibitem{electronics11050689}
	V.~R.~G. da~Silva, A.~B.~N. da~Silva, C.~Valderrama, P.~Manneback, and
	S.~Xavier-de Souza, ``A minimally intrusive approach for automatic assessment
	of parallel performance scalability of shared-memory hpc applications,''
	\emph{Electronics}, vol.~11, no.~5, 2022. [Online]. Available:
	\url{https://www.mdpi.com/2079-9292/11/5/689}

\end{thebibliography}